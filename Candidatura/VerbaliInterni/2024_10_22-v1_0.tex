\documentclass[a4paper, 12pt]{article}

\usepackage{custom}


%--------------------VARIABILI--------------------
\def\lastversion{v1.0}
\def\title{Verbale interno}
\def\date{22 ottobre 2024}
%-------------------------------------------------

\begin{document}

\primapagina

\begin{registromodifiche}
        v1.0 & 27 ottobre 2024 & Pedro Leoni & & Approvazione documento \\
    \hline
        v0.1 & 27 ottobre 2024 & Eghosa Matteo Igbinedion Osamwonyi & Marko Peric & Prima stesura del documento \\ 
    \hline
\end{registromodifiche}

\tableofcontents

\newpage
\section{Registro presenze}
   \begin{itemize}
        \item[] \textbf{Data}: 22 ottobre 2024
        \item[] \textbf{Ora inizio}: 18:00
        \item[] \textbf{Ora fine}: 19:30
        \item[] \textbf{Piattaforma}: Discord	
    \end{itemize}
\begin{table}[!h]
    \centering
    {\renewcommand{\arraystretch}{2}
    \begin{tabularx}{\textwidth}{| X | X |}
        \hline
            \textbf{\large Componente} & 
            \textbf{\large Presenza} \\ 
        \hline
        \hline
            Eghosa Matteo Igbinedion Osamwonyi& Presente \\
        \hline 
            Guirong Lan & Presente \\
        \hline 
            Enrico Bianchi& Presente \\
        hline 
            Francesco Savio& Presente \\
        \hline 
            Marko Peric& Presente \\
        \hline 
            Pedro Leoni& Presente \\
        \hline 
            
    \end{tabularx}}
\end{table}

\newpage

\section{Verbale}
Il gruppo ha discusso sul workflow da adottare per il completamento dei requisiti minimi per la candidatura a uno dei capitolati, optando per l’uso di \textit{gitflow} come modello di sviluppo. Questo modello garantisce un flusso di lavoro ordinato e scalabile, riducendo i conflitti e migliorando il controllo sulle modifiche del codice. Il team ha deciso di implementare i branch principali \textit{main}, \textit{develop} e \textit{feature} per gestire rispettivamente il codice stabile, lo sviluppo integrato e le singole funzionalità.

È stato inoltre concordato di utilizzare \textit{Issues}, \textit{Milestone} e \textit{Tables} nella repository per tracciare e assegnare compiti e responsabilità, monitorare i progressi e creare compiti aggiuntivi. Ogni membro del gruppo sarà responsabile della gestione delle issue legate ai propri compiti, garantendo trasparenza nei progressi.

Durante il brainstorming, il team ha stilato una lista di domande per le riunioni con le aziende riguardanti i capitolati C5, C7 e C9. Queste includono domande generali e tecniche, utili a comprendere meglio requisiti e tecnologie necessarie, nonché dettagli sugli ambienti di sviluppo e piattaforme richieste.

\section{To Do}
Il gruppo ha identificato le seguenti attività da completare:
\begin{itemize}
    \item Creazione di un template in \LaTeX{} per i documenti di gruppo, da completare
    \item Preparazione di domande mirate per chiarire i requisiti e le tecnologie dei capitolati C5 C7 e C9, sia a livello generale sia per i dettagli specifici relativi ad ogni azienda.
    \item Organizzazione di una prossima riunione per discutere le risposte ricevute dalle aziende e definire i passi successivi in base agli approfondimenti ottenuti.
\end{itemize}
\end{document}
