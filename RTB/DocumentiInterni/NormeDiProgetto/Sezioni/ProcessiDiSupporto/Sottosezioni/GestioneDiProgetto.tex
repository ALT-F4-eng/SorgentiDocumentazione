\subsection{Gestione di progetto}
\label{subsec:gestione_progetto}

\subsubsection{SCRUM}
Il gruppo ha deciso di utilizzare alcuni concetti del \glossario{framework} di gestione di progetto \glossario{agile} chiamato \glossario{SCRUM}.
\glossario{SCRUM} richiede l'utilizzo di un \glossario{modello di sviluppo a periodi} chiamati \textbf{\glossario{sprint}}.
Il gruppo ha deciso di utilizzare \glossario{sprint} della durata di una settimana(da giovedì a giovedì) per i seguenti motivi:
\begin{itemize}
    \item Aumenta il coinvolgimento dei membri del gruppo.
    \item Ottiene \glossario{feedback} continui riducendo la gravità di eventuali problematiche.
\end{itemize}
Il \glossario{framework} definisce un insieme di:
\begin{enumerate}
    \item \textbf{\glossario{Artefatti}}: informazioni e strumenti a supporto delle \glossario{cerimonie}.
   
    \item \textbf{Ruoli}: descrivono le responsabilità chiave dei membri del gruppo di lavoro.
    
    \item \textbf{\glossario{cerimonie}}: incontri interni al gruppo eseguite in momenti precisi del \glossario{ciclo di vita} del software.
\end{enumerate}
Il gruppo come detto non segue il \glossario{framework} alla lettera ma prende spunto da alcuni dei suoi elementi.
\glossario{SCRUM} è stato scelto perché è molto flessibile e molto utilizzato. 

\subsubsection{Ruoli}
\label{subsubsec:ruoli}
Il gruppo seguendo le regole di progetto abbraccia una definizione e distinzione più rigida dei ruoli dei membri del gruppo rispetto a quella data da \glossario{SCRUM}.  
Di seguito vengono elencate le definizioni dei ruoli e la regola di rotazione.

\paragraph{Responsabile}
Il responsabile del progetto è incaricato di coordinare le attività del gruppo di lavoro, pianificare e monitorare i progressi, e gestire efficacemente le risorse disponibili. In sintesi, egli si assicura che il progetto venga portato a termine nei tempi stabiliti e in conformità con le risorse assegnate.


\paragraph{Amministratore}
L'amministratore di progetto ha il compito di gestire il \glossario{way of working} del gruppo e gli strumenti a supporto dello stesso.

\paragraph{Analista}
L'analista è responsabile dell’analisi delle funzionalità del software, definendo i requisiti e i casi d’uso pertinenti.

\paragraph{Progettista}
Il progettista è responsabile della definizione dell’architettura del software, identificando le componenti e le relazioni tra di esse, sulla base dei requisiti stabiliti dall’analista. 

\paragraph{Programmatore}
Il programmatore si occupa di scrivere il codice sorgente del software, seguendo le specifiche elaborate dal progettista.

\paragraph{Verificatore}
Il verificatore di un progetto software ha il compito di garantire che il software prodotto e la documentazione associata siano conformi alle normative e alle specifiche definite. 

\paragraph{Rotazione dei ruoli}
Per aumentare la produttività il gruppo ha deciso di preassegnare solo il ruolo di responsabile.
Ogni richiesta di modifica pianificata durante lo \glossario{sprint} specificherà il ruolo per il quale è competenza svolgerla.
Di conseguenza, il membro che si auto-assegnerà la relativa \glossario{issue} assumerà il ruolo a essa associata per il tempo necessario al suo compimento.
Così facendo è possibile massimizzare l'efficienza del gruppo, infatti ogni membro ha la possibilità di ottimizzare il tempo a disposizione realizzando le richieste di modifica libere.


\subsubsection{Artefatti}
Le \glossario{cerimonie} usate dal gruppo sono: \glossario{product backlog} e \glossario{sprint backlog}. 
Di seguito vengono spiegati questi \glossario{Artefatti} indicando:
\begin{itemize}
    \item Descrizione.
    \item Scopo.
\end{itemize}

\paragraph{Product backlog}
Il \glossario{product backlog} è una \glossario{Kanban board} che contiene tutte le richieste di modifica che riguardano il progetto.
Le colonne del \glossario{product backlog} come implementato dal gruppo sono: "To Do", "In Progress", "To Review", "In Review", "Re Open" e "Done".
Il significato delle colonne viene spiegato alla sezione \hyperref[par:ciclo_vita_richieste_di_modifica]{\glossario{ciclo di vita} richieste di modifica}.
L'implementazione concreta è indicata alla sezione \hyperref[subpar:project]{\glossario{GitHub projects}}.

\paragraph{Sprint backlog}
\glossario{Kanban board} uguale alla \glossario{product backlog} con la differenza che contiene le richieste di modifica dello \glossario{sprint} corrente.
L'implementazione concreta è indicata alla sezione \hyperref[subpar:project]{\glossario{GitHub projects}}.

\subsubsection{Cerimonie}
Le \glossario{cerimonie} usate dal gruppo sono: retrospettiva, revisione e pianificazione. 
Di seguito vengono spiegate le \glossario{cerimonie} indicando:
\begin{itemize}
    \item Momento in cui viene eseguita.
    \item Membri del gruppo coinvolti.
    \item Scopo.
    \item \glossario{Artefatti} o documenti influenzati.
\end{itemize}

\paragraph{Retrospettiva}
Riunione eseguita a fine \glossario{sprint}, coinvolge tutto il gruppo e ha lo scopo di discutere le problematiche riscontrate.
I responsabili designati per lo \glossario{sprint} appena terminato devono riportare le problematiche riscontrate nella sezione retrospettiva dello \glossario{sprint} nel documento piano di progetto.


\paragraph{Revisione}
Eseguita a fine \glossario{sprint}, coinvolge i Responsabili designati per lo \glossario{sprint}.
Questa cerimonia produce la modifica del \textbf{piano di progetto} eseguendo:
\begin{enumerate}
    \item Per ogni richiesta di modifica facente parte della \glossario{sprint} board indicare nella sezione consuntivo dello \glossario{sprint} corrente:
    \begin{itemize}
        \item Identificativo.
        \item Figura professionale che lo ha preso in carico.
        \item Risorse utilizzate.
        \item Confronto tra risorse utilizzate e quelle preventivate.
        \item Stato in cui si trova.
    \end{itemize}
    \item Aggiornamento delle risorse rimaste nella relativa sezione dello \glossario{sprint} corrente.
    \item Modifica dei rischi nel piano di progetto.
    \item Documentare gli eventuali rischi riscontrati e l'efficacia delle azioni di mitigazione applicate.
    \item Analisi ed eventuale modifica del preventivo dei costi.
\end{enumerate}

\paragraph{Pianificazione}
La pianificazione è la cerimonia principale, viene realizzata all’inizio di ogni \glossario{sprint} e coinvolge l'intero gruppo di lavoro.
Lo scopo è quello di aggiornare il \glossario{product backlog} e definire quali richieste di modifica in esso contenute devono essere poste nello \glossario{sprint backlog}.
La pianificazione delle attività segue la seguente mappatura:
\begin{enumerate}
    \item \textbf{Identificazione delle attività}.
    
    Durante questa fase il gruppo guidato dal responsabile ha il compito di valutare l’avanzamento del progetto identificando le nuove richieste di modifica.

    \item \textbf{Analisi delle attività}.
    
    Il responsabile deve associare a ogni richiesta di modifica il ruolo che deve implementarla tra i ruoli definiti nella sezione \hyperref[subsubsec:ruoli]{Ruoli}.
    Il responsabile deve anche associare a ogni richiesta di modifica un valore di priorità in una scala da 1(minimo) a 3(massimo) e una stima delle ore previste per il loro completamento.
    Se una richiesta di modifica richiede una quantità di ore che supera le 8 è necessario suddividerla in più richieste di modifica.
    Il risultato della identificazione e della analisi delle attività è un documento condiviso realizzato su Google Docs dove per ogni attività ne vengono indicate le seguenti caratteristiche:
    \begin{itemize}
        \item File coinvolti.
        \item Descrizione a punti del lavoro da svolgere.
        \item Ruolo di competenza.
        \item Priorità.
        \item Ore stimate per l'implementazione.
    \end{itemize}

    \item \textbf{Modifica product backlog}.
    
    Le richieste di modifica risultanti dalla precedente fase vengono registrate nel \glossario{product backlog}.
    Questa operazione viene implementata usando le informazioni indicate alla sezione \hyperref[subpar:ITS]{\glossario{Issue}}.


    \item \textbf{Pianificazione \glossario{sprint}}.
    
    Questa fase consiste nella realizzazione di un piano da seguire durante lo svolgimento dello \glossario{sprint} successivo.
    Il responsabile deve:
    \begin{enumerate}
        \item \textbf{Calcolo ore disponibili per lo \glossario{sprint}}.
        
        Il responsabile deve chiedere a ogni  membro le ore che può dedicare al successivo \glossario{sprint}.
        Così facendo riesce a calcolare le ore totali produttive che il gruppo si impegna a dedicare nella settimana a venire.

        \item \textbf{Calcolo ore totali svolgimento attività}.
        
        Il responsabile deve selezionare il maggior numero di attività con priorità maggiore che possono essere portate a termine nel prossimo \glossario{sprint}.
    \end{enumerate}

    \item \textbf{Popolazione \glossario{sprint backlog}}.
    
    Popolare lo \glossario{sprint backlog} usando le richieste di modifica risultanti dalla precedente fase.
    Il popolamento viene fatto usando le informazioni indicate alla sezione \hyperref[subpar:ITS]{\glossario{Issue}}.
\end{enumerate}

\paragraph{Controllo}
Il responsabile ha il compito di mantenersi in contatto con i membri del gruppo durante ogni \glossario{sprint}.
In particolare deve monitorare giornalmente lo stato di avanzamento del lavoro prefissato contattando tutti i membri tramite i canali di comunicazione interni indicati nella sezione \hyperref[subpar:canali_interni]{Canali di comunicazione interni}.
Ogni giorno il responsabile deve verificare di quali attività si è già iniziato lo svolgimento e monitorarne lo stato di avanzamento.
Il responsabile deve:
\begin{enumerate}
    \item \textbf{Aggiornare la pianificazione \glossario{sprint}}.
    
    Se di uno o più membri del gruppo notificano la necessità di più ore di lavoro rispetto alla previsione per terminare un'attività, il responsabile se lo ritiene necessario, deve aggiornare la pianificazione dello \glossario{sprint}.  

    Per fare ciò il responsabile deve richiedere ai membri del gruppo le ore in "eccedenza" necessarie per poi:
    \begin{enumerate}
        \item Identificare le attività di priorità minima che devono ancora essere assegnate a un membro.
        \item Valutare il numero di attività che devono essere posticipate allo \glossario{sprint} successivo, basandosi sulle ore di lavoro totali disponibili rimaste e il preventivo di tempo necessario al completamento delle attività ancora da svolgere.
        \item Aggiornare lo \glossario{sprint backlog} togliendo le attività che dovranno essere posticipate.
    \end{enumerate}
    In questo modo si porteranno a termine il maggior numero di attività possibili.

    \item \textbf{Controllo dei rischi}.
    
    Il responsabile deve monitorare che i rischi individuati tramite l'attività di \hyperref[]{Gestione dei rischi} vengano gestiti correttamente.
    
    Il responsabile deve rendersi conto del verificarsi dei rischi.
    Per fare ciò deve:
    \begin{enumerate}
        \item Tenersi in contatto con i membri del gruppo per assicurarsi dell'assenza di rischi.
        \item Conoscere le strategie di rilevazione di ogni rischio che devono essere applicate giornalmente.
    \end{enumerate}

    Nel caso in cui un rischio si presenti il responsabile deve attuare nel minor tempo possibile tutte le attività di mitigazione specifiche indicate per il rischio stesso.
    Le attività di mitigazione possono richiedere il coinvolgimento con uno o più membri del gruppo.
    In tal caso il responsabile deve accertarsi che le attività vengano applicate correttamente. 
\end{enumerate}

\subsection{Gestione dei rischi}
Lo standard \glossario{ISO 31000} è uno standard internazionale per la gestione dei rischi, progettato per essere applicabile  in qualsiasi tipo di contesto, compresa la progettazione software.
Nello specifico fornisce un quadro strutturato per la gestione dei rischi, composto dalle seguenti fasi principali: 
\begin{itemize}
    \item \textbf{definizione del contesto}: comprendere l'ambiente esterno e interno in cui si sta operando e gli obiettivi del sistema.
    \item \textbf{identificazione dei rischi}: individuare i rischi e i fattori che potrebbero compromettere il raggiungimento degli obiettivi e dei requisiti necessari allo sviluppo.
    \item \textbf{analisi dei rischi}:  comprendere la natura dei rischi, compresa la causa per cui potrebbero verificarsi e la probabilità, le conseguenze sul sistema e sull'attività di lavoro.
    \item \textbf{ponderazione dei rischi}: valutare la gravità che comporterebbe il verificarsi di un rischio, identificare se è accettabile il verificarsi di uno o più rischi e quali invece dovrebbero richiedere un trattamento immediato.
    \item \textbf{trattamento dei rischi}: identificare i ruoli che hanno responsabilità nel trattamento dei rischi, i processi di mitigazione e le azioni da svolgere, nel caso dovesse verificarsi un rischio, per ridurre l'impatto sul sistema e sullo svolgimento del prodotto software. 
    \item \textbf{monitoraggio}: identificazione delle metodologie che permettono un continuo monitoraggio e rilevamento dei rischi, identificazione anche di eventuali nuovi rischi con conseguente aggiornamento della documentazione relativa alla analisi dei rischi.
    \item \textbf{comunicazione}: indicare strategie di comunicazione e ruoli coinvolti a seguito del rilevamento di un rischio indicato all'interno dell'analisi dei rischi.
\end{itemize}
