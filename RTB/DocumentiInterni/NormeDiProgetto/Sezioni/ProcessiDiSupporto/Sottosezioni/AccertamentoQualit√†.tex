\subsection{Accertamento della Qualità}
\label{subsec:accertamento_qualità}
Il processo di accertamento della qualità ha l'obiettivo di garantire che il progetto rispetti gli standard e i requisiti definiti, sia per quanto riguarda i prodotti sviluppati che per i processi adottati. 
Questo processo rappresenta un elemento centrale nella gestione del progetto, in quanto assicura il monitoraggio continuo e la verifica della conformità alle specifiche tecniche, agli obiettivi di progetto e alle normative applicabili.
Il processo si basa su principi fondamentali quali tracciabilità, trasparenza e miglioramento continuo, e utilizza metriche e strumenti oggettivi 
per valutare sia l'efficacia dei processi che la qualità dei prodotti. 
Attraverso una pianificazione accurata e un monitoraggio costante, il processo di accertamento della qualità contribuisce a ridurre i rischi, migliorare le performance e soddisfare le aspettative degli stakeholder.
\subsubsection{Attività}
Per l'accertamento della qualità, il gruppo adotta il ciclo di Deming, o PDCA (Plan-Do-Check-Act), un approccio iterativo e continuo volto a migliorare i processi e garantire il raggiungimento degli obiettivi di qualità.
Il Ciclo di Deming è composto da 4 fasi fondamentali:
\begin{itemize}
    \item \textbf{Plan}: In questa fase, si definiscono gli obiettivi di qualità e le strategie per garantirne il raggiungimento.
    È essenziale pianificare le attività che saranno svolte durante il ciclo di vita del progetto, stabilendo le metriche da monitorare e i criteri di accettazione.
    \item \textbf{Do}: In questa fase, le attività pianificate vengono effettivamente eseguite.
    Si mettono in atto le procedure di controllo qualità stabilite per monitorare la qualità del progetto in fase di sviluppo.
    \item \textbf{Check}: Questa fase consiste nel confrontare i risultati ottenuti con gli obiettivi di qualità definiti.
    Si verificano i dati raccolti durante la fase "Do" per determinare se i processi e i risultati sono allineati con gli standard di qualità e i requisiti.
    \item \textbf{Act}: In questa fase finale, vengono intraprese azioni correttive e preventive in base ai risultati della fase "Check".
\end{itemize}
\subsubsection{Standard di riferimento per la qualità di prodotto del software}
Per la valutazione della qualità del software con conseguente stesura delle metriche viene seguita la normativa ISO/IEC 9126, secondo lo standard le qualità sono suddivise in:
\begin{itemize}
    \item \textbf{qualità esterne}: misurano i comportamenti del software durante la sua esecuzione
    \item \textbf{qualità interne}: si applicano al software non eseguibile, permettono di individuare eventuali problemi che potrebbero influire sulla qualità finale del prodotto prima che sia realizzato il software eseguibile
    \item \textbf{qualità in uso}: rappresentano il punto di vista dell'utente sul software, permettono di stabilire i seguenti obiettivi:
    \begin{itemize}
        \item \textsl{Efficacia}: capacità del software di permettere agli autenti di raggiungere gli obiettivi specificati con accuratezza e completezza
        \item \textsl{Produttività}: capcaità del software di permettere agli utenti spendere una quantità opportuna di risorse in relazione all'efficacia ottenuta
        \item \textsl{Soddisfazione}: capacità del software si soddisfare gli utenti
        \item \textsl{Sicurezza}: capacità del software di raggiungere livelli accettabili di rischi di danni a persone e apparecchiature
    \end{itemize}
\end{itemize}

Lo standard normativo stabilisce un modello definendo un set di caratteristiche che consentono di misurare e valutare diversi aspetti della qualità del prodotto software:
\begin{itemize}
    \item \textbf{Funzionalità}: capacità del software di fornire le funzioni adatte a soddisfare le esigenze stabilite
    \item \textbf{Affidabilità}: capacità del software di mantenere un livello di prestazioni specifico quando usato in date condizioni per un dato periodo di tempo
    \item \textbf{Efficienza}: capacità del software di fornire adeguate prestazioni relativamente alla quantità di risorse utilizzate
    \item \textbf{Usabilità}: capacità del software di essere propriamente compreso ed utilizzato dall'utente 
    \item \textbf{Manutenibilità}: capacità del software di essere modificato per introdurre migliorie o adattamenti
    \item \textbf{Portabilità}: capacità del software di essere di essere trasportato da un ambiente di lavoro ad un altro
\end{itemize}
\subsubsection{Notazione Metriche di Qualità}
Ogni metrica è identificata in un modo univoco seguendo la notazione: \textsl{M}.[\textsl{Tipo}].[\textsl{Abbreviazione Nome}]
Dove:
\begin{itemize}
    \item \textbf{M}: indica "metrica"
    \item \textbf{Tipo}: sarà PC per indicare che la metrica misura la qualità di un processo o PR per indicare che la metrica misura la qualità di un prodotto
    \item \textbf{Abbreviazione Nome}: verrà inserito l'acronimo basato sul nome completo della metrica  
\end{itemize}
\subsubsection{Descrizione Metriche di Qualità}
Le metriche sono descritte tramite i seguenti campi:
\begin{itemize}
    \item \textbf{Notazione}: seguendo le indicazioni sopra elencate
    \item \textbf{Nome}: nome completo della metrica 
    \item \textbf{Descrizione}: descrizione della metrica 
    \item \textbf{caratteristiche}: presente unicamente nelle metriche di prodotto, indica a quale caratteristica, tra quelle indicate nella sezione relativa allo standard di riferimento, fa riferimento la metrica
    \item \textbf{Formula di misurazione}: formula per la misurazione del valore della metrica 
\end{itemize}
Verranno poi indicati all'interno del documento "Piano di Qualifica" i valori tollerabili e i valori ottimali per ogni metrica e, nel caso delle metriche di processo, anche il processo a cui fanno riferimento.
\subsubsection{Elenco delle Metriche di Qualità} 
L'elenco delle metriche di qualità da adottare per il progetto è dettagliato secondo la struttura definita e copre diverse aree rilevanti per il processo di accertamento della qualità. Le metriche includono:
\begin{itemize}
    \item \textbf{Planned Value (1M.PC.PV):} Valore pianificato per monitorare il progresso del progetto.
    \item \textbf{Earned Value (2M.PC.EV):} Valore guadagnato, rappresenta il valore del lavoro effettivamente completato.
    \item \textbf{Actual Cost (3M.PC.AC):} Costo effettivo sostenuto per il lavoro completato.
    \item \textbf{Schedule Variance (4M.PC.SV):} Differenza tra il valore guadagnato e il valore pianificato.
    \item \textbf{Cost Variance (5M.PC.CV):} Differenza tra il valore guadagnato e il costo effettivo.
    \item \textbf{Cost Performance Index (6M.PC.CPI):} Indice di prestazione dei costi, calcolato come rapporto tra il valore guadagnato e il costo effettivo.
    \item \textbf{Schedule Performance Index (7M.PC.SPI):} Indice di prestazione della schedulazione, calcolato come rapporto tra il valore guadagnato e il valore pianificato.
    \item \textbf{Estimate at Completion (8M.PC.EAC):} Stima dei costi totali alla fine del progetto.
    \item \textbf{Estimate to Complete (9M.PC.ETC):} Stima dei costi necessari per completare il lavoro rimanente.
    \item \textbf{On Time Delivery Rate (10M.PC.OTDR):} Tasso di consegna puntuale, misura la percentuale di consegne effettuate nei tempi previsti.
    \item \textbf{Percentuale requisiti obbligatori (11M.PC.PRO):} Percentuale di requisiti obbligatori implementati rispetto al totale.
    \item \textbf{Percentuale requisiti desiderabili (12M.PC.PRD):} Percentuale di requisiti desiderabili implementati rispetto al totale.
    \item \textbf{Percentuale requisiti opzionali (13M.PC.PRO):} Percentuale di requisiti opzionali implementati rispetto al totale.
    \item \textbf{Profondità delle Gerarchie (14M.PR.PG):} Numero massimo di livelli gerarchici presenti in una struttura o un sistema.
    \item \textbf{Parametri Per Metodo (15M.PR.PPM):} Numero medio di parametri passati ai metodi. Un numero elevato di parametri può indicare che un metodo è troppo complesso o che potrebbe essere suddiviso in metodi più piccoli.
    \item \textbf{Campi Per Classe (16M.PR.CPC):} Numero medio di campi (variabili di istanza) per classe. Un numero elevato di campi dati può indicare che una classe sta facendo troppo e che potrebbe essere suddivisa in classi più piccole.
    \item \textbf{Linee di Codice Per Metodo (17M.PR.LCPM):} Numero medio di linee di codice per metodo. Metodi troppo lunghi possono essere difficili da leggere, capire e mantenere.
    \item \textbf{Complessità Ciclomatica (18M.PR.CCM):} La complessità ciclomatica rappresenta la complessità di un programma sulla base del numero di percorsi lineari indipendenti attraverso il codice sorgente. Un valore elevato indica un codice più complesso e potenzialmente più difficile da mantenere.
    \item \textbf{Indice Gulpease (19M.PR.IG):} L'indice Gulpease è un indice di leggibilità di un testo tarato sulla lingua italiana. Misura la lunghezza delle parole e delle frasi rispetto al numero di lettere.
    \item \textbf{Correttezza Ortografica (20M.PR.CO):} La correttezza ortografica indica la presenza di errori ortografici nei documenti.
    \item \textbf{Facilità di Utilizzo (21M.PR.FU):} Rappresenta il livello di usabilità del prodotto software mediante il numero di errori riscontrati durante l'utilizzo del prodotto da parte di un utente generico.
    \item \textbf{Tempo di Apprendimento (22M.PR.TA):} Indica il tempo massimo richiesto da parte di un utente generico per apprendere l’utilizzo del prodotto.
    \item \textbf{Tempo di Risposta (23M.PR.TR):} Indica il tempo massimo di risposta del sistema sotto carico rilevato.
    \item \textbf{Tempo di Elaborazione (24M.PR.TE):} Indica il tempo massimo di elaborazione di un dato grezzo fino alla sua presentazione rilevato.
    \item \textbf{Metriche di Qualità Soddisfatte (25M.PC.QMS):} Indica il numero di metriche implementate e soddisfatte, tra quelle definite.
    \item \textbf{Code Coverage (26M.PC.CC):} La Code Coverage indica quale percentuale del codice sorgente è stata eseguita durante i test. Serve per capire quanto del codice è stato verificato dai test automatizzati.
    \item \textbf{Branch Coverage (27M.PC.BC):} La Branch Coverage indica quale percentuale dei rami decisionali (percorsi derivanti da istruzioni condizionali come if, for, while) del codice è stata eseguita durante i test.
    \item \textbf{Statement Coverage (28M.PC.SC):} La Statement Coverage indica quale percentuale di istruzioni del codice è stata eseguita durante i test.
    \item \textbf{Failure Density (29M.PC.FD):} La Failure Density indica il numero di difetti trovati in un software o in una parte di esso durante il ciclo di sviluppo rispetto alla dimensione del software stesso.
    \item \textbf{Passed Test Case Percentage (30M.PC.PTCP):} La Passed Test Case Percentage indica la percentuale di test che sono stati eseguiti con successo su una base di test.
    \item \textbf{Risk Mitigation Rate (31M.PC.RMR):} La Risk Mitigation Rate indica la percentuale di rischi identificati che sono stati mitigati con successo.
    \item \textbf{Rischi Non Calcolati (32M.PC.NCR):} Indica il numero di rischi occorsi che non sono stati preventivati durante l’analisi dei rischi.
    \item \textbf{Requirements Stability Index (33M.PC.RSI):} Il Requirements Stability Index (RSI) indica la percentuale di requisiti che sono rimasti invariati rispetto al totale dei requisiti inizialmente definiti. Si tratta di una metrica utilizzata per misurare quanto i requisiti di un progetto rimangono stabili durante il ciclo di vita del progetto stesso, ed è particolarmente utile per comprendere l’impatto delle modifiche ai requisiti sul progetto.
    \item \textbf{Variazione del budget tra preventivo e consuntivo (34M.PC.BV):} Misura la variazione tra il costo pianificato e il costo effettivo di un progetto alla data corrente. Permette di valutare con che velocità il team sta spendendo il proprio budget rispetto al preventivo. Formula: $Variazione = \frac{Cp - Ca}{Cp} \times 100$, dove $Cp$ è il costo pianificato e $Ca$ è il costo attuale.
    \item \textbf{Variazione del piano tra preventivo e consuntivo (35M.PC.PV):} Misura la variazione tra il numero di task pianificati e quelli completati entro un certo periodo di tempo.
\end{itemize}

\end{document}
