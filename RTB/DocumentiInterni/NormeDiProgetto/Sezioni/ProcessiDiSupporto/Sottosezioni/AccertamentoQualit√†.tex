\subsection{Accertamento della Qualità}
\label{subsec:accertamento_qualità}
Il processo di accertamento della qualità ha l'obiettivo di garantire che il progetto rispetti gli standard e i requisiti definiti, sia riguardo ai prodotti sviluppati sia ai processi adottati. 
Questo processo rappresenta un elemento centrale nella gestione del progetto, in quanto assicura il monitoraggio continuo e la verifica della conformità alle specifiche tecniche, agli obiettivi di progetto e alle normative applicabili.
Il processo si basa su principi fondamentali quali tracciabilità, trasparenza e miglioramento continuo.
Utilizza inoltre metriche e strumenti oggettivi per valutare sia l'efficacia dei processi che la qualità dei prodotti.
Attraverso una pianificazione accurata e un monitoraggio costante, il processo di accertamento della qualità contribuisce a ridurre i rischi, migliorare le performance e a soddisfare le aspettative degli stakeholder.
\subsubsection{Attività}
Per l'accertamento della qualità, il gruppo adotta il ciclo di Deming, o PDCA (Plan-Do-Check-Act), un approccio iterativo e continuo volto a migliorare i processi e garantire il raggiungimento degli obiettivi di qualità.
Il Ciclo di Deming è composto da 4 fasi fondamentali:
\begin{itemize}
    \item \textbf{Plan}: In questa fase, si definiscono gli obiettivi di qualità e le strategie per garantirne il raggiungimento.
    È essenziale pianificare le attività che saranno svolte durante il ciclo di vita del progetto, stabilendo le metriche da monitorare e i criteri di accettazione.
    \item \textbf{Do}: In questa fase, le attività pianificate vengono effettivamente eseguite.
    Si mettono in atto le procedure di controllo della qualità stabilite per monitorare la qualità del progetto in fase di sviluppo.
    \item \textbf{Check}: Questa fase consiste nel confrontare i risultati ottenuti con gli obiettivi di qualità definiti.
    Si verificano i dati raccolti durante la fase "Do" per determinare se i processi e i risultati sono allineati con gli standard di qualità e i requisiti.
    \item \textbf{Act}: In questa fase finale, vengono intraprese azioni correttive o preventive in base ai risultati della fase "Check".
\end{itemize}
\subsubsection{Standard di riferimento per la qualità di prodotto del software}
Per la valutazione della qualità del software con conseguente stesura delle metriche viene seguita la normativa ISO/IEC 9126, secondo lo standard le qualità sono suddivise nelle seguenti categorie:
\begin{itemize}
    \item \textbf{qualità esterne}: misurano i comportamenti del software durante la sua esecuzione;
    \item \textbf{qualità interne}: si applicano al software non eseguibile, permettono di individuare eventuali problemi che potrebbero influire sulla qualità finale del prodotto prima che sia realizzato il software eseguibile;
    \item \textbf{qualità in uso}: rappresentano il punto di vista dell'utente sul software, consentono di stabilire i seguenti obiettivi:
    \begin{itemize}
        \item \textit{Efficacia}: capacità del software di permettere agli utenti di raggiungere gli obiettivi specificati con accuratezza e completezza;
        \item \textit{Produttività}: capacità del software di permettere agli utenti di spendere una quantità opportuna di risorse in relazione all'efficacia ottenuta;
        \item \textit{Soddisfazione}: capacità del software di soddisfare gli utenti;
        \item \textit{Sicurezza}: capacità del software di rimanere entro livelli accettabili di rischi di danni a persone e apparecchiature.
    \end{itemize}
\end{itemize}
Lo standard normativo stabilisce un modello definendo un set di caratteristiche che consentono di misurare e valutare diversi aspetti della qualità del prodotto software:
\begin{itemize}
    \item \textbf{Funzionalità}: capacità del software di fornire le funzioni adatte a soddisfare le esigenze stabilite;
    \item \textbf{Affidabilità}: capacità del software di mantenere un livello di prestazioni specifico quando usato in date condizioni per un dato periodo di tempo;
    \item \textbf{Efficienza}: capacità del software di fornire adeguate prestazioni relativamente alla quantità di risorse utilizzate;
    \item \textbf{Usabilità}: capacità del software di essere propriamente compreso ed utilizzato dall'utente;
    \item \textbf{Manutenibilità}: capacità del software di essere modificato per introdurre migliorie o adattamenti;
    \item \textbf{Portabilità}: capacità del software di essere di essere trasportato da un ambiente di lavoro ad un altro.
\end{itemize}
\subsubsection{Notazione Metriche di Qualità}
Ogni metrica è identificata in un modo univoco seguendo la notazione: \textit{M}.[\textit{Tipo}].[\textit{Abbreviazione Nome}]
Dove:
\begin{itemize}
    \item \textbf{M}: indica "metrica"
    \item \textbf{Tipo}: sarà PC per indicare che la metrica misura la qualità di un processo o PR per indicare che la metrica misura la qualità di un prodotto
    \item \textbf{Abbreviazione Nome}: viene inserito l'acronimo basato sul nome completo della metrica
\end{itemize}
\subsubsection{Descrizione Metriche di Qualità}
Le metriche sono descritte tramite i seguenti campi:
\begin{itemize}
    \item \textbf{Notazione}: seguendo le indicazioni sopra elencate;
    \item \textbf{Nome}: nome completo della metrica;
    \item \textbf{Descrizione}: descrizione della metrica;
    \item \textbf{Caratteristiche}: presente unicamente nelle metriche di prodotto, indica a quale caratteristica, tra quelle indicate nella sezione relativa allo standard di riferimento, fa riferimento la metrica;
    \item \textbf{Formula di misurazione}: formula per la misurazione del valore della metrica.
\end{itemize}
Verranno poi indicati all'interno del documento "Piano di Qualifica" i valori tollerabili e i valori ottimali per ogni metrica e il processo a cui fanno riferimento.
\subsubsection{Elenco delle Metriche di Qualità} 
L'elenco delle metriche di qualità da adottare per il progetto è dettagliato secondo la struttura definita e copre diverse aree rilevanti per il processo di accertamento della qualità. Le metriche includono:
\paragraph{Metriche di processo}
\textbf{Planned Value}
\begin{itemize}
    \item \textbf{Notazione}: M.PC.PV
    \item \textbf{Descrizione}: rappresenta il costo stimato del lavoro programmato entro un determinato momento del progetto, secondo il piano di progetto originale
    \item \textbf{Formula}: $BAC \times (\% Lavoro Pianificato)$, dove $BAC$ indica Budget at Completion cioè il budget previsto per la realizzazione del progetto
\end{itemize}
\textbf{Earned Value}
\begin{itemize}
    \item \textbf{Notazione}: M.PC.EV
    \item \textbf{Descrizione}: rappresenta il valore del lavoro effettivamente completato alla data corrente
    \item \textbf{Formula}: $BAC \times (\% Lavoro Completato)$, dove $BAC$ indica Budget at Completion cioè il budget previsto per la realizzazione del progetto
\end{itemize}
\textbf{Actual Cost}
\begin{itemize}
    \item \textbf{Notazione}: M.PC.AC
    \item \textbf{Descrizione}: rappresenta il costo sostenuto fino alla data corrente
    \item \textbf{Formula}: costo speso per la realizzazione delle attività svolte fino data corrente
\end{itemize}
\textbf{Schedule Variance}
\begin{itemize}
    \item \textbf{Notazione}: M.PC.SV
    \item \textbf{Descrizione}: misura la variazione (in percentuale) tra il valore del lavoro effettivamente completato e il valore del lavoro pianificato, indica se il progetto è in ritardo o in anticipo rispetto a quanto preventivato
    \item \textbf{Formula}: $\frac{EV - PV}{EV} \times 100$
\end{itemize}
\textbf{Cost Variance}
\begin{itemize}
    \item \textbf{Notazione}: M.PC.CV
    \item \textbf{Descrizione}: misura la variazione (in percentuale) tra il valore del lavoro completato e il costo effettivo sostenuto per tale lavoro, indica se il progetto sta rispettando il budget pianificato
    \item \textbf{Formula}: $\frac{EV - AC}{EV} \times 100$
\end{itemize}
\textbf{Variazione del piano}
\begin{itemize}
    \item \textbf{Notazione}: M.PC.VP
    \item \textbf{Descrizione}: misura la variazione (in percentuale) tra il numero di task pianificati per un determinato periodo di tempo e il numero di task realmente realizzati, misura la qualità della pianificazione del lavoro per un periodo di tempo
    \item \textbf{Formula}: $\frac{T_p - T_c}{T_p} \times 100$, dove $T_p$ indica il numero di task pianificati e $T_c$ indica il numero di task completati
\end{itemize}
\textbf{Estimated at Completion}
\begin{itemize}
    \item \textbf{Notazione}: M.PC.EAC
    \item \textbf{Descrizione}: rappresenta una stima aggiornata del costo totale previsto per la realizzazione del progetto
    \item \textbf{Formula}: $AC + (BAC - EV)$, (costo sostenuto + stima costi da sostenere), dove $BAC$ indica il Budget at Completion cioè il budget previsto per la realizzazione del progetto
\end{itemize}
\textbf{Rischi inattesi}
\begin{itemize}
    \item \textbf{Notazione}: M.PC.RI
    \item \textbf{Descrizione}: indica il numero dei rischi che si sono verificati in un determinato periodo e che non erano stati preventivati tramite l'Analisi dei Rischi
    \item \textbf{Formula}: Numero di rischi occorsi e non preventivati
\end{itemize}
\textbf{Risk Mitigation Rate}
\begin{itemize}
    \item \textbf{Notazione}: M.PC.RMR
    \item \textbf{Descrizione}: indica la percentuale dei rischi occorsi durante lo sviluppo in un determinato periodo che sono stati mitigati correttamente tramite le strategie di mitigazione indicate nella Analisi dei Rischi
    \item \textbf{Formula}: $\frac{R_g}{R_t} \times 100$, dove $R_g$ indica il numero di rischi gestiti correttamente e $R_t$ il numero totale di rischi verificati nel periodo in analisi
\end{itemize}
\textbf{Metriche Soddisfatte}
\begin{itemize}
    \item \textbf{Notazione}: M.PC.MS
    \item \textbf{Descrizione}: indica la percentuale di metriche soddisfatte, cioè con un valore calcolato che rispetta il valore minimo ammissibile indicato all'interno del Piano di Qualifica
    \item \textbf{Formula}: $\frac{M_s}{M_t} \times 100$, dove $M_s$ indica il numero di metriche soddisfatte mentre $M_t$ il numero totale delle metriche analizzate 
\end{itemize}
\paragraph{Metriche di prodotto}
\textbf{Percentuale Requisiti Obbligatori Soddisfatti}
\begin{itemize}
    \item \textbf{Notazione}: M.PR.PRM
    \item \textbf{Descrizione}: rappresenta la percentuale di requisiti obbligatori soddisfatti rispetto i requisiti obbligatori totali inseriti all'interno del documento Analisi dei Requisiti
    \item \textbf{Caratteristiche}: Funzionalità
    \item \textbf{Formula}: $\frac{RMS}{RMT} \times 100$, dove $RMS$ indica il numero di requisiti obbligatori soddisfatti e $RMT$ il numero di requisiti obbligatori totale 
\end{itemize}
\textbf{Percentuale Requisiti Desiderabili Soddisfatti}
\begin{itemize}
    \item \textbf{Notazione}: M.PR.PRD
    \item \textbf{Descrizione}:rappresenta la percentuale di requisiti desiderabili soddisfatti rispetto i requisiti desiderabili totali inseriti all'interno del documento di Analisi dei Requisiti
    \item \textbf{Caratteristiche}: Funzionalità
    \item \textbf{Formula}: $\frac{RDS}{RDT} \times 100$, dove $RDS$ indica il numero di requisiti desiderabili soddisfatti e $RDT$ il numero di requisiti desiderabili inseriti all'interno del documento Analisi dei Requisiti
\end{itemize}
\textbf{Percentuale Requisiti Opzionali Soddisfatti}
\begin{itemize}
    \item \textbf{Notazione}: M.PR.PRO
    \item \textbf{Descrizione}: rappresenta la percentuale di requisiti opzionali soddisfatti rispetto al numero totale di requisiti opzionali inseriti all'interno del documento di Analisi dei Requisiti
    \item \textbf{Caratteristiche}: Funzionalità
    \item \textbf{Formula}: $\frac{ROs}{ROT} \times 100$, dove $ROS$ indica il numero di requisiti opzionali soddisfatti e $ROT$ il numero di requisiti opzionali totale 
\end{itemize}
\textbf{Correttezza Ortografica}
\begin{itemize}
    \item \textbf{Notazione}: M.PR.CO
    \item \textbf{Descrizione}: rappresenta il numero di errori ortografici rilevato all'interno di un documento 
    \item \textbf{Caratteristiche}: Usabilità
    \item \textbf{Formula}: N° errori ortografici rilevati
\end{itemize}

   
% altre metriche di prodotto riguardanti codice sorgente che dovranno essere inserite

%\textbf{Parametri Per Metodo (M.PR.PPM):} Numero medio di parametri passati ai metodi. Un numero elevato di parametri può indicare che un metodo è troppo complesso o che potrebbe essere suddiviso in metodi più piccoli.
% \textbf{Campi Per Classe (M.PR.CPC):} Numero medio di campi (variabili di istanza) per classe. Un numero elevato di campi dati può indicare che una classe sta facendo troppo e che potrebbe essere suddivisa in classi più piccole.
%  \textbf{Linee di Codice Per Metodo (M.PR.LCPM):} Numero medio di linee di codice per metodo. Metodi troppo lunghi possono essere difficili da leggere, capire e mantenere.
% \textbf{Complessità Ciclomatica (M.PR.CCM):} La complessità ciclomatica rappresenta la complessità di un programma sulla base del numero di percorsi lineari indipendenti attraverso il codice sorgente. Un valore elevato indica un codice più complesso e potenzialmente più difficile da mantenere
 % \textbf{Tempo di Apprendimento (M.PR.TA):} Indica il tempo massimo richiesto da parte di un utente generico per apprendere l’utilizzo del prodotto.
 %  \textbf{Tempo di Risposta (M.PR.TR):} Indica il tempo massimo di risposta del sistema sotto carico rilevato.
 % \textbf{Code Coverage (M.PC.CC):} La Code Coverage indica quale percentuale del codice sorgente è stata eseguita durante i test. Serve per capire quanto del codice è stato verificato dai test automatizzati.
  %  \textbf{Branch Coverage (M.PC.BC):} La Branch Coverage indica quale percentuale dei rami decisionali (percorsi derivanti da istruzioni condizionali come if, for, while) del codice è stata eseguita durante i test.
   %\textbf{Failure Density (M.PC.FD):} La Failure Density indica il numero di difetti trovati in un software o in una parte di esso durante il ciclo di sviluppo rispetto alla dimensione del software stesso.
  % \textbf{Passed Test Case Percentage (M.PC.PTCP):} La Passed Test Case Percentage indica la percentuale di test che sono stati eseguiti con successo su una base di test.
%


