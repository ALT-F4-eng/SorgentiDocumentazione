\subsection{Documentazione}
\label{subsec:documentazione}
Il processo di documentazione ha lo scopo di registrare le informazioni prodotte da un processo primario garantendo la produzione di documenti coerenti e di qualità.

\subsubsection{Standard di formato}
I seguenti standard di formato sono validi per tutte le tipologie di documenti.
Questi standard devono sottostare agli standard specifici per ogni tipologia di documento indicata nel piano di documentazione.
Gli standard più specifici possono sovrascrivere i seguenti o specializzarli(es. data).

\paragraph{Standard di scrittura}
\begin{itemize}
    \item Le tabelle e le immagini devono preferibilmente comparire nella posizione in cui sono specificate all'interno del codice sorgente.
    \item Le tabelle e le immagini devono essere identificate da una label che deve essere usata ogni qual volta sia necessario farvi riferimento.
    \item Le tabelle e le immagini devono essere accompagnate da una caption che ne riassume il contenuto.
    \item Si devono utilizzare frasi non più lunghe di tre righe.
    \item I termini che appartengono al glossario devono essere indicati con una G a pedice e in corsivo.
    \item Si devono usare dei link per fare riferimento a elementi del documento.
\end{itemize}
Per informazioni pratiche sui punti sopra indicati leggere la sezione \hyperref[par:comandi_di_base]{Comandi di base}.

\paragraph{Standard di forma}

\subparagraph{Intestazione}
Ogni pagina deve contenere nell'intestazione le seguenti informazioni:
\begin{lstlisting}
    Nome documento - <versione>
\end{lstlisting}
Dove la versione rispetta le regole indicate alla sezione \hyperref[par:versione_documenti]{Versione dei documenti}. 

\subparagraph{Prima pagina}
La prima pagina deve contenere le seguenti informazioni:
\begin{itemize}
    \item Nome documento.     
    \item Nome gruppo.
    \item Data. 
    \item Versione.
    \item Logo del gruppo.
\end{itemize}

\subparagraph{Seconda pagina}
La seconda pagina deve contenere il registro delle modifiche per ogni file sottoposto a controllo di configurazione.
Il contenuto e la gestione del registro è indicato alla sezione \hyperref[par:registro_delle_modifiche]{Registro delle modifiche}.

\subparagraph{Terza pagina}
La terza pagina deve contenere l'indice.

\subsubsection{Macro categorie di documenti}
Di seguito vengono elencate le due macro categorie di documenti. 

\paragraph{Documenti interni}
Servono al team di lavoro.
Sono:
\begin{itemize}
    \item Verbali interni.
    \item Norme di progetto.
\end{itemize}

\paragraph{Documenti esterni}
Servono al team di lavoro e alla proponente.
Sono:
\begin{itemize}
    \item Piano di progetto.
    \item Verbali esterni.
    \item Analisi dei requisiti.
    \item Piano di qualifica.
\end{itemize}

\subsubsection{Piano di documentazione}
Di seguito viene riportato il piano di documentazione in cui si definiscono le tipologie di documenti che verranno prodotti durante il ciclo di vita del prodotto e le loro caratteristiche.

\paragraph{Verbali interni}

\subparagraph{Scopo}
I verbali interni sono documenti che hanno lo scopo di registrare il prodotto di una riunione interna al team di lavoro.
Permettono quindi di avere una conoscenza condivisa delle decisioni prese e delle cose da fare.

\subparagraph{Autore}
L'autore dei verbali interni deve essere il Responsabile.

\subparagraph{Input}
Gli input per la stesura di questi documenti derivano direttamente dalla riunione interna eseguita solitamente a fine sprint ovvero ogni giovedì.

\subparagraph{Struttura}
I verbali interni sono composti dalle seguenti sezioni e sottosezioni:
\begin{enumerate}
    \item \textbf{Registro presenze}.
    
    Contiene le seguenti informazioni:
    \begin{enumerate}
        \item Data.
        \item Ora inizio.
        \item Ora fine.
        \item Piattaforma usata.
        \item Tabella che attesta la presenza dei membri del team.
        Intestazioni: Componente e Presenza.
    \end{enumerate}
    \item \textbf{Verbale}.
    \begin{enumerate}
        \item \textbf{Argomenti trattati}
        
        Contiene un riassunto dei temi trattati durante la riunione.
        Indica anche eventuali perplessità o difficoltà da introdurre nel diario di bordo.
        \item \textbf{Decisioni prese}.
        Contiene un riassunto delle decisioni prese durante la riunione indicando anche una giustificazione.
    \end{enumerate}

    \item \textbf{To Do}.
    
    Indica una lista delle cose da fare nel prossimo sprint collegandole alle informazioni di gestione della configurazione.
\end{enumerate}

\subparagraph{Standard di scrittura specifici}
\begin{itemize}
    \item La data del documento riguarda il giorno in cui è avvenuta la riunione interna.
\end{itemize}

\paragraph{Verbali esterni}

\subparagraph{Scopo}
I verbali esterni sono documenti che hanno lo scopo di registrare la conoscenza del team sulle necessità della proponente a seguito di una riunione esterna.
Permettono quindi la conferma di una conoscenza comune(team e proponente) sulle necessità che il prodotto colma.
Funzionano quindi da garanzia al team sul fatto che la sua direzione sia corretta.

\subparagraph{Autore}
L'autore dei verbali esterni deve essere il Responsabile.

\subparagraph{Input}
Gli input per la stesura di questi documenti derivano direttamente dalla riunione esterna che deve essere concordata in anticipo con la proponente.

\subparagraph{Struttura}
I verbali esterni sono composti dalle seguenti sezioni e sottosezioni:
\begin{enumerate}
    \item \textbf{Registro presenze}.
    
    Contiene le seguenti informazioni:
    \begin{enumerate}
        \item Data.
        \item Ora inizio.
        \item Ora fine.
        \item Piattaforma.
        \item Tabella che attesta la presenza dei membri del team.
        Intestazioni: Componente e Presenza.
        \item Tabella che attesta i rappresentati della proponente che hanno partecipato alla riunione.
        Intestazioni: Componente e Presenza.
    \end{enumerate}
    A fondo pagina viene lasciato spazio per permettere al proponente di firmare il verbale esterno.

    \item \textbf{Domande}.
    
    Lista delle domande esposte dal team.

    \item \textbf{Conclusioni}.
    
    Lista delle conclusioni che il team ha tratto dalle risposte date dai rappresentati della proponente.
\end{enumerate}

\subparagraph{Standard di scrittura specifici}
\begin{itemize}
    \item La data del documento riguarda il giorno in cui è avvenuta la riunione esterna.
\end{itemize}

\paragraph{Analisi dei requisiti}

\subparagraph{Scopo}
Questo documento ha lo scopo di racchiudere i requisiti utente e i requisiti software sul prodotto oggetto del capitolato.

\subparagraph{Autore}
Gli autori di questo documento sono gli Analisti.

\subparagraph{Input}
Gli input per la stesura del documento di analisi dei requisiti derivano dall'attività di analisi dei requisiti.

\subparagraph{Struttura}
La struttura del documento di analisi dei requisiti è composta dalle seguenti sezioni:
\begin{enumerate}
    \item \textbf{Descrizione prodotto}.
    
    Ha lo scopo di descrivere il sistema a un alto livello di astrazione.
    Questa sezione è composta dalle seguenti sottosezioni:
    \begin{enumerate}
        \item \textbf{Obiettivi prodotto}.
        
        Descrive il bisogno che ha portato alla stesura del capitolato e come il prodotto le soddisfa.
        
        \item \textbf{Funzioni prodotto}.

        Descrive le funzioni principali del prodotto.

        \item \textbf{Caratteristiche utente}.
        
        Descrive gli utenti che useranno il sistema e le loro caratteristiche.
    \end{enumerate}

    \item \textbf{Use case}.
    
    Descrive i requisiti utente e l'analisi che porta ai requisiti software.
    

    \item \textbf{Requisiti funzionali di qualità e di vincolo}.
    
    Elenca i requisiti funzionali e non.
    
\end{enumerate}

\paragraph{Norme di progetto}
Il presente documento ha lo scopo indicato nell'introduzione.

\subparagraph{Autore} 
L'autore di questo documento è l'Amministratore.

\subparagraph{Input}
Gli input per la stesura del documento delle norme di progetto derivano dal processo di gestione.

\subparagraph{Struttura}
Deve esistere una sezione per ogni categoria di processo e una sottosezione per ogni processo indicato nello standard IEEE 12207:1996 usato dal team.

\paragraph{Piano di progetto}
Il documento piano di progetto contiene le informazioni che permettono la gestione di progetto da parte del Responsabile.

\subparagraph{Autore} 
L’autore di questo documento è il Responsabile.

\subparagraph{Input}
Gli input per la stesura del documento delle norme di progetto derivano dal processo di gestione.

\subparagraph{Struttura}
Il piano di progetto è composto dalle seguenti sezioni:
\begin{enumerate}
    \item \textbf{Analisi dei rischi}.
    
    Contiene l'output dell'attività di analisi dei rischi.

    \item \textbf{Stima dei costi}.
    
    Contiene la stima delle ore che il gruppo ritiene di consumare e i costi che derivano dalle stesse. 
    
    \item \textbf{Milestone principali}.
    
    Definisce le milestone principali che devono essere sottoposte alla validazione del proponente e/o del committente.
    Per ogni milestone vengono indicate:
    \begin{itemize}
        \item Data di consegna.
        \item Baseline.
        \item Risorse preventivate.
    \end{itemize}
    
    \item \textbf{Primo periodo}.
    
    Indica l'insieme di risorse utilizzate nel primo periodo di progetto durante il quale la pianificazione era ancora confusionaria.
    
    Viene quindi mostrato lo stato delle risorse col termine del primo periodo nelle sottosezioni:
    \begin{enumerate}
        \item \textbf{Consuntivo}.
        
        Tabella composta dalle colonne: "Ruolo", "Ore consumate".

        \item \textbf{Aggiornamento rimaste}.
        
        Tabella composta dalle colonne: "Ruolo", "Ore rimaste".
    \end{enumerate}
    
    \item \textbf{Sprint}
    
    Per ogni sprint contiene una sottosezione chiamata \texttt{sprint n} che a sua volta contiene le seguenti sottosezioni:
    \begin{enumerate}
        \item \textbf{Obbiettivi}.
        
        Descrizione degli obbiettivi dello sprint tramite una lista puntata contenente la descrizione di essi e le issue assegnate a ogni obbiettivo.
        \item \textbf{Pianificazione}.
        
        Tabella composta dalle colonne: "Identificativo richiesta di modifica", "Ore preventivate" e "Ruolo".

        \item \textbf{Consuntivo}.
        
        Tabella composta dalle colonne: "Identificativo richiesta di modifica", "Ore sviluppo", "Ore verifica" e "Stato".
        Dove "Stato" viene indicato seguendo la convenzione \hyperref[par:stati]{Stati delle richieste di modifica}.
        
        \item \textbf{Retrospettiva}.
        
        Descrizione dei problemi e dei rischi verificati durante lo sprint.

        \item \textbf{Aggiornamento risorse rimaste}.
        
        Aggiornamento delle risorse restanti in seguito allo sprint tramite la tabella:
        "Ruolo", "Ore rimanenti".
        
    \end{enumerate}

\end{enumerate}