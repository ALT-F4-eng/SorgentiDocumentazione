\subsection{Processo di Verifica}
\label{subsec:proc_verifica}
Il processo di verifica ha come obiettivo fondamentale l'accertamento che:
\begin{itemize}
    \item i \textbf{prodotti di lavoro} generati durante il \glossario{ciclo di vita} del software rispettino i requisiti specificati
    \item i \textbf{processi seguiti} siano conformi allo standard, alle linee guida e ai piani definiti
\end{itemize}
\subsubsection{Descrizione}
Questo processo consiste nel fornire prove oggettive che i risultati di una specifica fase dello sviluppo software rispettino tutti i requisiti previsti. 
Si basa sull’analisi e revisione del contenuto per valutare la coerenza, la completezza e la correttezza dei risultati. Nel caso di codice, include anche il testing per assicurarsi che i risultati siano conformi alle aspettative definite.
Le task fondamentali previste dal processo sono:
\begin{itemize}
    \item verifica dei processi;
    \item verifica dei requisiti;
    \item verifica della progettazione;
    \item verifica del codice;
    \item verifica della documentazione.
\end{itemize}
Per garantire l'accertamento della conformità, ogni volta che si apporta una modifica, è necessario sottoporre l'intero contenuto aggiornato a una verifica. 
L'incremento della versione del prodotto aggiornato avviene esclusivamente se la modifica viene verificata e la verifica ha esito positivo.
Il processo di verifica viene svolto dai membri incaricati come verificatori (che si segneranno all'interno del registro delle modifiche del documento). 
i quali non possono essere la stessa persona a cui è stata assegnata la realizzazione del prodotto da verificare.
\subsubsection{Analisi statica}
L'\glossario{analisi statica} è un approccio alla verifica che non richiede l'esecuzione del codice dell'oggetto di verifica per individuare i difetti del prodotto software 
e accertarne la sua completezza e coerenza.
Si applica non solo al codice, ma anche alla documentazione, verificando la conformità alle regole del prodotto, l'assenza di difetti e la presenza delle proprietà desiderate.
Dal team viene utilizzata una tecnica standard per l'\glossario{analisi statica} dei prodotti: \textbf{\glossario{Walkthrough}}.
\paragraph{Walkthrough}
Il \glossario{Walkthrough} è uno dei metodi di lettura nell'\glossario{analisi statica} utilizzato per esaminare e verificare una parte del prodotto, che sia documento o codice, per accertarne la conformità ai requisiti o vincoli stabiliti precedentemente.
Si tratta di un approccio collaborativo tra autore e verificatore durante il quale viene esaminato un prodotto o una sua parte, seguendo un percorso prestabilito e cercando di identificare difetti attraverso una lettura critica ad ampio spettro, priva di assunzioni.
Nel caso del controllo del codice, il verificatore deve simulare diverse possibili esecuzioni, mentre per i documenti deve analizzarne il contenuto.
Le fasi che vengono svolte durante questa tecnica di \glossario{analisi statica} sono:
\begin{itemize}
    \item \textbf{lettura}: il verificatore effettua una lettura critica dell'oggetto in esame cercando eventuali errori;
    \item \textbf{discussione}: al termine della lettura, nel caso vengano rilevati problemi, il verificatore comunica con gli autori e propone eventuali suggerimenti, con l'obiettivo di correggere i difetti;
    \item \textbf{correzione e repeat}: una volta terminata la discussione e rilevati i difetti, gli autori sono pregati di correggere tali difetti seguendo le indicazioni discusse. Successivamente, si passa di nuovo al passo 2.
\end{itemize}
\paragraph{verifica della documentazione}
La verifica della documentazione, composta solamente da \glossario{analisi statica} dei documenti realizzati e/o modificati, ha lo scopo di verificare
la qualità, completezza, coerenza e conformità dei documenti tecnici relativi al prodotto software da realizzare.
Durante l'esecuzione della tecnica di \glossario{Walkthrough} descritta precedentemente, il verificatore ha lo scopo di 
assicurarsi che vengano seguiti correttamente gli standard di scrittura e di forma, sia generici che specifici, indicati all'interno
del \hyperref[subsec:documentazione]{processo di Documentazione}.
Oltre a ciò, è stata realizzata una \glossario{GitHub Action} che, all'apertura di una \glossario{pull request}, realizza un'analisi grammaticale dei file 
modificati e coinvolti nella \glossario{pull request}, fornendo successivamente un \glossario{feedback} al gruppo di lavoro.
La \glossario{GitHub Action} assicura la correttezza grammaticale dei documenti verificati migliorando il processo di verifica, aumentandone l'efficienza 
e riducendo il tempo richiesto per le revisioni manuali, permettendo al verificatore di concentrarsi maggiormente su aspetti 
di forma e di contenuto.
\subsubsection{Analisi dinamica}
L'\glossario{analisi dinamica} è un approccio all'analisi di sitemi informatici e prodotti software che si concentra sull'osservazione e verifica
del loro comportamento durante l'esecuzione.
Quest'ultima è ampiamente impiegata per individuare errori a runtime, ottimizzare l'efficienza e testare la robustezza contro scenari imprevisti.
Grazie alla sua capacità di fornire dati concreti e rilevanti, rappresenta un elemento fondamentale nel processo di sviluppo e manutenzione di applicazioni e sistemi complessi.
Essa prevede la definizione di una suite di test, generalmente automatizzati e riproducibili, che vengono eseguiti a runtime per valutare il comportamento del sistema in risposta a specifici input. 
Questi test verificano la correttezza delle funzionalità, l'efficienza delle prestazioni e l'assenza di errori o anomalie operative.
I principali tipi di test che vengono utilizzati per l'\glossario{analisi dinamica} sono: 
\begin{itemize}
    \item \textbf{test di unità}: Mirano a verificare il corretto funzionamento di singole unità o componenti del software (ad esempio, funzioni o metodi). 
    Sono solitamente automatizzati e si concentrano su un ambito ristretto per identificare errori locali;
    \item \textbf{test di integrazione}: Valutano come le diverse unità o moduli del software interagiscono tra loro. 
    L'obiettivo è assicurarsi che le componenti integrate funzionino correttamente come un sistema coerente;
    \item \textbf{test di sistema}: Analizzano il comportamento dell'intero sistema per verificare che soddisfi i requisiti specificati. 
    Considerano il software come un unico blocco, includendo interazioni con l'ambiente e altre applicazioni.
\end{itemize}
All'interno del \hyperref[subsection:processo_sviluppo]{processo di Sviluppo}, nello specifico nella sezione di testing, vi è una descrizione 
più approfondita delle varie tipologie di test adottate durante lo sviluppo del prodotto software, ogni test utilizzato verrà poi 
codificato e indicato all'interno del documento "Piano di Qualifica".
