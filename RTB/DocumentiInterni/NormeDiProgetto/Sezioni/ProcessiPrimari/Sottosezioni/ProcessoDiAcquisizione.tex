\subsection{Processo di acquisizione}
\label{subsection:Processo_acquisizione}
Il processo di acquisizione, secondo la norma ISO/IEC 12207, comprende l'insieme delle attività che un'organizzazione, cioè il \glossario{Fornitore}, deve svolgere per ottenere un sistema, un prodotto software o un servizio software. Esso prevede l'identificazione dei requisiti, la selezione di un Fornitore e un \glossario{Proponente}, la stipula di un contratto, la supervisione della fornitura e, infine, l'accettazione e il completamento del prodotto o servizio acquisito.
Dunque, per il nostro gruppo, il processo di acquisizione consiste nelle seguenti attività principali:

\begin{itemize}
    \item \textbf{Iniziazione}; 
    \item \textbf{Studio di fattibilità}; 
    \item \textbf{Selezione del capitolato}; 
    \item \textbf{Preparazione della candidatura}; 
    \item \textbf{Accettazione della candidatura}; 
\end{itemize}

\subsubsection{Iniziazione}
Il Fornitore effettua un'analisi preliminare dei capitolati d’appalto, la comprensione degli obiettivi e dei requisiti iniziali per lo sviluppo, includendo il confronto e la discussione con i proponenti.

\subsubsection{Studio di fattibilità}
Il Fornitore analizza i capitolati per identificare eventuali punti critici e valuta le idee ricevute dai proponenti.

\subsubsection{Selezione del capitolato}
Il Fornitore sceglie e prepara il capitolato d'appalto che meglio soddisfa le sue aspettative.

\subsubsection{Preparazione della candidatura}
Il Fornitore prepara e definisce la redazione dei documenti necessari cioè la lettera di candidatura, la stima dei costi, l'assunzione d’impegni e la valutazione dei capitolati, revisionando il contenuto per correggere eventuali errori e, se necessario, aggiorna il documento prima della presentazione ai \glossario{Committenti}.

\subsubsection{Accettazione della candidatura}
Approvazione della soluzione selezionata e formalizzazione del contratto relativo.