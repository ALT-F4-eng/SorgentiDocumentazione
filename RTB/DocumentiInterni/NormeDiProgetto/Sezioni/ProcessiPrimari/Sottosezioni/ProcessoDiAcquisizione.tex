\subsection{Processo di acquisizione}
\label{subsection:Processo_acquisizione}
Il processo di acquisizione, secondo la norma \glossario{ISO/IEC 12207:1996}, comprende l'insieme delle attività che un'organizzazione, cioè il \glossario{Fornitore}, deve svolgere per ottenere un sistema, un prodotto software o un servizio software. Esso prevede l'identificazione dei requisiti, la selezione di un \glossario{Fornitore} e un \glossario{Proponente}, la stipula di un contratto, la supervisione della fornitura e, infine, l'accettazione e il completamento del prodotto o servizio acquisito.
Dunque, per il nostro gruppo, il processo di acquisizione consiste nelle seguenti attività principali:

\begin{itemize}
    \item \textbf{Iniziazione}; 
    \item \textbf{Studio di fattibilità}; 
    \item \textbf{Selezione del \glossario{capitolato}}; 
    \item \textbf{Preparazione della candidatura}; 
    \item \textbf{Accettazione della candidatura}; 
\end{itemize}

\subsubsection{Iniziazione}
Il \glossario{Fornitore} effettua un'analisi preliminare dei capitolati d’appalto, la comprensione degli obiettivi e dei requisiti iniziali per lo sviluppo, includendo il confronto e la discussione con i proponenti.

\subsubsection{Studio di fattibilità}
Il \glossario{Fornitore} analizza i capitolati per identificare eventuali punti critici e valuta le idee ricevute dai proponenti.

\subsubsection{Selezione del capitolato}
Il \glossario{Fornitore} sceglie e prepara la candidatura per il \glossario{capitolato} che meglio soddisfa le sue aspettative.

\subsubsection{Preparazione della candidatura}
Il \glossario{Fornitore} prepara e definisce la stesura dei documenti necessari:
\begin{itemize}
    \item \textbf{Lettera di candidatura};
    \item \textbf{Stima dei costi};
    \item \textbf{Assunzione d’impegni};
    \item \textbf{Valutazione dei capitolati}.
\end{itemize}
Revisionando il contenuto per correggere eventuali errori e, se necessario, aggiorna il documento prima della presentazione ai \glossario{Committenti}.

\subsubsection{Accettazione della candidatura}
L'approvazione della soluzione selezionata e la conseguente formalizzazione del contratto spettano ai committenti, i quali, con la loro approvazione, rendono ufficiale il contratto.