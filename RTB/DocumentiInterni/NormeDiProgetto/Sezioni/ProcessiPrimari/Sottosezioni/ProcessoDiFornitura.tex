\subsection{Processo di fornitura}
\label{subsection:processo_fornitura}
Il processo di fornitura, definito dalla norma ISO/IEC 12207, comprende le attività svolte dal Fornitore e inizia con la presentazione di una proposta al Proponente o la stipula di un contratto. Nel nostro caso, il processo riguarda la Proponente Zucchetti S.p.A. e si avvia al termine del processo di acquisizione, ossia al completamento della fase di accettazione della candidatura.
Il processo prevede la pianificazione e l’organizzazione delle risorse, delle procedure e dei piani necessari per la gestione del progetto, fino alla consegna del sistema, prodotto o servizio software. Questo processo è fondamentale per garantire che il software soddisfi i requisiti del cliente, sia di alta qualità e venga realizzato rispettando tempi e costi concordati. L’obiettivo principale è allineare costantemente le aspettative dell’acquirente con i risultati ottenuti durante l’esecuzione del progetto.
Dunque, per il nostro gruppo, il processo si articola nelle seguenti attività principali:
\begin{itemize}
    \item \textbf{Pianificazione};
    \item \textbf{Esecuzione e controllo};
    \item \textbf{Revisione e valutazione};
    \item \textbf{Consegna e completamento};
\end{itemize}

\subsubsection{Pianificazione}
Il Fornitore definisce obiettivi, risorse e procedure necessari per l’esecuzione del progetto, identificando i requisiti per la gestione, la misurazione della qualità e lo svolgimento delle attività. Questa fase sta principalmente nella redazione del Piano di Progetto, che pianifica l’utilizzo delle risorse e documenta i risultati attesi.

\subsubsection{Esecuzione e controllo}
Il Fornitore attua le attività pianificate nel Piano di Progetto, rispettando le norme stabilite. Inoltre, viene effettuato un controllo continuo sullo stato di avanzamento e sulla gestione delle risorse, garantendo la rendicontazione rispetto agli obiettivi prefissati. 

\subsubsection{Revisione e valutazione}
Il Fornitore definisce criteri e procedure per la revisione ed esegue le operazioni in conformità a tali criteri. L'obiettivo è verificare che il progetto soddisfi i requisiti e rispetti gli standard stabiliti.

\subsubsection{Consegna e completamento}
Consegna del progetto, verifica finale e accettazione da parte dell’acquirente.

\subsubsection{Contatti con la Proponente dell’azienda}
La Proponente, Zucchetti S.p.A., fornisce l'indirizzo email del proprio rappresentante per la comunicazione asincrona e per la pianificazione di videochiamate su Google Meet con il Fornitore. Le comunicazioni tra Proponente e Fornitore riguardano vari aspetti del progetto, tra cui la raccolta dei requisiti, la raccolta di feedback sui risultati ottenuti e le indicazioni sull'avanzamento del progetto. Durante il corso del progetto, che ha carattere didattico, la Proponente assume il ruolo di cliente, mantenendo un ampio grado di libertà per il Fornitore, il quale è responsabile della realizzazione del prodotto. Per ogni colloquio con l'azienda Proponente, ovvero per ogni videochiamata tramite Google Meet, sarà redatto un Verbale Esterno che riassume i punti chiave discussi durante l'incontro.

\subsubsection{Documentazione fornita}
Di seguito viene descritta la documentazione che il gruppo rende disponibile alla Proponente Zucchetti S.p.A. e ai committenti, Prof. Tullio Vardanega e Prof. Riccardo Cardin.

\paragraph{Piano di progetto}
Il Piano di Progetto è un documento redatto dal responsabile del progetto, che fornisce una visione dettagliata della gestione e dell’organizzazione del gruppo di lavoro. La sua funzione principale è quella di pianificazione, monitoraggio e controllo delle attività, al fine di garantire il raggiungimento degli obiettivi nei tempi e nei costi stabiliti. Inoltre, il piano include l'analisi e la gestione dei rischi, nonché la pianificazione, il preventivo e il consuntivo di ciascun sprint, monitorando costantemente l'avanzamento del progetto e le risorse utilizzate.
Il documento è suddiviso nelle seguenti sezioni:
\begin{itemize}
    \item \textbf{Introduzione}: Una breve descrizione dello scopo del documento e delle varie sezioni che lo compongono.
    \item \textbf{Analisi dei rischi}: Riguarda l'identificazione dei potenziali rischi che potrebbero sorgere durante il corso del progetto, i quali potrebbero causare ritardi o ostacoli nella sua progressione. Vengono inoltre sviluppate strategie di prevenzione per evitare che tali rischi si manifestino, nonché strategie di mitigazione per ridurne l'impatto nel caso in cui si verificassero, al fine di garantire la continuità del progetto.
    \item \textbf{Stima dei costi}: Calcolo delle risorse necessarie per il completamento del progetto, che viene aggiornato ad ogni sprint.
    \item \textbf{\glossario{Milestone} principali}: Sezione dedicata all'indicazione delle milestone fondamentali e delle baseline di riferimento del progetto.
    \item \textbf{Primo periodo}: Rappresenta la fase iniziale del gruppo.
    \item \textbf{\glossario{Sprint}}: Gli sprint rappresentano fasi del progetto in cui vengono definiti obiettivi specifici e attività da completare. Durante ogni sprint, il gruppo si impegna a raggiungere tali obiettivi. Inoltre, si stabiliscono una revisione retrospettiva e la data per lo sprint successivo.
    La sottosezione si articola nelle seguenti parti:
    \begin{enumerate}
        \item \textbf{Obiettivi}: Definisce gli obiettivi specifici dello sprint.
        \item \textbf{Pianificazione}: Presenta il preventivo dei costi e delle risorse previste per lo sprint.
        \item \textbf{Consuntivo}: Riporta il resoconto dei costi e delle risorse effettivamente utilizzati.
        \item \textbf{Retrospettiva}: Riporta la conclusione dello sprint e propone miglioramenti sui punti deboli emersi.
        \item \textbf{Aggiornamento risorse rimaste}: Aggiorna la situazione delle risorse disponibili per il progetto.
    \end{enumerate}
\end{itemize}

\paragraph{Analisi dei requisiti}
L’analisi dei requisiti è un documento redatto dall’analista che fornisce una visione chiara delle richieste e delle aspettative dell’azienda Proponente. Va ad includere un elenco dettagliato delle funzionalità da sviluppare e implementare, nonché i casi d’uso che definiscono le interazioni tra il sistema e l’utente.
Il documento è suddiviso nelle seguenti sezioni:
\begin{enumerate}
    \item \textbf{Introduzione}: Una breve descrizione dello scopo del documento e delle varie sezioni che lo compongono.
    \item \textbf{Descrizione del prodotto}: Descrive gli obiettivi del prodotto, le sue funzioni principali e le caratteristiche.
    \item \textbf{Casi d’uso}: Indica tutti i casi d’uso individuati dal gruppo durante analisi.
    \item \textbf{Requisiti}: Elenco completo dei requisiti del prodotto, organizzato per categorie e con riferimento alle fonti da cui proviene il tracciamento.
\end{enumerate}

\paragraph{Piano di Qualifica}
Il Piano di Qualifica è un documento che definisce le attività del verificatore nel progetto, stabilendo le strategie e gli approcci per garantire la qualità del prodotto software in fase di sviluppo. Redatto dal verificatore, questo documento descrive le modalità di verifica e validazione. Tutti i membri del team si baseranno su questo documento per garantire il raggiungimento della qualità desiderata.
Il documento è suddiviso nelle seguenti sezioni: 
\begin{enumerate}
    \item \textbf{Introduzione}: Una breve descrizione dello scopo del documento e delle varie sezioni che lo compongono.
    %\item \textbf{Obiettivi di qualità}: Questa sezione presenta i valori accettabili e gli ambiti per le metriche definite dal team, le metriche sono divise:Qualità di processo,Qualità di prodotto,Qualità per obiettivo.
    %\item \textbf{Metodologie di testing}: Include tutti i test necessari per verificare che il prodotto rispetti i requisiti specificati.
    %\item \textbf{Cruscotto di valutazione della qualità}: Vengono scritte tutte le attività di verifica effettuate e le problematiche emerse durante lo sviluppo del software.
\end{enumerate}

\paragraph{Glossario}
Il Glossario è un elenco dettagliato e organizzato di tutti termini, acronimi e definizioni utilizzati nella documentazione. L’obiettivo principale di questo documento è fornire una comprensione chiara dei concetti e dei termini specifici impiegati nel progetto, garantendo una comunicazione coerente e precisa tra tutti i membri del gruppo e con gli \glossario{Stakeholder}. In questo modo, si facilita il lavoro collaborativo e si assicura un allineamento efficace su linguaggio e significati durante l'intero ciclo di vita del progetto.

\paragraph{Lettera di presentazione}
La Lettera di Presentazione è il documento con cui il gruppo comunica la propria candidatura alle revisioni di avanzamento \glossario{Requirements and Technology Baseline} e \glossario{Product Baseline}. Nel primo caso essa include informazioni sui repository di documentazione e codice sorgente, il riferimento al \glossario{Proof of Concept} e un aggiornamento sugli impegni con la stima del preventivo "a finire". 