\documentclass[a4paper, 12pt]{article}

\usepackage{custom}


%--------------------VARIABILI--------------------
\def\lastversion{v0.1}
\def\title{Verbale interno}
\def\date{10 dicembre 2024}
%------------------------------------------------

\begin{document}

\primapagina

\begin{registromodifiche}
        v0.1 & 10 dicembre 2024 & Eghosa Matteo Igbinedion Osamwonyi & Enrico Bianchi & Prima stesura del documento\\
    \hline 
\end{registromodifiche}

\tableofcontents

\newpage

\section{Registro presenze}
\begin{itemize}
    \item[] \textbf{Data}: 10 Dicembre 2024
    \item[] \textbf{Ora inizio}:  15:30
    \item[] \textbf{Ora fine}: 16:30
    \item[] \textbf{Piattaforma}: Discord	
\end{itemize}
\begin{table}[H]
\centering
{\renewcommand{\arraystretch}{2}
\begin{tabularx}{\textwidth}{| X | X |}
    \hline
        \textbf{\large Componente} & 
        \textbf{\large Presenza} \\
    \hline 
    \hline
        Eghosa Matteo Igbinedion Osamwonyi&
        Presente \\
    \hline 
        Guirong Lan&
        Presente \\
    \hline 
        Enrico Bianchi&
        Presente \\
    \hline 
        Francesco Savio&
        Presente \\
    \hline 
        Marko Peric&
        Presente \\
    \hline 
        Pedro Leoni&
        Presente \\
    \hline 

\end{tabularx}}
\end{table}

\newpage

\section{Verbale}
\subsection{Argomenti trattati}

Durante l'incontro, sono stati trattati aspetti relativi all'organizzazione della \glossario{repository} in cui organizzarvi la stesura di ulteriori file, il completamento del piano di progetto, e la pianificazione degli \glossario{sprint}. 
Si è discusso sulla necessità di ultimare le prime sezioni del piano di progetto, con focus su analisi dei rischi, stima dei costi e \glossario{milestone} principali.

\begin{itemize}
    \item \textbf{Ultimazione della documentazione:} Completamento della stesura prima sezione del piano di progetto.
    \item \textbf{Strutturazione della \glossario{repository}:} Organizzazione della \glossario{repository} per una maggiore profondità e ampiezza al fine di ordinare le informazioni in modo più funzionale.
\end{itemize}

\subsection{Decisioni prese}

È stato deciso di strutturare la \glossario{repository} separando chiaramente i file relativi agli \glossario{sprint}, ai processi primari e alle loro sotto-sezioni. 
Inoltre, si è stabilito che ogni \glossario{sprint} durerà una settimana e terminerà ogni martedì. Infine, per migliorare il monitoraggio delle risorse, ogni membro del team dovrà registrare le ore previste nella descrizione delle \glossario{issue} e il tempo effettivo impiegato in un commento della \glossario{pull request} relativa.

\begin{itemize}
    \item \textbf{Strutturazione della \glossario{repository}:} Separazione dei file in sezioni e sottosezioni che poi verranno unite nel file principale.
    \item \textbf{Cadenza degli \glossario{sprint}:} È stato deciso di fare \glossario{Sprint} settimanali con scadenza ogni martedì.
    \item \textbf{Monitoraggio delle risorse:} Registrazione delle ore previste ed effettive nelle \glossario{issue} e \glossario{pull request}.
\end{itemize}

\section{To Do}
\begin{itemize}
    \item Redigere la definizione di obiettivi, funzioni e caratteristiche utente del prodotto, per il documento di analisi dei requisiti
    \item Scrivere i processi primari da includere nelle norme di progetto
    \item Creare e completare i file \texttt{sprint1.tex} e \texttt{primoperiodo.tex}
    \item Stendere il piano di progetto con analisi dei rischi, stima dei costi, \glossario{milestone} principali e la sezione relativa al primo \glossario{sprint} svolto
    \item Implementare la \glossario{GitHub Action} per il calcolo automatico delle ore di lavoro
\end{itemize}

\end{document}
