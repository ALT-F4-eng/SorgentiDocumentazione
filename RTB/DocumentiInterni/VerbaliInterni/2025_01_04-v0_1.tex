\documentclass[a4paper, 12pt]{article}

\usepackage{custom}


%--------------------VARIABILI--------------------
\def\lastversion{v0.1}
\def\title{Verbale interno}
\def\date{02 gennaio 2025}
%------------------------------------------------

\begin{document}

\primapagina


\begin{registromodifiche}
        \lastversion & 04 gennaio 2025 & Guirong Lan & Francesco Savio & Stesura verbale interno \\
        \hline 
\end{registromodifiche}

\tableofcontents

\newpage

\section{Registro presenze}
\begin{itemize}
    \item[] \textbf{Data}: 02 gennaio 2025
    \item[] \textbf{Ora inizio}:  18:00
    \item[] \textbf{Ora fine}: 20:00
    \item[] \textbf{Piattaforma}: Discord	
\end{itemize}

\begin{table}[H]
\centering
{\renewcommand{\arraystretch}{2}
\begin{tabularx}{\textwidth}{| X | X |}
    \hline
        \textbf{\large Componente} & 
        \textbf{\large Presenza} \\ 
    \hline 
    \hline
        Eghosa Matteo Igbinedion Osamwonyi&
        Presente \\
    \hline 
        Guirong Lan&
        Presente \\
    \hline 
        Enrico Bianchi&
        Presente \\
    \hline 
        Francesco Savio&
        Presente \\
    \hline 
        Marko Peric&
        Presente \\
    \hline 
        Pedro Leoni&
        Presente \\
    \hline 

\end{tabularx}}
\end{table}

\newpage

\section{Verbale}
\subsection{Argomenti trattati}
Durante l'incontro, sono stati trattati i seguenti argomenti:
\begin{itemize}
    \item Sono state valutate e discusse le metriche da inserire nel piano di qualifica;
    \item Finire il prima possibile gli \glossario{use case} e i \glossario{requisiti software};
    \item Consuntivo per la retrospettiva dello \glossario{sprint} 3;
    \item Pianificazione dello \glossario{sprint} 4;
\end{itemize}
\subsection{Decisioni prese}
Si è dunque deciso di:
\begin{itemize}
    \item Inserire nel piano di qualifica le metriche:
    \glossario{Planned value}, 
    \glossario{End value},
    \glossario{Earned value},
    \glossario{Actual cost},
    \glossario{Schedule variance} con il grafico,
    \glossario{Cost variance} con il grafico,
    Variazione del piano on il grafico,
    \item Avere un l'indice di \glossario{Gulpease} per completare la tabella nel piano di Qualifica;
    \item Avere un ruolo nella tabella consuntivo per maggiore chiarezza;
    
\end{itemize}
\section{To Do}
\begin{itemize}
    \item Implementare l'indice di \glossario{Gulpease};
    \item Stendere il verbale interno del 02 gennaio 2025;
    \item Aggiornare il \textit{Piano di Progetto} con lo sprint 4;
    \item Continuare e migliorare con la stesura del \textit{Piano di Qualifica};
    \item Migliorare le tabelle consuntivo del piano di progetto e aggiungere la spiegazione nelle \textit{norme di progetto};
    \item Aggiornare l'\textit{Analisi dei Requisiti} con gli use case e i requisiti software;
    \item Continuare con lo studio delle tecnologie necessarie per lo sviluppo del \glossario{PoC};
\end{itemize}
\end{document}