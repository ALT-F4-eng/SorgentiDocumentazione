\documentclass[a4paper, 12pt]{article}

\usepackage{custom}


%--------------------VARIABILI--------------------
\def\lastversion{v0.1}
\def\title{Verbale interno}
\def\date{23 gennaio 2025}
%------------------------------------------------

\begin{document}

\primapagina

\begin{registromodifiche}
        \lastversion & 11 Febbraio 2025 & Marko Peric & Enrico Bianchi & Stesura verbale interno \\
        \hline 
\end{registromodifiche}

\tableofcontents

\newpage

\section{Registro presenze}
\begin{itemize}
    \item[] \textbf{Data}: 23 gennaio 2025
    \item[] \textbf{Ora inizio}:  10:00
    \item[] \textbf{Ora fine}: 11:00
    \item[] \textbf{Piattaforma}: Discord	
\end{itemize}

\begin{table}[H]
\centering
{\renewcommand{\arraystretch}{2}
\begin{tabularx}{\textwidth}{| X | X |}
    \hline
        \textbf{\large Componente} & 
        \textbf{\large Presenza} \\
    \hline 
    \hline
        Eghosa Matteo Igbinedion Osamwonyi&
        Presente \\
    \hline 
        Guirong Lan&
        Assente \\
    \hline 
        Enrico Bianchi&
        Presente \\
    \hline 
        Francesco Savio&
        Assente \\
    \hline 
        Marko Peric&
        Presente \\
    \hline 
        Pedro Leoni&
        Presente \\
    \hline 

\end{tabularx}}
\end{table}

\newpage

\section{Verbale}
\subsection{Argomenti trattati}
Gli argomenti trattati sono stati i seguenti:
\begin{itemize}
    \item Come gestire le issue rimaste indietro nello sprint precedente senza impattare sullo sprint attuale, considerati gli esami in corso;
    \item Come gestire la visualizzazione di un grande set di coppie di domande risposte nella web app;
    \item Quali altri valori sono necessari al front-end per una visualizzazione dei dati completa;
\end{itemize}

\subsection{Decisioni prese}
In seguito sono state prese le seguenti decisioni:
\begin{itemize}
    \item I membri che hanno meno esami da sostenere si occuperanno di risolvere più issue possibili tra quelle rimaste indietro nello sprint precedente;
    \item Si è deciso di utilizzare un sistema di impaginazione per visualizzare le coppie di domande risposte nella web app;
    \item Si è deciso che il front-end necessita dei valori della media e della deviazione standard per visualizzare i dati in modo completo.
    \item L'obbiettivo attuale del gruppo è completare il POC entro la fine del mese di febbraio a causa dei ritardi.
\end{itemize}

\section{To Do}
\begin{itemize}
    \item finire le attività pianificate ma non completate
    \item completare lo sviluppo del Proof of Concept
\end{itemize}

\end{document}
