\documentclass[a4paper, 12pt]{article}

\usepackage{custom}


%--------------------VARIABILI--------------------
\def\lastversion{v0.1}
\def\title{Verbale interno}
\def\date{11 Marzo 2025}
%------------------------------------------------

\begin{document}

\primapagina

\begin{registromodifiche}
        \lastversion & 11 Marzo 2025 & Guirong Lan & & Stesura verbale interno \\
        \hline 
\end{registromodifiche}

\tableofcontents

\newpage

\section{Registro presenze}
\begin{itemize}
    \item[] \textbf{Data}: \date
    \item[] \textbf{Ora inizio}:  16:00
    \item[] \textbf{Ora fine}: 17:00
    \item[] \textbf{Piattaforma}: Discord	
\end{itemize}

\begin{table}[H]
\centering
{\renewcommand{\arraystretch}{2}
\begin{tabularx}{\textwidth}{| X | X |}
    \hline
        \textbf{\large Componente} & 
        \textbf{\large Presenza} \\
    \hline 
    \hline
        Eghosa Matteo Igbinedion Osamwonyi&
        Presente \\
    \hline 
        Guirong Lan&
        Presente \\
    \hline 
        Enrico Bianchi&
        Presente \\
    \hline 
        Francesco Savio&
        Presente \\
    \hline 
        Marko Peric&
        Presente \\
    \hline 
        Pedro Leoni&
        Presente \\
    \hline 

\end{tabularx}}
\end{table}

\newpage

\section{Verbale}
\subsection{Argomenti trattati}
Durante l’incontro sono stati affrontati i seguenti temi
\begin{itemize}
    \item Discussione e riflessione sul commento di RTB del Prof. Cardin.
    \item Revisione dei restanti documenti da sistemare per la seconda parte del RTB.
    \item Breve analisi dei requisiti non funzionali e individuazione di eventuali domande da sottoporre alla Responsabile dell’azienda.
\end{itemize}

\subsection{Decisioni prese}
In seguito sono state prese le seguenti decisioni:
\begin{itemize}
    \item Migliorare il documento di analisi dei requisiti, seguendo le indicazioni fornite dal Prof. Cardin.
    \item Rafforzare i documenti che presentano lacune per garantirne la completezza e la qualità.
\end{itemize}

\section{To Do}
\begin{itemize}
    \item Sviluppare i requisiti non funzionali, sia di qualità che di vincolo.
    \item Verificare e validare tutti i documenti in vista della seconda parte del RTB.
\end{itemize}

\end{document}
