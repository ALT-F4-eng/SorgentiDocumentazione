\documentclass[a4paper, 12pt]{article}

\usepackage{custom}


%--------------------VARIABILI--------------------
\def\lastversion{v1.0}
\def\title{Verbale interno}
\def\date{19 dicembre 2024}
%------------------------------------------------

\begin{document}

\primapagina

\begin{registromodifiche}
        \lastversion & 17 Marzo 2025 &  & Enrico Bianchi & Release\\
    \hline
        v0.1 & 22 dicembre 2024 & Eghosa Matteo Igbinedion Osamwonyi & Francesco Savio & Prima stesura del documento\\
    \hline 
\end{registromodifiche}

\tableofcontents

\newpage

\section{Registro presenze}
\begin{itemize}
    \item[] \textbf{Data}: 19 Dicembre 2024
    \item[] \textbf{Ora inizio}:  21:00
    \item[] \textbf{Ora fine}: 22:00
    \item[] \textbf{Piattaforma}: Discord	
\end{itemize}
\begin{table}[H]
\centering
{\renewcommand{\arraystretch}{2}
\begin{tabularx}{\textwidth}{| X | X |}
    \hline
        \textbf{\large Componente} & 
        \textbf{\large Presenza} \\
    \hline 
    \hline
        Eghosa Matteo Igbinedion Osamwonyi&
        Presente \\
    \hline 
        Guirong Lan&
        Presente \\
    \hline 
        Enrico Bianchi&
        Presente \\
    \hline 
        Francesco Savio&
        Presente \\
    \hline 
        Marko Peric&
        Presente \\
    \hline 
        Pedro Leoni&
        Presente \\
    \hline 

\end{tabularx}}
\end{table}

\newpage

\section{Verbale}
\subsection{Argomenti trattati}

Durante l'incontro, sono stati trattati i seguenti aspetti:
\begin{itemize}
    \item Riorganizzazione di Excel con un nuovo tipo di tabella.
    \item \glossario{Github action} per il calcolo delle ore.
    \item Discussione in relazione al tempo impiegato per le rispettive \glossario{issue} assegnate nel consultivo.
    \item Struttura per rappresentare gli \glossario{sprint} nel piano di progetto.
\end{itemize}

\subsection{Decisioni prese}

Le seguenti decisioni sono state prese durante l'incontro:
\begin{itemize}
    \item Sono state calcolate le ore usate nel primo periodo dove il team non usava ancora la metodologia a \glossario{sprint} per inserirle nel piano di progetto.
    \item È stata decisa la nuova struttura della \glossario{dashboard} excel per agevolare l'utilizzo della \glossario{github action}.
    \item È stata modificata la \glossario{dashboard} per tenere traccia degli \glossario{sprint} e delle ore usate nei relativi ruoli.
    \item È stata decisa la struttura definitiva per rappresentare gli \glossario{sprint} nel piano di progetto.
\end{itemize}

\section{To Do}
\begin{itemize}
    \item Scrivere gli \glossario{Use Case}.
    \item Scrittura nel piano di progetto della pianificazione dello \glossario{sprint} 2 e consuntivo dello \glossario{sprint} 1.
    \item Aggiornamento del processo di documentazione, indicando la parte relativa agli \glossario{sprint}.
    \item Aggiornare la parte della \glossario{GitHub Action} (grammatica e calcolo ore).
    \item Scrivere la parte relativa alla \glossario{dashboard} su Excel nel "processo di infrastruttura".
    \item Processo di verifica (statica: riferimento \glossario{GitHub Action} grammatica).
    \item Ragionare sulla divisione del piano di qualifica in sezioni e scrivere la parte introduttiva.
\end{itemize}

\end{document}

