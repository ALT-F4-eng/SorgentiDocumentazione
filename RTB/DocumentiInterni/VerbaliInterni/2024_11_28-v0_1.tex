\documentclass[a4paper, 12pt]{article}

\usepackage{custom}

%--------------------VARIABILI--------------------
\def\lastversion{v0.1}
\def\title{Verbale interno}
\def\date{28 novembre 2024}
%------------------------------------------------

\begin{document}

\primapagina

\begin{registromodifiche}
        v0.1 & 1 dicembre 2024 & Marko Peric & Pedro Leoni & Prima stesura\\
    \hline 
\end{registromodifiche}


\tableofcontents

\newpage

\section{Registro presenze}
\begin{itemize}
    \item[] \textbf{Data}: 28 novembre 2024
    \item[] \textbf{Ora inizio}:  15:00
    \item[] \textbf{Ora fine}: 16:00
    \item[] \textbf{Piattaforma}: Discord	
\end{itemize}
\begin{table}[!h]
\centering
{\renewcommand{\arraystretch}{2}
\begin{tabularx}{\textwidth}{| X | X |}
    \hline
        \textbf{\large Componente} & 
        \textbf{\large Presenza} \\ 
    \hline 
    \hline
        Eghosa Matteo Igbinedion Osamwonyi&
        Presente \\
    \hline 
        Guirong Lan&
        Presente \\
    \hline 
        Enrico Bianchi&
        Presente \\
    \hline 
        Francesco Savio&
        Presente \\
    \hline 
        Marko Peric&
        Presente \\
    \hline 
        Pedro Leoni&
        Presente \\
    \hline 

\end{tabularx}}
\end{table}

\newpage

\section{Verbale}

\subsection{Argomenti trattati}
Il gruppo si è riunito per discutere su:
\begin{itemize}
    \item L'utilizzo dei \glossario{diagrammi di gantt} per la pianificazione del progetto;
    \item La divisione dei compiti per la stesura dei \glossario{diagrammi UML} di analisi dei requisiti;
    \item Il procedimento del documento di Norme di Progetto;
    \item Come devono essere gestiti i cambiamenti;
\end{itemize}

\subsection{Decisioni prese}
\begin{itemize}
    \item Non utilizzare i \glossario{diagrammi di gantt} per la pianificazione del progetto poiché non adatti alla suddivisione del lavoro seguendo uno schema agile;
    \item Organizzare una chiamata con il proponente per avere un confronto sui requisiti analizzati;
    \item La gestione dei cambiamenti deve essere fatta tramite \glossario{issue} su \glossario{GitHub} e seguendo lo standard \glossario{IEEE 828};
\end{itemize}

\section{To Do}
Da ciò che è stato discusso, il gruppo ha deciso di:
\begin{itemize}
    \item Fare le \glossario{issue} per i \glossario{diagrammi UML} dei requisiti, dalla \textbf{\#99} alla \textbf{\#109} (Analista);
    \item Continuare con la stesura del documento di Norme di Progetto (Amministratore);
\end{itemize}
\end{document}