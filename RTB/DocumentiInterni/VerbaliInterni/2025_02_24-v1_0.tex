\documentclass[a4paper, 12pt]{article}

\usepackage{custom}


%--------------------VARIABILI--------------------
\def\lastversion{v1.0}
\def\title{Verbale interno}
\def\date{24 Febbraio 2025}
%------------------------------------------------

\begin{document}

\primapagina

\begin{registromodifiche}
        \lastversion & 17 Marzo 2025 &  & Enrico Bianchi & Release\\
        \hline    
    v0.1 & 25 Febbraio 2025 & Guirong Lan & Francesco Savio & Stesura verbale interno \\
        \hline 
\end{registromodifiche}

\tableofcontents

\newpage

\section{Registro presenze}
\begin{itemize}
    \item[] \textbf{Data}: \date
    \item[] \textbf{Ora inizio}:  22:00
    \item[] \textbf{Ora fine}: 23:00
    \item[] \textbf{Piattaforma}: Discord	
\end{itemize}

\begin{table}[H]
\centering
{\renewcommand{\arraystretch}{2}
\begin{tabularx}{\textwidth}{| X | X |}
    \hline
        \textbf{\large Componente} & 
        \textbf{\large Presenza} \\
    \hline 
    \hline
        Eghosa Matteo Igbinedion Osamwonyi&
        Presente \\
    \hline 
        Guirong Lan&
        Presente \\
    \hline 
        Enrico Bianchi&
        Presente \\
    \hline 
        Francesco Savio&
        Presente \\
    \hline 
        Marko Peric&
        Presente \\
    \hline 
        Pedro Leoni&
        Presente \\
    \hline 

\end{tabularx}}
\end{table}

\newpage

\section{Verbale}
\subsection{Argomenti trattati}
Gli argomenti trattati sono stati i seguenti:
\begin{itemize}
    \item Ricerche delle tecnologie utilizzate
    \item Discussione sul \glossario{Proof of Concept (PoC)} 
    \item Breve controllo dei nuovi \glossario{use case} e dei requisiti software nella documentazione dell'analisi dei requisiti
    \item Dialogo sugli altri documenti: piano di progetto, piano di qualifica e glossario
\end{itemize}

\subsection{Decisioni prese}
In seguito, sono state prese le seguenti decisioni in preparazione dell'incontro per \glossario{RTB}:
\begin{itemize}
    \item Migliorare la presentazione PowerPoint e approfondire l'analisi delle tecnologie
    \item Ottimizzare la visualizzazione del \glossario{Proof of Concept (PoC)} 
    \item Rafforzare l’analisi dei requisiti
\end{itemize}

\section{To Do}
\begin{itemize}
    \item Completare tutti i documenti, inclusi l'Analisi dei requisiti, il Piano di qualifica e il Glossario
    \item Aggiornare il Piano di progetto
    \item Verificare e validare tutti i documenti in vista della release
\end{itemize}

\end{document}
