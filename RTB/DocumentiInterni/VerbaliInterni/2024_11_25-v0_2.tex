\documentclass[a4paper, 12pt]{article}

\usepackage{custom}

%--------------------VARIABILI--------------------
\def\lastversion{v0.2}
\def\title{Verbale interno}
\def\date{28 novembre 2024}
%------------------------------------------------

\begin{document}

\primapagina

\begin{registromodifiche}
        v0.1 & 27 Novembre 2024 & Francesco Savio & Marko Peric & Stesura verbale interno\\
    \hline 
        \lastversion & 19 Dicembre 2024 & Enrico Bianchi & Marko Peric & Miglioramento stesura verbale\\
    \hline 
\end{registromodifiche}

\tableofcontents

\newpage

\section{Registro presenze}
\begin{itemize}
    \item[] \textbf{Data}: 25 Novembre 2024
    \item[] \textbf{Ora inizio}:  21.00
    \item[] \textbf{Ora fine}: 22.30
    \item[] \textbf{Piattaforma}: Discord	
\end{itemize}
\begin{table}[!h]
\centering
{\renewcommand{\arraystretch}{2}
\begin{tabularx}{\textwidth}{| X | X |}
    \hline
        \textbf{\large Componente} & 
        \textbf{\large Presenza} \\ 
    \hline 
    \hline
        Eghosa Matteo Igbinedion Osamwonyi&
        Presente \\
    \hline 
        Guirong Lan&
        Presente \\
    \hline 
        Enrico Bianchi&
        Presente \\
    \hline 
        Francesco Savio&
        Presente \\
    \hline 
        Marko Peric&
        Presente \\
    \hline 
        Pedro Leoni&
        Presente \\
    \hline 

\end{tabularx}}
\end{table}

\newpage

\section{Verbale}
\subsection{Argomenti trattati}
\begin{itemize}
    \item Confronto sui documenti prodotti dal gruppo e chiarimenti dei dubbi di alcuni processi.
    \item Miglioramento sito web con javascript per l'automazione del caricamento dei documenti.
\end{itemize}

\subsection{Decisioni prese}
\begin{itemize}
    \item Il gruppo ha deciso di eseguire uno sprint della durata di una settimana in cui si impegna di fare i seguenti punti.
    \begin{itemize}
        \item Unire in latex tutti i file realizzati nel documento norme di progetto.
        \item Fare i verbali interni.
        \item Migliorare la parte di documentazione sull'analisi dei requisiti.
        \item Finire il documento analisi dei rischi.
        \item Migliorare alcune parti del file prodotti.
        \item Proseguire con il documento analisi dei requisiti.
    \end{itemize}
\end{itemize}

\section{To Do}
    \begin{itemize}
        \item Completare le issue create rispetto alle decisioni prese entro lunedì 2 dicembre (viene indicato tra parentesi il ruolo di competenza per lo svolgimento della issue)
        \begin{itemize}
            \item Revisione e aggiornamento completo delle norme di progetto (amministratore).
            \item Finire il documento analisi dei rischi e verbali interni(responsabile).
            \item verifica documenti (verificatore).
            \item Continuare le stesura del documento analisi dei requisti (analista).
        \end{itemize}
        \item Fare diario di bordo
    \end{itemize}
\end{document}