\documentclass[a4paper, 12pt]{article}

\usepackage{custom}


%--------------------Lettera--------------------
\def\lastversion{}
\def\title{Lettera di presentazione RTB}
\def\date{27 febbraio 2025}
%------------------------------------------------

\begin{document}

\primapagina

\newpage

\noindent
\textbf{Egregio Professor Vardanega}, \\
Con la presente, il gruppo \textit{Alt+F4} desidera candidarsi per la seconda fase della revisione di avanzamento RTB(\textit{Requirements and technology Baseline}) del progetto denominato \textbf{Artificial QI},proposto dall'azienda  \textbf{Zucchetti S.p.A.}\\
\\
La pagina di presentazione del gruppo, insieme alla documentazione relativa all'RTB,è disponibile al seguente indirizzo: 
\begin{center}
    \href{https://alt-f4-eng.github.io/Documentazione}{https://alt-f4-eng.github.io/Documentazione}\\
\end{center}
Desideriamo informarla che, nella suddetta pagina, troverà come i documenti esterni:

\begin{itemize}
    \item \textit{Analisi dei requisiti}
    \item \textit{Piano di progetto}
    \item \textit{Piano di qualifica}
    \item \textit{Verbali esterni}:
    \begin{itemize}
        \item verbale esterno del 2025/03/11
        \item verbale esterno del 2025/01/13
        \item verbale esterno del 2024/12/05
    \end{itemize}
\end{itemize}
\noindent
E come i documenti interni:
\begin{itemize}
    \item \textit{Glossario}
    \item \textit{Norme di progetto}
    \item \textit{Verbali interni}:
    \begin{itemize}
        \item verbale interno del 2024/11/21
        \item verbale interno del 2024/11/25
        \item verbale interno del 2024/11/28
        \item verbale interno del 2024/12/10
        \item verbale interno del 2024/12/19
        \item verbale interno del 2024/12/26
        \item verbale interno del 2025/01/04
        \item verbale interno del 2025/01/09
        \item verbale interno del 2025/01/23
        \item verbale interno del 2025/02/24
        \item verbale interno del 2025/03/11
    \end{itemize}
\end{itemize}
\vspace{0.75cm} 
\noindent
Il gruppo \textit{Alt+F4} è composto dai seguenti membri:
\begin{table}[H]
    \centering
    \begin{tabular}{| l | l |}
    \hline
    \textbf{Nome Cognome} & 
    \textbf{Matricola}\\ 
        \hline
            Enrico Bianchi&
            2040978 \\
        \hline 
            Eghosa Matteo Igbinedion Osamwonyi&
            2042888 \\
        \hline 
            Guirong Lan&
            2042368 \\
        \hline 
            Pedro Leoni&
            2042359 \\
        \hline 
            Marko Peric&
            2011067 \\
        \hline 
            Francesco Savio&
            2085846 \\
        \hline 
    \end{tabular}
\end{table}
\noindent
Cordiali saluti,\\
\textbf{il gruppo \textit{Alt+F4}}

\end{document}
