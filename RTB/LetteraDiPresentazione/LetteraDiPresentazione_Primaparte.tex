\documentclass[a4paper, 12pt]{article}

\usepackage{custom}


%--------------------Lettera--------------------
\def\lastversion{}
\def\title{Lettera di presentazione RTB}
\def\date{27 febbraio 2025}
%------------------------------------------------

\begin{document}

\primapagina

\newpage

\noindent
\textbf{Egregio Professor Cardin}, \\
Con la presente, il gruppo \textit{Alt+F4} desidera comunicare la propria intenzione di sostenere la prima fase della revisione RTB(\textit{Requirements and technology Baseline}) del progetto denominato \textbf{Artificial QI},proposto dall'azienda  \textbf{Zucchetti S.p.A.}\\
\\
Di seguito, vengono riportati i link ai repository pubblici contenenti la documentazione e il codice del progetto:
\begin{itemize}

\item La documentazione di progetto è disponibile al seguente indirizzo: \\ 
\href{https://alt-f4-eng.github.io/Documentazione}{https://alt-f4-eng.github.io/Documentazione}\\


\item Il codice sorgente del PoC (\textit{Proof of Concept}) è disponibile al seguente indirizzo:\\
\href{https://github.com/ALT-F4-eng/PoC}{https://github.com/ALT-F4-eng/PoC}

\end{itemize}

\noindent
Il gruppo \textit{Alt+F4} è composto dai seguenti membri:
\begin{table}[H]
    \centering
    \begin{tabular}{| l | l |}
    \hline
    \textbf{Nome Cognome} & 
    \textbf{Matricola}\\ 
        \hline
            Enrico Bianchi&
            2040978 \\
        \hline 
            Eghosa Matteo Igbinedion Osamwonyi&
            2042888 \\
        \hline 
            Guirong Lan&
            2042368 \\
        \hline 
            Pedro Leoni&
            2042359 \\
        \hline 
            Marko Peric&
            2011067 \\
        \hline 
            Francesco Savio&
            2085846 \\
        \hline 
    \end{tabular}
\end{table}
\noindent
Cordiali saluti,\\
\textbf{il gruppo \textit{Alt+F4}}

\end{document}
