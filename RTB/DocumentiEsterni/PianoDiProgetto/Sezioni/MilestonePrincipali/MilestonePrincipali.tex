\section{Milestone principali}
\label{sec:milestone_principali}
\subsection{RTB: Requirements and Technology Baseline}
Si tratta della prima revisione che viene realizzata per valutare lo stato di avanzamento de lavoro e la qualità con cui è stato svolto.
Raggiunta la data stabilita dal gruppo per l'RTB dovranno essere realizzati i seguenti file:

\begin{itemize}
\item Analisi dei Requisiti
\item Piano di Progetto
\item Piano di Qualifica
\item Proof of Concept (PoC)
\end{itemize}
Il gruppo considera le seguenti date per lo sviluppo dell'RTB:
\begin{itemize}
    \item Data di inizio: 12 Novembre 2024;
    \item Data di fine: 26 Gennaio 2025;
\end{itemize}
L'analisi realizzata ha permesso di stabilire i compiti per ogni ruolo fino alla data dell'RTB:
\begin{itemize}
\item Responsabile: dovrà organizzare ogni settimana le attività da svolgere, il prospetto è di utilizzare il 45\% delle ore totali assegnate.
\item Amministratore: dovrà organizzare gli strumenti necessari allo svolgimento delle attività per l'RTB, mantenendo aggiornato il documento delle Norme di progetto.
Il prospetto è di utilizzare il 59\% delle ore totali assegnate.
\item Analista: per questo ruolo ci si aspetta di utilizzare quasi interamente le ore assegnate (vengono tenute ore a disposizione che facciano da cuscinetto in caso di necessità), avrà il compito di realizzare l'analisi dei requisiti e di effettuare la stesura del documento.
Il prospetto è di utilizzare il 90\% delle ore totali assegnate.
\item Progettista: ci si aspetta non vengano utilizzate le sue ore assegnate in quanto, siccome il PoC consiste in un artefatto usa e getta, non verrà realizzata progettazione del software fino alla data dell'RTB.
Il prospetto è di utilizzare lo 0\% delle ore totali assegnate.
\item Programmatore: avrà il compito di realizzare il PoC per dimostrare la fattibilità del progetto e la capacità del gruppo di realizzarlo.
Il prospetto è di utilizzare il 27\% delle ore totali assegnate.
\item Verificatore: avrà il compito di effettuare la verifica dei documenti realizzati e dei requisiti analizzati dall'analista, non ci si aspetta verifica di codice per il PoC, non necessaria in quanto è un artefatto usa e getta.
Il prospetto è di utilizzare il 30\% delle ore totali assegnate.
\end{itemize}
E' stata realizzata una stima delle ore necessarie e quindi anche una stima dei costi fino all'RTB:
\begin{table}[H]
    \centering
    \resizebox{\textwidth}{!}{
    \begin{tabular}{| l | l | l | l |}
        \hline
            \textbf{Ruolo} &  
            \textbf{Ore per ruolo} & 
            \textbf{Costo orario} &
            \textbf{Costo totale} \\
        \hline
        \hline
            Responsabile & 20 & 30€ & 600€ \\
        \hline
            Amministratore & 32 & 20€ & 640€ \\
        \hline 
            Analista & 70 & 25€ & 1750€ \\
        \hline 
            Progettista & 0 & 25€ & 0€ \\
        \hline 
            Programmatore & 40 & 15€ & 600€ \\
        \hline 
            Verificatore & 45 & 15€ & 675€ \\
        \hline 
            \textbf{Totale} & \textbf{207} & \textbf{N/A} & \textbf{4265€} \\ 
        \hline  
    \end{tabular}}
    \caption{Ripartizione ore e costi fino RTB.}
    \label{tab:stima_costi_RTB} 
\end{table}

In seguito viene mostrato un diagramma a torta in cui viene indicata la ripartizione dei ruoli in percentuale rispetto le ore totali preventivate per la realizzazione dei prodotti da consegnare entro la data stabilita per l'RTB.

\begin{figure}[H]
    \centering
    \begin{tikzpicture}
        \pie{9.66/Responsabile, 15.46/Amministratore, 33.82/Analista, 19.32/Programmatore, 21.74/Verificatore}
    \end{tikzpicture}
    \caption{Ripartizione ore e ruoli fino a RTB.}
    \label{fig:pie_ruoli_RTB}
\end{figure}

\subsection{PB: Product Baseline}
Durante questa fase verrà esposto il proprio prodotto alla azienda proponente che effettuerà la valutazione per l'MVP (Minimum Viable Product).
Arrivati a questa fase verranno valutati la baseline architetturale del progetto software e la sua realizzazione, sarà necessario avere un design definitivo 
che verrà descritto nel documento di Specifica Tecnica.
Nello specifico verranno valutati l'architettura di deployment, che corrisponde alla allocazione delle componenti architetturali nel sistema, 
i design pattern utilizzati e ogni altro aspetto architetturale che definisce il design per la progettazione del software prodotto.
Il risultato sarà l'accettazione del software come MVP, che deve approssimare al meglio il prodotto atteso rispettando i requisiti funzionali 
e gli obiettivi di qualità prefissati inizialmente.