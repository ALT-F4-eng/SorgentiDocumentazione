\subsection{Rischi tecnologici}

\subsubsection{RT1: Inesperienza sulle tecnologie da adottare}
\begin{itemize}
    \item \textbf{Descrizione}: La mancata competenza di alcuni o tutti i membri del gruppo nell'utilizzo di tecnologie utili alla realizzazione del progetto potrebbero causare rallentamenti nello sviluppo. 
    \item \textbf{Pericolosità}: Alta.
    \item \textbf{Probabilità}: Alta.
    \item \textbf{Prevenzione}: All'inizio di ogni sprint è necessario che, a fronte delle attività che si dovranno svolgere, ogni membro del gruppo esponga eventuali dubbi per quanto riguarda l'utilizzo di tecnologie adottate. 
    Sarà inoltre compito del responsabile controllare alla fine di ogni sprint le ore assegnate inizialmente e poi quelle realmente impiegate per svolgere ogni attività, in modo da comprendere quali tecnologie hanno causato rallentamenti durante lo svolgimento.
    \item \textbf{Mitigazione}:  Innanzitutto ogni membro garantisce impegno individuale nello studio delle tecnologie necessarie.
    È stato realizzato un documento condiviso su cui ogni membro andrà ad aggiungere una descrizione degli argomenti appresi durante lo studio individuale.
    Così facendo tutto il gruppo ha la possibilità di restare aggiornato sulle tecnologie che potrebbero essere adottate nello sviluppo del progetto.
    È stata creata una chat Discord in cui verranno condivisi articoli e documenti inerenti alle tecnologie studiate.
    Ogni membro del gruppo è inoltre disponibile per chiarimenti e supporto per quanto riguardano le tecnologie in cui possiede maggiore esperienza.
\end{itemize}

\subsubsection{RT2: Malfunzionamenti hardware}
\begin{itemize}
    \item \textbf{Descrizione}: Eventuali guasti e anomalie ai componenti hardware utilizzati dai membri del gruppo potrebbero causare 
    interruzioni nello sviluppo del progetto con la possibilità di ritardi nella consegna.
    \item \textbf{Pericolosità}: Alta.
    \item \textbf{Probabilità}: Bassa.
    \item \textbf{Prevenzione}: È necessaria una segnalazione immediata a tutti i membri del gruppo in caso di un guasto al proprio sistema.
    \item \textbf{Mitigazione}: Nel caso in cui un membro non fosse disponibile per via di guasti hardware il responsabile deve  rivalutare il Piano di Progetto e rivedere le risorse disponibili per lo sprint in base alla disponibilità degli altri membri.
    È inoltre essenziale lavorare maggiormente possibile in un ambiente condiviso per ridurre la perdita di informazioni.
\end{itemize}

\subsubsection{RT3: Inadeguata conoscenza degli strumenti di sviluppo software}
\begin{itemize}
    \item \textbf{Descrizione}: Alcuni membri del gruppo potrebbero avere competenze ridotte nell'utilizzo di software di terze parti necessari allo sviluppo.
    Alcuni esempi di tali strumenti sono: sistemi di versionamento del codice, di automazione della build o di containerizzazione. 
    \item \textbf{Pericolosità}: Alta.
    \item \textbf{Probabilità}: Alta.
    \item \textbf{Prevenzione}: Ogni membro del gruppo deve verificare la sua capacità di utilizzo degli strumenti adottati e segnalare ai compagni eventuali dubbi.
    \item \textbf{Mitigazione}: L'amministratore deve aggiornare continuamente le Norme di Progetto con una descrizione esaustiva e comprensibile a tutti i membri degli strumenti adottati, dei loro casi d'uso e del loro funzionamento.
\end{itemize}

\subsubsection{RT4: Incompatibilità tra strumenti e tecnologie adottate}
\begin{itemize}
    \item \textbf{Descrizione}: Le tecnologie che verranno adottate, come librerie o framework, potrebbero non essere completamente compatibili tra loro portando allo sviluppo di un software inefficiente e malfunzionante.
    \item \textbf{Pericolosità}: Alta.
    \item \textbf{Probabilità}: Media.
    \item \textbf{Prevenzione}: Durante lo sviluppo del \glossario{Proof of Concept} verranno analizzati diversi strumenti e tecnologie in modo da verificare quali possono integrarsi senza problemi.
    \item \textbf{Mitigazione}: È importante lo studio approfondito della documentazione di ogni tecnologia adottata in modo da comprendere a priori se possono insorgere problemi di compatibilità con altre tecnologie.
    Sarà inoltre compito di programmatore e verificatore, durante lo sviluppo, di tenere traccia di eventuali errori e Malfunzionamenti che possono derivare da problemi di incompatibilità.
\end{itemize}