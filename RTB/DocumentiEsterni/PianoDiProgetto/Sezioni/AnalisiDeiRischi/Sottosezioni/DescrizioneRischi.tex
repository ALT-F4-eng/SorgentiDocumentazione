\subsection{Descrizione dei rischi}
I rischi individuati verranno descritti nel seguente modo:
\begin{itemize}
    \item \textbf{tipologia}: cioè la categoria a cui appartiene il rischio, può essere organizzativa o tecnologica.
    \item \textbf{indice}: indice numerico che identifica un rischio per ogni tipologia.
    \item \textbf{descrizione}: descrizione dl rischio.
    \item \textbf{pericolosità}: può essere bassa, media o alta e indica l'impatto che avrebbe sul progetto il suo verificarsi.
    \item \textbf{probabilità}: può essere bassa, media o alta e indica la probabilità del rischio di verificarsi.
    \item \textbf{prevenzione}: indica la metodologia utilizzare per identificare il rischio.
    \item \textbf{mitigazione}: azioni da intraprendere quando si verifica il rischio.   
\end{itemize}
Ogni rischio verrà indicato con la sigla \texttt{R[tipologia][identificativo]}.