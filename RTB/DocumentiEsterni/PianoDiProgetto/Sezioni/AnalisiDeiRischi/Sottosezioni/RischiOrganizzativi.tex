\subsection{Rischi organizzativi}

\subsubsection{RO1: Impegni universitari}
\begin{itemize}
    \item \textbf{Descrizione}: la preparazione per esami universitari può diminuire la disponibilità di uno o più membri del gruppo.
    \item \textbf{Pericolosità}: Media.
    \item \textbf{Probabilità}: Alta.
    \item \textbf{prevenzione}: è importante condividere le date in cui si verranno svolti gli esami e i periodi in cui si pensa di poter dare meno disponibilità per il progetto.
    \item \textbf{Mitigazione}: sarà compito del responsabile gestire il carico di lavoro e le attività da svolgere, basandosi sulle ore disponibili per ogni membro, in modo tale da garantire il rispetto delle scadenze. 
    In casi gravi il responsabile dovrà occuparsi della modifica delle risorse per lo sprint o della scadenza.
\end{itemize}

\subsubsection{RO2: Impegni personali}
\begin{itemize}
    \item \textbf{Descrizione}: un membro potrebbe garantire una minore disponibilità durante un periodo a causa di impegni di natura personale.
    \item \textbf{Pericolosità}: Bassa.
    \item \textbf{Probabilità}: Media.
    \item \textbf{prevenzione}: importante avvisare tempestivamente e con il maggior anticipo possibile i giorni in cui non si sarà disponibili. 
    \item \textbf{Mitigazione}: è compito del responsabile calcolare le ore disponibili del gruppo durante lo sprint e organizzare il carico di lavoro tenendo conto di tutti gli impegni precedentemente comunicati dai compagni. 
\end{itemize}

\subsubsection{RO3: Errori di pianificazione delle attività}
\begin{itemize}
    \item \textbf{Descrizione}: l'inesperienza dei membri del gruppo nel contesto della pianificazione di progetto è un fattore di rischio che può interferire sulla corretta realizzazione del progetto. 
    Nello specifico, la mancata esperienza nell'ambito può portare a una sottostima delle risorse necessarie allo sviluppo delle attività o a un errata valutazione delle tempistiche. 
    \item \textbf{Pericolosità}: Alta.
    \item \textbf{Probabilità}: Alta.
    \item \textbf{prevenzione}: continuo controllo del Piano di Progetto per monitorare le risorse utilizzate e il tempo in cui sono state impiegate.
    Aggiornamento tra i membri del gruppo tramite Telegram sul proseguo nello sviluppo delle attività.
    \item \textbf{Mitigazione}: in caso dovessero verificarsi difficoltà o ritardi è compito del responsabile rivedere il Piano di Progetto e modificare il numero delle risorse allocate per lo sprint o, in casi gravi, la scadenza.
\end{itemize}

\subsubsection{RO4: Scarsa collaborazione di uno o più membri del gruppo}
\begin{itemize}
    \item \textbf{Descrizione}: rischio dovuto alla possibilità che uno o più membri impieghino scarso impegno nella realizzazione del progetto.
    \item \textbf{Pericolosità}: Alta.
    \item \textbf{Probabilità}: Bassa.
    \item \textbf{prevenzione}: è stata da subito richiesta la massima trasparenza tra i membri del gruppo.
    Sarà compito del responsabile verificare che ogni membro abbia partecipato equamente nella realizzazione delle attività dello sprint.
    \item \textbf{Mitigazione}: Il responsabile dovrà contattare e avvisare il membro che, senza preavviso, non avrà impiegato le giuste risorse personali nello sviluppo delle attività dello sprint, in modo tale da chiarire eventuali complicanze o fraintendimenti. 
\end{itemize}

\subsubsection{RO5: Disaccordi all'interno del gruppo}
\begin{itemize}
    \item \textbf{Descrizione}: possono generarsi discussioni a causa di opinioni e ideologie diverse all'interno del gruppo.
    \item \textbf{Pericolosità}: Alta.
    \item \textbf{Probabilità}: Media.
    \item \textbf{prevenzione}: importante che vi sia chiarezza e trasparenza durante le chiamate del gruppo, bisogna dichiarare da subito eventuali discordanze su idee e opinioni in modo da trovare un punto d'incontro più rapidamente possibile.
    \item \textbf{Mitigazione}: in caso in cui non si riesca a raggiungere un compromesso definitivo le varie opzioni su cui si sta discutendo verranno messe a votazione, verrà scelta l'opzione, o le opzioni, che hanno la maggioranza dei voti. 
\end{itemize}

\subsubsection{RO6: Risorse disponibili ma non impiegate}
\begin{itemize}
    \item \textbf{Descrizione}: ruoli come amministratore, progettista o verificatore passeranno dei periodi in cui avranno un carico di lavoro e uno sforzo produttivo inferiore ad altri ruoli. 
    Se quindi uno o più membri del gruppo focalizzeranno il proprio lavoro solo su uno di questi ruoli potrebbero non impiegare tutte le loro ore produttive disponibili durante lo sprint.
    \item \textbf{Pericolosità}: Alta.
    \item \textbf{Probabilità}: Bassa.
    \item \textbf{prevenzione}: viene creata una tabella per la rendicontazione delle ore per ogni sprint nel piano di progetto che ogni membro dovrà compilare.
    Il responsabile potrà quindi, attraverso la tabella, verificare le attività e le ore impiegate per ogni ruolo in modo tale da rilevare se c'è una disponibilità di risorse non sfruttate.
    \item \textbf{Mitigazione}:  Per ogni sprint viene creata una dashboard su GitHub con tutte le attività da svolgere durante lo sprint.
    Ogni membro potrà assegnarsi attività da svolgere anche di ruoli diversi in modo tale che ognuno impieghi nel modo piè efficiente possibile le proprie ore produttive.
\end{itemize}