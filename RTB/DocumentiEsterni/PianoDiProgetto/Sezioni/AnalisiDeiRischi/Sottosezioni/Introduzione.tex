\subsection{Introduzione}
L'analisi dei rischi ha lo scopo di identificare, valutare e mitigare i vari problemi che potrebbero compromettere la buona riuscita del progetto. 
Quindi è necessario prevedere eventi indesiderati che potrebbero influenzare negativamente i tempi, i costi e la qualità del prodotto finale.
Il processo di gestione dei rischi è suddiviso in quattro fasi:
\begin{enumerate}
    \item \textbf{identificazione}: scovare il maggior numero di fattori di rischio per il progetto.
    \item \textbf{analisi}: studio delle probabilità di occorrenza dei rischi e le conseguenze nel caso in cui dovessero verificarsi.
    \item \textbf{pianificazione}: fase in cui si analizza come evitare i rischi e come mitigarne gli effetti.
    \item \textbf{controllo}: durante questa fase vi è una continua verifica di rilevazioni di rischi e l'attuazione delle strategie di mitigazione. 
\end{enumerate}