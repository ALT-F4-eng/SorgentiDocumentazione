\section{Introduzione}
\label{sec:introduzione_pq}
Il seguente documento definisce il Piano di Qualifica del software Artificial QI, descrivendo le attività  pianificate dal gruppo
per garantire il raggiungimento degli obiettivi di qualità, sia in termini di prodotto che di processi impiegati nella realizzazione.
La qualifica del software rappresenta un passaggio fondamentale per assicurare che il ciclo di sviluppo rispetti elevati standard di qualità, 
promuovendo l'efficienza operativa e la conformità alle specifiche tecniche e normative.
Oltre a ciò, la definizione di metriche di qualità permette di stabilire criteri oggettivi che permettono di validare il prodotto finale 
assicurandone l'affidabilità e la completezza.
Questo documento andrà ad indicare:
\begin{itemize}
    \item gli obiettivi di qualità e le metriche associate
    \item le strategie di verifica e validazione per i processi e i prodotti realizzato
    \item la strategia di test che andrà a verificare il codice del software realizzato
    \item cruscotto di valutazione della qualità
\end{itemize}
Il Piano di Qualifica si propone di garantire una visione chiara e completa delle attività di controllo qualità, 
assicurando che il prodotto finale soddisfi pienamente gli standard prefissati e supporti il successo del progetto.
\subsection{Riferimenti}
\subsubsection{Normativi}
\begin{itemize}
    \item Norme di progetto
    \item Regolamento del progetto: \url{https://www.math.unipd.it/~tullio/IS-1/2024/Dispense/T03.pdf}
\end{itemize}
\subsubsection{Informativi}


