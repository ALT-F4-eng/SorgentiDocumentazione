\subsection{Metriche per processo}
\label{subsec:processi_metriche}
Per ciascun processo delineato dallo standard ISO/IEC 12207 e descritto nel documento delle Norme di Progetto, 
vengono indicate le \glossario{metriche} di riferimento, se disponibili.

\subsubsection{Processi primari}
\paragraph{Fornitura}
\begin{table}[H]
    \centering
    \begin{tabular}{| l | l | l | l |}
    \hline
    \textbf{Identificativo} & 
    \textbf{Nome} &
    \textbf{Valore ammissibile} &
    \textbf{Valore ottimo}\\
    \hline
        M.PC.PV & Planned Value & $\geq 0$ & $\leq BAC$ \\
    \hline
        M.PC.EV & Earned Value & $\geq 0$ & $\leq EAC$ \\
    \hline
        M.PC.AC & Actual Cost & $\geq 0$ & $\leq EAC$ \\
    \hline
        M.PC.SV & Schedule Variance & $\geq -15\%$ & $0\%$ \\
    \hline
        M.PC.CV & Cost Variance & $\geq -15\%$ & $0\%$ \\
    \hline  
        M.PC.VP & Variazione del Piano & $\leq 20\%$ & $\leq 5\%$ \\
    \hline
        M.PC.EAC & Estimated at Completion & $\leq BAC+5\% BAC$ & $\leq BAC$ \\
    \hline
\end{tabular}
\caption{Metriche per processo di Fornitura}
\label{tab:metriche_fornitura} 
\end{table}

\paragraph{Sviluppo}
\begin{table}[H]
    \centering
    \begin{tabular}{| l | l | l | l |}
    \hline
    \textbf{Identificativo} & 
    \textbf{Nome} &
    \textbf{Valore ammissibile} &
    \textbf{Valore ottimo}\\
    \hline
        M.PR.PRM & \makecell{Percentuale requisiti \\ obbligatori soddisfatti} & $100\%$ & $100\%$ \\
    \hline
        M.PR.PRO & \makecell{Percentuale requisiti \\ opzionali soddisfatti} & $\geq 0\%$ & $100\%$ \\
    \hline    
\end{tabular}
\caption{Metriche per processo di Sviluppo}
\label{tab:metriche_sviluppo} 
\end{table}

\subsubsection{Processi di supporto}
\paragraph{Documentazione}
\begin{table}[H]
    \centering
    \resizebox{\textwidth}{!}{
    \begin{tabular}{| l | l | l | l |}
    \hline
        \textbf{Identificativo} & 
        \textbf{Nome} &
        \textbf{Valore ammissibile} &
        \textbf{Valore ottimo}\\
    \hline
        M.PR.CO & Correttezza Ortografica & $0$ & $0$ \\
    \hline
\end{tabular}}
\caption{Metriche per processo di Documentazione}
\label{tab:metriche_documentazione} 
\end{table}

\paragraph{Accertamento della Qualità}
\begin{table}[H]
    \centering
    \resizebox{\textwidth}{!}{
    \begin{tabular}{| l | l | l | l |}
    \hline
        \textbf{Identificativo} & 
        \textbf{Nome} &
        \textbf{Valore ammissibile} &
        \textbf{Valore ottimo}\\
    \hline
        M.PC.MS & Metriche Soddisfatte & $\geq 75\%$ & $100\%$ \\
    \hline
\end{tabular}}
\caption{Metriche per processo di Accertamento della Qualità}
\label{tab:metriche_accertamento} 
\end{table}

\subsubsection{Processi Organizzativi}
\paragraph{Gestione dei Rischi}
\begin{table}[H]
    \centering
    \resizebox{\textwidth}{!}{
    \begin{tabular}{| l | l | l | l |}
    \hline
        \textbf{Identificativo} & 
        \textbf{Nome} &
        \textbf{Valore ammissibile} &
        \textbf{Valore ottimo}\\
    \hline
        M.PC.RMR & Risk Mitigation Rate & $\geq 75\%$ & $100\%$ \\
    \hline
\end{tabular}}
\caption{Metriche per processo di Gestione dei Rischi}
\label{tab:metriche_rischi} 
\end{table}
