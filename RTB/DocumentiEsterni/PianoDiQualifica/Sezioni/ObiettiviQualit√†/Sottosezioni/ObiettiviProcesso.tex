\subsection{Qualità di processo}
La qualità di processo si riferisce all'efficacia con cui vengono implementati e gestiti i processi durante il ciclo di vita dello sviluppo software, 
con l'obiettivo di garantire che il prodotto finale soddisfi i requisiti prefissati. 
Per monitorare e migliorare i processi, vengono adottate metriche di processo, ovvero indicatori chiave che misurano l'efficienza, l'affidabilità 
e la conformità delle attività svolte. 
Questi parametri, selezionati dal team, consentono di identificare aree critiche, ottimizzare le procedure operative e migliorare la produttività complessiva. 
L'uso delle metriche di processo contribuisce al controllo della qualità e alla riduzione dei rischi associati a ritardi o difetti.
Per ogni metrica di processo sarà indicato anche il processo specifico a cui essa si applica, evidenziando come la metrica contribuisca a monitorare 
e valutare la qualità delle attività svolte all'interno di quel contesto operativo.