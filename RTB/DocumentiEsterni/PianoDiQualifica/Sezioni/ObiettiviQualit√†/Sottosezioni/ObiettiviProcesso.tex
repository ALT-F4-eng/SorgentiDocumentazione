\subsection{Qualità di processo}
\label{subsec:obiettivi_processo}
La qualità di processo si riferisce all'efficacia con cui vengono implementati e gestiti i processi durante il ciclo di vita dello sviluppo software, 
con l'obiettivo di garantire che il prodotto finale soddisfi i requisiti prefissati. 
Per monitorare e migliorare i processi, vengono adottate metriche di processo, ovvero indicatori chiave che misurano l'efficienza, l'affidabilità 
e la conformità delle attività svolte. 
Questi parametri, selezionati dal team, consentono di identificare aree critiche, ottimizzare le procedure operative e migliorare la produttività complessiva. 
L'uso delle metriche di processo contribuisce al controllo della qualità e alla riduzione dei rischi associati a ritardi o difetti.


\begin{table}[H]
    \centering
    \begin{tabular}{| l | l | l | l |}
    \hline
    \textbf{Identificativo} & 
    \textbf{Nome} &
    \textbf{Valore ammissibile} &
    \textbf{Valore ottimo}\\
    \hline
        M.PC.PV & Planned Value & $\geq 0$ & $\leq BAC$ \\
    \hline
        M.PC.EV & Earned Value & $\geq 0$ & $\leq EAC$ \\
    \hline
        M.PC.AC & Actual Cost & $\geq 0$ & $\leq EAC$ \\
    \hline
        M.PC.SV & Schedule Variance & $\geq -15\%$ & $0\%$ \\
    \hline
        M.PC.CV & Cost Variance & $\geq -15\%$ & $0\%$ \\
    \hline  
        M.PC.VP & \makecell{Variazione del Piano \\ tra costo effettivo \\ e costo preventivato} & $\leq 20\%$ & $\leq 5\%$ \\
    \hline
        M.PC.EAC & Estimated at Completion & $\leq BAC+5\% BAC$ & $\leq BAC$ \\
    \hline
        M.PC.RMR & Risk Mitigation Rate & $\geq 75\%$ & $100\%$ \\
    \hline
        M.PC.MS & Metriche Soddisfatte & $\geq 75\%$ & $100\%$ \\
    \hline
    \end{tabular}
    \caption{Metriche di processo}
    \label{tab:metriche_processo} 
\end{table}