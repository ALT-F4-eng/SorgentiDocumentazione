\subsection{Qualità di prodotto}
\label{subsec:obiettivi_prodotto}
La qualità di prodotto si riferisce al grado in cui un software soddisfa i requisiti specificati e le aspettative degli utenti.
Per valutarla vengono utilizzate metriche di prodotto, che rappresentano indicatori chiave per valutare le caratteristiche principali del software, che sono:
funzionalità, affidabilità, efficienza, usabilità, manutenibilità e portabilità.
Questi indicatori permettono di identificare lacune all'interno del codice permettendo un monitoraggio e un miglioramento continuo del prodotto
e assicurando che rispetti i requisiti funzionali e non funzionali prefissati.
L'adozione di metriche di prodotto permette quindi di assicurare la qualità del prodotto software realizzato e di ottimizzare l'esperienza 
dell'utente finale.

\begin{table}[H]
    \centering
    \resizebox{\textwidth}{!}{
    \begin{tabular}{| l | l | l | l |}
    \hline
        \textbf{Identificativo} & 
        \textbf{Nome} &
        \textbf{Valore ammissibile} &
        \textbf{Valore ottimo}\\
    \hline
        M.PR.PRM & Percentuale requisiti obbligatori soddisfatti & $100\%$ & $100\%$ \\
    \hline    
        M.PR.PRO & Percentuale requisiti opzionali soddisfatti & $\geq 0\%$ & $100\%$ \\
    \hline
        M.PR.CO & Correttezza Ortografica & $0$ & $0$ \\
    \hline
    \end{tabular}}
    \caption{Metriche di prodotto}
    \label{tab:metriche_prodotto} 
\end{table}

\subsubsection{Caratteristiche di prodotto}
Di seguito vengono elencate le metriche di prodotto associate a ciascuna delle caratteristiche generali descritte dallo standard \glossario{ISO/IEC 9126}, 
come indicato nel processo di Accertamento della Qualità nelle Norme di Progetto.
\begin{table}[H]
    \centering
    \begin{tabularx}{\textwidth}{| X | X | X |}
    \hline
        \textbf{Caratteristica} & 
        \textbf{Descrizione} &
        \textbf{Metriche associate}\\
    \hline
        Funzionalità & Capacità del software di fornire funzioni adatte a rispettare i requisiti sviluppati nel documento Analisi dei Requisiti & M.PR.PRM, M.PR.PRO \\
    \hline
        Affidabilità & Capacità del software di mantenere uno specificato livello di prestazioni in presenza di errori o malfunzionamenti & \\
    \hline
        Efficienza & Capacità del software di fornire appropriate prestazioni in relazione alle risorse usate & \\
    \hline
        Usabilità & Capacità del software di facilitare il reperimento delle informazioni dall'utente in modo che siano propriamente comprese & M.PR.CO \\
    \hline
        Manutenibilità & Facilità nella modifica del software per l'aggiunta di nuove funzionalità & \\
    \hline
        Portabilità & Capacità del software di essere adattato a differenti ambienti operativi & \\
    \hline
    \end{tabularx}
    \caption{Caratteristiche di prodotto}
    \label{tab:caratteristiche_prodotto} 
\end{table}





