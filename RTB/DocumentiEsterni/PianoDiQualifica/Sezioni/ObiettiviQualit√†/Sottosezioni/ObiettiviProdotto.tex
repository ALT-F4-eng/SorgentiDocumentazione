\subsection{Qualità di prodotto}
La qualità di prodotto si riferisce al grado in cui un software soddisfa i requisiti specificati e le aspettative degli utenti.
Per valutarla vengono utilizzate metriche di prodotto, che rappresentano indicatori chiave per valutare le caratteristiche principali del software, come 
funzionalità, affidabilità, efficienza e manutenibilità.
Questi indicatori permettono di identificare lacune all'interno del codice permettendo un monitoraggio e un miglioramento continuo del prodotto
e assicurando che rispetti i requisiti funzionali e non funzionali prefissati.
L'adozione di metriche di prodotto permette quindi di assicurare la qualità del prodotto software realizzato e di ottimizzare l'esperienza 
dell'utente finale.