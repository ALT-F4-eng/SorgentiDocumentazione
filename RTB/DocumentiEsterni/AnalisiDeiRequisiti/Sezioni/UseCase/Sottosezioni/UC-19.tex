\subsection{UC-19}
\label{subsec:UC-19}

\begin{usecase}{UC-19}{Visualizzazione diagramma di dispersione dei risultati di un test}

    \req{\hyperref[item:RU-6]{RU-6}} 

    \pre{
        \item Il sistema è attivo e funzionante
        \item È stato eseguito correttamente il test sul dataset caricato
    }

    \post{
        \item L'utente visualizza un diagramma di dispersione
    }
    
    \actor{Utente}

    \subactors{LLM}

    \trigger{}
    
    \inc{}

    \base{}

    \scenario{
        \item Visualizzazione di un grafico di dispersione.
        L'asse delle ordinate rappresenta il numero della domanda.
        L'asse delle ascisse rappresenta il grado di somiglianza tra la risposta attesa e la risposta effettiva.
        
        \item Rappresentazione dei risultati come punti distinguibili tra risposta corretta e sbagliata.
        
        \item Calcolo della media e rappresentazione come una retta nel grafico.
        
        \item Calcolo della deviazione standard come una retta nel grafico.
        
        \item Mostrare la legenda del grafico.
    }

\end{usecase}