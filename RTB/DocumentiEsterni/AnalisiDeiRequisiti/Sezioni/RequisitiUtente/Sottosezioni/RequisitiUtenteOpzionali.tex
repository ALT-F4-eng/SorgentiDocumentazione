\subsection{Requisiti utente opzionali}
\begin{table}[H]
    \begin{tabularx}{\textwidth}{|c|X|}
        \hline
        \textbf{ID} & \textbf{Descrizione} \\
        \hline
        \label{ru:RUF-1} RUF-1 &  L'utente dovrebbe poter gestire i risultati dei test(salvare, rinominare, visualizzare ed eliminare)\\
        \label{ru:RUF-2} RUF-2 & L'utente dovrebbe poter ricercare i test salvati per nome \\
        \label{ru:RUF-3} RUF-3 & L'utente dovrebbe poter visualizzare i risultati di un test salvato \\
        \label{ru:RUF-4} RUF-4 & L'utente dovrebbe poter salvare un dataset usando un file in formato strutturato \\
        \label{ru:RUF-5} RUF-5 & L'utente dovrebbe poter confrontare i risultati di due test archiviati \\
        \label{ru:RUF-6} RUF-6 & L'utente dovrebbe poter gestire i modelli da testare(salvare, modificare ed eliminare)\\
        \hline
    \end{tabularx}
    \vspace{10px}
    \caption{Requisiti utente opzionali}
\end{table}