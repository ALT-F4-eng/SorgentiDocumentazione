\subsection{Requisiti utente obbligatori}
\begin{table}[H]
    \begin{tabularx}{\textwidth}{|c|X|}
        \hline
        \textbf{ID} & \textbf{Descrizione} \\
        \hline
        \label{ru:RUO-1} RUO-1 & L'utente deve poter gestire il contenuto del dataset caricato come corrente(visualizzare, modificare, creare ed eliminare elementi)\\
        \label{ru:RUO-2} RUO-2 & L'utente deve poter cercare un insieme di elementi nel dataset tramite parole chiave \\
        \label{ru:RUO-3} RUO-3 & L'utente deve poter gestire un insieme di dataset salvati (visualizzare, rinominare, creare, copiare ed eliminare)\\
        \label{ru:RUO-4} RUO-4 & L'utente deve poter cercare un insieme di dataset salvati tramite parole chiave \\
        \label{ru:RUO-5} RUO-5 & L'utente deve poter archiviare il dataset caricato \\
        \label{ru:RUO-6} RUO-6 & L'utente deve poter caricare un dataset precedentemente archiviato \\
        \label{ru:RUO-7} RUO-7 & L'utente deve poter eseguire il test sul dataset caricato \\
        \label{ru:RUO-8} RUO-8 & L'utente deve poter visualizzare i risultati dell'esecuzione del test \\
        \hline
    \end{tabularx}
    \vspace{10px}
    \caption{Requisiti utente obbligatori}
\end{table}

