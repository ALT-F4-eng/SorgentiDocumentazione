\section{Requisiti utente}
\label{sec:requisiti_utente}
I \glossario{requisiti utente} catturano le funzionalità che il sistema deve offrire agli utenti per poter colmare la loro necessità.
Questa tipologia di requisiti viene prodotta tenendo a mente la prospettiva degli utenti finali.
I \glossario{requisiti utente} vengono estratti dal \glossario{capitolato}.
Per semplificare la lettura e la comprensione dei \glossario{requisiti utente} sono stati rappresentati in forma tabellare divisi per \glossario{requisiti utente} obbligatori e \glossario{requisiti utente} opzionali.
A ogni requisito utente viene assegnato un identificativo univoco che segue la forma:
I \glossario{requisiti utente} obbligatori vengono identificati come segue:
\begin{lstlisting}
    RU[tipo]-[ID_numerico]
\end{lstlisting}
Dove l'\lstinline{ID_numerico} è un valore intero positivo crescente mentre tipo può assumere i valori \lstinline{O} per obbligatorio o \lstinline{F} per facoltativo.

\input{Sezioni/RequisitiUtente/Sottosezioni/RequisitiUtenteObbligatori.tex}

\input{Sezioni/RequisitiUtente/Sottosezioni/RequisitiUtenteOpzionali.tex}