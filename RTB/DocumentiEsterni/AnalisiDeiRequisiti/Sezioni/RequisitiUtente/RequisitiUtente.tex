\section{Requisiti utente}
\label{sec:requisiti_utente}
I requisiti utente catturano le funzionalità che il sistema deve offrire agli utenti per poter colmare la loro necessità.
Questa tipologia di requisiti viene prodotta tenendo a mente la prospettiva degli utenti finali.
I requisiti utente vengono estratti dal capitolato.
Per semplificare la lettura e la comprensione dei requisiti utente sono stati rappresentati in forma tabellare divisi per requisiti utente obbligatori e requisiti utente opzionali.
Ad ogni requisito utente viene assegnato un identificativo univoco che segue la forma:
I requisiti utente obbligatori vengono identificati come segue:
\begin{lstlisting}
    RU[tipo]-[ID_numerico]
\end{lstlisting}
Dove l'\lstinline{ID_numerico} è un valore intero positivo crescente mentre tipo può assumere i valori \lstinline{O} per obbligatorio o \lstinline{F} per facoltativo.

\subsection{Requisiti utente obbligatori}
\begin{tabularx}{\textwidth}{|c|X|}
    \hline
    \textbf{ID} & \textbf{Descrizione} \\
    \hline
    \label{ru:RUO-1} RUO-1 & L'utente deve poter gestire il contenuto del dataset caricato come corrente(visualizzare, modificare, creare ed eliminare elementi)\\ 
    \hline
    \label{ru:RUO-2} RUO-2 & L'utente deve poter cercare un insieme di elementi nel dataset tramite parole chiave \\ 
    \hline
    \label{ru:RUO-3} RUO-3 & L'utente deve poter gestire un insieme di dataset salvati (visualizzare, rinominare, creare, copiare ed eliminare)\\ \hline
    \label{ru:RUO-4} RUO-4 & L'utente deve poter cercare un insieme di dataset salvati tramite parole chiave \\ 
    \hline
    \label{ru:RUO-5} RUO-5 & L'utente deve poter archiviare il dataset caricato \\ 
    \hline
    \label{ru:RUO-6} RUO-6 & L'utente deve poter caricare un dataset precedentemente archiviato \\ 
    \hline
    \label{ru:RUO-7} RUO-7 & L'utente deve poter eseguire il test sul dataset caricato \\ \hline
    \label{ru:RUO-8} RUO-8 & L'utente deve poter visualizzare i risultati dell'esecuzione del test \\ \hline
    \label{ru:RUO-9} RUO-9 & Le funzionalità devono essere fornite da un unico sistema\\ \hline
    \label{ru:RUO-10} RUO-10 & Il test non deve essere troppo lento \\ 
    \hline 
    \label{ru:RUO-11} RUO-11 & Il sistema deve rispettare un grado minimo di accessibilità \\ 
    \hline
    \label{ru:RUO-12} RUO-12 & La logica di esecuzione dei test è particolarmente interessante e potrebbe essere modificata/estesa \\ 
    \hline
    \label{ru:RUO-13} RUO-13 & Il sistema deve essere pubblicato di modo da essere accessibile alla proponente \\ 
    \hline
    \label{ru:RUO-14} RUO-14 & L'utente deve disporre delle risorse necessarie per apprendere a utilizzare il sistema \\ 
    \hline
    \label{ru:RUO-15} RUO-15 &  Il sistema deve essere sviluppato come una "web app"\\ 
    \hline
    \label{ru:RUO-16} RUO-16 &  Deve essere compatibile con le ultime versioni dei browser più comuni \\ 
    \hline 
\end{tabularx}



\subsection{Requisiti utente opzionali}
\begin{table}[H]
    \begin{tabularx}{\textwidth}{|c|X|}
        \hline
        \textbf{ID} & \textbf{Descrizione} \\
        \hline
        \label{ru:RUF-1} RUF-1 &  L'utente dovrebbe poter gestire i risultati dei test(salvare, rinominare, visualizzare ed eliminare)\\
        \label{ru:RUF-2} RUF-2 & L'utente dovrebbe poter ricercare i test salvati per nome \\
        \label{ru:RUF-3} RUF-3 & L'utente dovrebbe poter visualizzare i risultati di un test salvato \\
        \label{ru:RUF-4} RUF-4 & L'utente dovrebbe poter salvare un dataset usando un file in formato strutturato \\
        \label{ru:RUF-5} RUF-5 & L'utente dovrebbe poter confrontare i risultati di due test archiviati \\
        \label{ru:RUF-6} RUF-6 & L'utente dovrebbe poter gestire i modelli da testare(salvare, modificare ed eliminare)\\
        \hline
    \end{tabularx}
    \vspace{10px}
    \caption{Requisiti utente opzionali}
\end{table}