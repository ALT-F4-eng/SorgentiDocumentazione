\subsection{Requisiti funzionali}
\label{subsec:requisiti_funzionali}
I requisiti funzionali sono raggruppati usando una divisione ad alto livello del sistema in due componenti, ovvero 
\begin{enumerate}
    \item Logica di presentazione: si occupa della presentazione delle informazioni agli utenti.
    \item Logica di business: si occupa dell'elaborazione delle informazioni mostrate agli utenti e ottenute dagli utenti.
\end{enumerate}

\subsubsection{Logica di presentazione}

\paragraph{Obbligatori}

\begin{freq}
    [
        \dependency{Se il sistema può gestire un insieme di LLM salvati quando l'utente richiede l'esecuzione di un test deve richiedere l'LLM da testare \hyperref[rf:RFF-1]{RFF-1}}
    ]
    {RFO-1}
    {Il sistema deve permettere la visualizzazione del dataset caricato dall'utente}
    \label{rf:RFO-1}%

    \subreq{RFO-1.01}{\hyperref[uc:UC-1]{UC-1}}{Il sistema deve visualizzare il nome del dataset caricato se non è temporaneo altrimenti indicare il fatto che è temporaneo}
    
    \subreq{RFO-1.02}{\hyperref[uc:UC-1]{UC-1}}{Se il dataset caricato è vuoto, il sistema deve visualizzare un messaggio che lo indica}

    \subreq{RFO-1.03}{\hyperref[uc:UC-1]{UC-1}}{Se l'utente ha già visualizzato il dataset a partire dal suo caricamento, il sistema deve visualizzare l'ultima pagina richiesta \hyperref[rf:RFO-2]{RFO-2}}
    
    \subreq{RFO-1.04}{\hyperref[uc:UC-26]{UC-26}}{Se l'utente visualizza per la prima volta il dataset a partire dal suo caricamento, il sistema deve visualizzare la prima pagina \hyperref[rf:RFO-2]{RFO-2}}
    
    \subreq{RFO-1.05}{\hyperref[uc:UC-1]{UC-1}}{Il sistema deve visualizzare un elemento che permetta la navigazione tra le pagine del dataset}
    
    \subreq{RFO-1.06}{\hyperref[uc:UC-4]{UC-4}}{Se il dataset caricato non è vuoto, il sistema deve visualizzare una barra di ricerca che fornisca all'utente le informazioni necessarie per il suo utilizzo}
    
    \subreq{RFO-1.07}{\hyperref[uc:UC-6]{UC-6}}{Il sistema deve visualizzare un pulsante che permetta l'aggiunta di un elemento al dataset caricato \hyperref[rf:RFO-4]{RFO-4}} 
    
    \subreq{RFO-1.08}{\hyperref[uc:UC-28]{UC-28}}{Se il dataset non è vuoto, il sistema deve visualizzare un pulsante che permetta l'esecuzione di un test} 
    
    \subreq{RFO-1.09}{\hyperref[uc:UC-29]{UC-29}}{Se l'utente cerca di eseguire un test su un dataset incompleto il sistema deve visualizzare una lista contenente i link agli elementi incompleti presenti nel dataset caricato \hyperref[rf:RFO-5]{RFO-5}}

    \subreq{RFO-1.10}{\hyperref[uc:UC-30]{UC-30}}{Se il sistema riscontra un errore durante l'esecuzione del test, deve visualizzare un messaggio di errore}
    
    \subreq{RFO-1.11}{\hyperref[uc:UC-9]{UC-9}}{Se il dataset caricato contiene modifiche non ancora salvate, il sistema deve visualizzare un pulsante che ne consenta il salvataggio}

    \subreq{RFO-1.12}{\hyperref[uc:UC-14]{UC-14}}{Se l'utente richiede il salvataggio di un dataset temporaneo, il sistema richiede il nome da assegnargli \hyperref[rf:RFO-6]{RFO-6}}

    \subreq{RFO-1.13}{\hyperref[uc:UC-13]{UC-13}}{Se il sistema riscontra un errore durante il salvataggio del dataset caricato, deve visualizzare un messaggio di errore}
\end{freq}

\begin{freq}
    {RFO-2}{Il sistema deve poter visualizzare una pagina del dataset caricato}
    \label{rf:RFO-2}%
    
    \subreq{RFO-2.01}{\hyperref[uc:UC-1]{UC-1}}{Il sistema deve visualizzare la pagina del dataset sotto forma di una lista scorrevole di elementi \hyperref[rf:RFO-3]{RFO-3}}

    \subreq{RFO-2.02}{\hyperref[uc:UC-2]{UC-2}}{Se il sistema verifica che è stata richiesta una pagina non valida visualizza un messaggio di errore}
        
    \subreq{RFO-2.03}{\hyperref[uc:UC-3]{UC-3}}{Se il sistema riscontra un errore durante il recupero degli elementi appartenenti alla pagina, deve visualizzare un messaggio di errore}

\end{freq}

\begin{freq}
    {RFO-3}
    {Il sistema deve poter visualizzare un elemento del dataset caricato}
    \label{rf:RFO-3}%
    
    \subreq{RFO-3.01}{\hyperref[uc:UC-1]{UC-1}}{Il sistema deve visualizzare la domanda contenuta nell'elemento}
    
    \subreq{RFO-3.02}{\hyperref[uc:UC-1]{UC-1}}{Il sistema deve visualizzare la risposta contenuta nell'elemento}
    
    \subreq{RFO-3.03}{\hyperref[uc:UC-1]{UC-1}}{Se domanda o risposta sono vuote il sistema deve indicarlo all'utente}
    
    \subreq{RFO-3.04}{\hyperref[uc:UC-7]{UC-7}}{Il sistema deve visualizzare un bottone che permette la modifica di un elemento \hyperref[rf:RFO-4]{RFO-4}}
    
    \subreq{RFO-3.05}{\hyperref[uc:UC-5]{UC-5}}{Il sistema deve visualizzare un bottone che permette l'eliminazione di un elemento}

    \subreq{RFO-3.06}{\hyperref[uc:UC-5]{UC-5}}{Se l'utente richiede l'eliminazione di un elemento il sistema deve richiedere la conferma}

\end{freq}

\begin{freq}
    {RFO-4}
    {Il sistema deve poter visualizzare gli elementi necessari per l'inserimento o la modifica di un elemento nel dataset caricato}
    \label{rf:RFO-4}%
    
    \subreq{RFO-4.01}{\hyperref[uc:UC-6]{UC-6}}{Il sistema deve visualizzare un elemento di input che permetta di specificare la domanda}
    
    \subreq{RFO-4.02}{\hyperref[uc:UC-6]{UC-6}}{Il sistema deve visualizzare un elemento di input che permetta di specificare la risposta attesa}
    
    \subreq{RFO-4.03}{\hyperref[uc:UC-6]{UC-6}}{Il sistema deve visualizzare un bottone che permetta la conferma}
    
    \subreq{RFO-4.04}{\hyperref[uc:UC-6]{UC-6}}{Il sistema deve visualizzare un bottone che permetta l'annullamento}
    
    \subreq{RFO-4.05}{\hyperref[uc:UC-8]{UC-8}}{Se si cerca di confermare un elemento non valido, il sistema deve visualizzare un messaggio di errore in cui viene indicato come compilare correttamente l'elemento}

\end{freq}

\begin{freq}
    {RFO-5}
    {Il sistema deve poter visualizzare la lista dei link agli elementi incompleti contenuti nel dataset caricato}
    \label{rf:RFO-5}%
    
    \subreq{RFO-5.01}{\hyperref[uc:UC-29]{UC-29}}{Il sistema deve visualizzare i link agli elementi incompleti presenti nel dataset usando una lista}
    
    \subreq{RFO-5.02}{\hyperref[uc:UC-29]{UC-29}}{Ogni link deve avere come ancora di destinazione la pagina del dataset caricato che contiene l'elemento incompleto}

    \subreq{RFO-5.03}{\hyperref[uc:UC-3]{UC-3}}{Se il sistema riscontra un errore interno durante l'ottenimento delle informazioni sugli elementi incompleti deve visualizzare un messaggio di errore}

\end{freq}

\begin{freq}
    {RFO-6}
    {Il sistema deve poter visualizzare gli elementi necessari per nominare e/o rinominare un dataset}
    \label{rf:RFO-6}%
    
    \subreq{RFO-6.01}{\hyperref[uc:UC-14]{UC-14}}{Il sistema deve visualizzare un elemento di input che permetta di specificare il nome}
    
    \subreq{RFO-6.02}{\hyperref[uc:UC-14]{UC-14}}{Il sistema deve visualizzare un bottone che permetta la conferma}
    
    \subreq{RFO-6.03}{\hyperref[uc:UC-14]{UC-14}}{Il sistema deve visualizzare un bottone che permetta l'annullamento}
    
    \subreq{RFO-6.04}{\hyperref[uc:UC-15]{UC-15}, \hyperref[uc:UC-16]{UC-16}}{Se si cerca di confermare un elemento non valido il sistema deve visualizzare un messaggio di errore}

\end{freq}

\begin{freq}
    [
        \dependency{Se il sistema può gestire la creazione di un nuovo dataset a partire da un file JSON deve mostrare un pulsante che permetta il caricamento del file \hyperref[rf:RFF-13]{RFF-13}}
    ]
    {RFO-7}
    {Il sistema deve permettere la visualizzazione dei dataset salvati}
    \label{rf:RFO-7}%
    
    \subreq{RFO-7.01}{\hyperref[uc:UC-19]{UC-19}}{Se non esistono dataset salvati il sistema deve visualizzare un messaggio che lo indica}
    
    \subreq{RFO-7.02}{\hyperref[uc:UC-25]{UC-25}, \hyperref[uc:UC-26]{UC-26}}{Il sistema deve visualizzare un bottone che permetta la  creazione e il caricamento di un nuovo dataset temporaneo vuoto}
    
    \subreq{RFO-7.03}{\hyperref[uc:UC-19]{UC-19}}{Il sistema deve visualizzare i dataset salvati sotto forma di una lista scorrevole \hyperref[rf:RFO-8]{RFO-8}}
        
    \subreq{RFO-7.04}{\hyperref[uc:UC-20]{UC-20}}{Il sistema, se esistono dataset salvati, deve visualizzare una barra di ricerca che specifichi all'utente le informazioni necessarie per l'esecuzione della ricerca stessa}

\end{freq}

\begin{freq}
    {RFO-8}
    {Il sistema deve poter visualizzare un dataset salvato}
    \label{rf:RFO-8}%
    
    \subreq{RFO-8.01}{\hyperref[uc:UC-19]{UC-19}}{Il sistema deve visualizzare il nome del dataset salvato}
    
    \subreq{RFO-8.02}{\hyperref[uc:UC-19]{UC-19}}{Il sistema deve visualizzare la data dell'ultima modifica}
    
    \subreq{RFO-8.03}{\hyperref[uc:UC-24]{UC-24}}{Il sistema deve visualizzare un bottone che permetta di rinominare il dataset salvato \hyperref[rf:RFO-6]{RFO-6}}
    
    \subreq{RFO-8.04}{\hyperref[uc:UC-21]{UC-21}}{Il sistema deve visualizzare un bottone che permetta di copiare il dataset salvato}
    
    \subreq{RFO-8.05}{\hyperref[uc:UC-22]{UC-22}}{Il sistema deve visualizzare un bottone che permetta di eliminare il dataset salvato}

    \subreq{RFO-8.06}{\hyperref[uc:UC-22]{UC-22}}{Se l'utente richiede l'eliminazione di un dataset salvato, il sistema deve richiedere la conferma}
    
    \subreq{RFO-8.07}{\hyperref[uc:UC-26]{UC-26}}{Il sistema deve visualizzare un bottone che permetta di caricare e visualizzare il dataset salvato \hyperref[rf:RFO-1]{RFO-1}}

    \subreq{RFO-8.08}{\hyperref[uc:UC-27]{UC-27}}{Se viene richiesto il caricamento di un dataset salvato quando il dataset attualmente caricato contiene modifiche non salvate, il sistema deve richiedere conferma della sovrascrittura}

    \subreq{RFO-8.09}{\hyperref[uc:UC-3]{UC-3}}{Se il sistema riscontra un errore interno durante l'ottenimento delle informazioni del dataset salvato, il sistema deve visualizzare un messaggio di errore}

 \end{freq}

\begin{freq}
    [
        \dependency{Se il sistema può gestire un insieme di test salvati, deve indicare il nome del test caricato, se non è temporaneo, altrimenti deve indicare il fatto che è temporaneo}
        \dependency{Se il sistema può gestire un insieme di LLM salvati deve visualizzare il nome dell'LLM testato}
        \dependency{Se il sistema può gestire il confronto tra test, deve visualizzare un pulsante che permetta il confronto del test caricato con un test salvato \hyperref[rf:RFF-9]{RFF-9}}
    ]
    {RFO-9}
    {Il sistema deve permettere la visualizzazione del test caricato}
    \label{rf:RFO-9}%
    
    \subreq{RFO-9.01}{\hyperref[uc:UC-31]{UC-31}}{Il sistema deve indicare la data di esecuzione del test}

    \subreq{RFO-9.02}{\hyperref[uc:UC-31]{UC-31}}{Il sistema deve visualizzare l'indice riassuntivo \hyperref[rf:RFO-10]{RFO-10}}
    
    \subreq{RFO-9.03}{\hyperref[uc:UC-32]{UC-32}}{Se l'utente ha già visualizzato il test a partire dal suo caricamento, il sistema deve visualizzare l'ultima pagina richiesta \hyperref[rf:RFO-11]{RFO-11}}
    
    \subreq{RFO-9.04}{\hyperref[uc:UC-32]{UC-32}}{Se l'utente visualizza per la prima volta il test a partire dal suo caricamento il sistema deve visualizzare la prima pagina \hyperref[rf:RFO-11]{RFO-11}}
    
    \subreq{RFO-9.05}{\hyperref[uc:UC-33]{UC-33}}{Il sistema deve visualizzare un elemento che permetta la navigazione tra le pagine di risultati}

    \subreq{RFO-9.06}{\hyperref[uc:UC-31]{UC-31}}{Il sistema deve visualizzare il nome del dataset utilizzato nel test se non è temporaneo}
    
 \end{freq}

 \begin{freq}
    {RFO-10}
    {Il sistema deve poter visualizzare l'indice riassuntivo di un test}
    \label{rf:RFO-10}%
    
    \subreq{RFO-10.01}{\hyperref[uc:UC-31]{UC-31}}{Il sistema deve visualizzare il numero di domande che compongono il dataset usato nel test}
    
    \subreq{RFO-10.02}{\hyperref[uc:UC-31]{UC-31}}{Il sistema deve visualizzare la media dei gradi di similarità}

    \subreq{RFO-10.03}{\hyperref[uc:UC-31]{UC-31}}{Il sistema deve visualizzare la deviazione standard dei gradi di similarità}
    
    \subreq{RFO-10.04}{\hyperref[uc:UC-31]{UC-31}}{Il sistema deve visualizzare un grafico a torta che indica la percentuale di risposte corrette ed errate}
    
    \subreq{RFO-10.05}{\hyperref[uc:UC-31]{UC-31}}{Il sistema deve visualizzare un grafico a barre delle frequenze relative dei risultati rispetto ai seguenti cinque range di valori di similarità: $[0, 0.2], [0.2, 0.4], [0.4, 0.6], [0.6, 0.8], [0.8, 1]$}
    
    \subreq{RFO-10.06}{\hyperref[uc:UC-3]{UC-3}}{Se il sistema riscontra un errore nell'ottenimento delle statistiche, deve mostrare un messaggio di errore}

 \end{freq}

 \begin{freq}
    {RFO-11}
    {Il sistema deve poter visualizzare una pagina del test caricato}
    \label{rf:RFO-11}%

    \subreq{RFO-11.01}{\hyperref[uc:UC-32]{UC-32}}{Il sistema deve visualizzare un diagramma di dispersione riguardante i risultati della pagina \hyperref[rf:RFO-13]{RFO-13}}
    
    \subreq{RFO-11.02}{\hyperref[uc:UC-32]{UC-32}}{Il sistema deve visualizzare i risultati appartenenti alla pagina come una lista scorrevole \hyperref[rf:RFO-12]{RFO-12}}

    \subreq{RFO-11.03}{\hyperref[uc:UC-2]{UC-2}}{Se il sistema verifica che è stata richiesta una pagina non valida visualizza un messaggio di errore}
        
    \subreq{RFO-11.04}{\hyperref[uc:UC-3]{UC-3}}{Se il sistema riscontra un errore durante l'ottenimento degli elementi appartenenti alla pagina, deve visualizzare un messaggio di errore}

 \end{freq}

 \begin{freq}
    {RFO-12}
    {Il sistema deve poter visualizzare un singolo risultato di un test}
    \label{rf:RFO-12}%
    
    \subreq{RFO-12.01}{\hyperref[uc:UC-32]{UC-32}}{Il sistema deve visualizzare la domanda}
    
    \subreq{RFO-12.02}{\hyperref[uc:UC-32]{UC-32}}{Il sistema deve visualizzare la risposta attesa}
    
    \subreq{RFO-12.03}{\hyperref[uc:UC-32]{UC-32}}{Il sistema deve visualizzare la risposta ottenuta}
    
    \subreq{RFO-12.04}{\hyperref[uc:UC-32]{UC-32}}{Il sistema deve visualizzare il grado di similarità tra la risposta attesa e quella ottenuta}
    
    \subreq{RFO-12.05}{\hyperref[uc:UC-32]{UC-32}}{Il sistema deve visualizzare il responso sulla correttezza della risposta ottenuta}

 \end{freq}

 \begin{freq}
    {RFO-13}
    {Il sistema deve poter visualizzare un diagramma di dispersione contenente un insieme di risultati di un test}
   \label{rf:RFO-13}%
    
    \subreq{RFO-13.01}{}{L'asse delle ascisse rappresenta il numero di domanda}
    
    \subreq{RFO-13.02}{}{L'asse delle ordinate rappresenta il grado di similarità tra la risposta attesa e la risposta ottenuta}
    
    \subreq{RFO-13.03}{}{I singoli risultati vengono raffigurati come punti nel grafico cliccabili che riportano al risultato rappresentato}
    
    \subreq{RFO-13.04}{}{I punti che rappresentano risposte ritenute corrette devono essere distinguibili da quelli che rappresentano risposte ritenute errate}
    
    \subreq{RFO-13.05}{}{Il sistema deve rappresentare la media dei valori di similarità come una linea retta}
 
\end{freq}

 

\paragraph{Facoltativi}

\begin{freq}
    {RFF-1}
    {Il sistema dovrebbe poter visualizzare la lista degli LLM salvati per permetterne la selezione in fase di test}
    \label{rf:RFF-1}%
    
    \subreq{RFF-1.01}{\hyperref[uc:UC-52]{UC-52}}{Il sistema deve visualizzare gli LLM salvati attraverso una lista scorrevole}
    
    \subreq{RFF-1.02}{\hyperref[uc:UC-52]{UC-52}, \hyperref[uc:UC-53]{UC-53}}{Ogni LLM salvato deve essere mostrato indicandone il nome e la data di ultima modifica}
    
    \subreq{RFF-1.03}{\hyperref[uc:UC-53]{UC-53}}{Il sistema deve permettere la selezione di uno degli LLM salvati}

    \subreq{RFF-1.04}{\hyperref[uc:UC-3]{UC-3}, \hyperref[uc:UC-53]{UC-53}}{Se il sistema riscontra un errore interno durante l'ottenimento delle informazioni sugli LLM salvati deve visualizzare un messaggio di errore}

\end{freq}

\begin{freq}
    {RFF-2}
    {Il sistema dovrebbe poter visualizzare la lista degli LLM salvati}
    \label{rf:RFF-2}%
    
    \subreq{RFF-2.01}{\hyperref[uc:UC-52]{UC-52}, \hyperref[uc:UC-53]{UC-53}}{Il sistema deve visualizzare gli LLM salvati attraverso una lista scorrevole indicando per ognuno il nome e la data di ultima modifica}
    
    \subreq{RFF-2.02}{\hyperref[uc:UC-53]{UC-53}}{Per ogni LLM salvato deve visualizzare un bottone per permetterne la visualizzazione \hyperref[rf:RFF-3]{RFF-3}}
    
    \subreq{RFF-2.03}{\hyperref[uc:UC-54]{UC-54}}{Per ogni LLM salvato il sistema deve visualizzare un bottone per l'eliminazione}

    \subreq{RFF-2.04}{\hyperref[uc:UC-40]{UC-40}, \hyperref[uc:UC-54]{UC-54}, \hyperref[uc:UC-55]{UC-55}}{Se l'utente richiede l'eliminazione di un LLM associato a uno o più test salvati, il sistema deve notificare che tali test verranno eliminati}

    \subreq{RFF-2.05}{\hyperref[uc:UC-54]{UC-54}}{Se l'utente richiede l'eliminazione di un LLM il sistema deve richiedere la conferma}
    
    \subreq{RFF-2.06}{\hyperref[uc:UC-46]{UC-46}}{Il sistema deve visualizzare un bottone per l'aggiunta di un nuovo LLM \hyperref[rf:RFF-5]{RFF-5}}

    \subreq{RFF-2.07}{\hyperref[uc:UC-3]{UC-3}, \hyperref[uc:UC-53]{UC-53}}{Se il sistema riscontra un errore interno durante l'ottenimento delle informazioni sugli LLM salvati deve visualizzare un messaggio di errore}

\end{freq}

\begin{freq}
    {RFF-3}
    {Il sistema dovrebbe poter visualizzare un LLM salvato}
    \label{rf:RFF-3}%
    
    \subreq{RFF-3.01}{\hyperref[uc:UC-53]{UC-53}}{Il sistema deve visualizzare il nome dell'LLM}
    
    \subreq{RFF-3.02}{\hyperref[uc:UC-53]{UC-53}}{Il sistema deve visualizzare la data dell'ultima modifica}
    
    \subreq{RFF-3.03}{\hyperref[uc:UC-47]{UC-47}, \hyperref[uc:UC-53]{UC-53}}{Il sistema deve visualizzare l'URL usato per l'esecuzione delle chiamate HTTP all'API dell'LLM}
    
    \subreq{RFF-3.04}{\hyperref[uc:UC-47]{UC-47}}{Il sistema deve visualizzare le coppie chiave-valore per la costruzione dell'header delle chiamate}
    
    \subreq{RFF-3.05}{\hyperref[uc:UC-47]{UC-47}}{Il sistema deve visualizzare le coppie chiave-valore per la costruzione del body delle chiamate}
    
    \subreq{RFF-3.06}{\hyperref[uc:UC-47]{UC-47}}{Il sistema deve visualizzare la chiave da utilizzare per specificare la domanda da porre all'LLM nel body delle richieste HTTP}
    
    \subreq{RFF-3.07}{\hyperref[uc:UC-47]{UC-47}}{Il sistema deve visualizzare la chiave da utilizzare per estrarre la risposta data dall'LLM contenuta nelle risposte HTTP}
    
    \subreq{RFF-3.08}{\hyperref[uc:UC-50]{UC-50}, \hyperref[uc:UC-51]{UC-51}, \hyperref[uc:UC-53]{UC-53}}{Il sistema deve visualizzare un bottone per la modifica dell'LLM visualizzato \hyperref[rf:RFF-4]{RFF-4}}

\end{freq}

\begin{freq}
    {RFF-4}
    {Il sistema dovrebbe poter visualizzare gli elementi necessari alla modifica di un LLM}
    \label{rf:RFF-4}%
    
    \subreq{RFF-4.01}{\hyperref[uc:UC-50]{UC-50}}{Il sistema deve visualizzare un elemento di input che permetta la modifica del nome}
    
    \subreq{RFF-4.02}{\hyperref[uc:UC-51]{UC-51}}{Il sistema deve visualizzare un elemento di input che permetta la modifica dell'URL}
    
    \subreq{RFF-4.03}{\hyperref[uc:UC-7]{UC-7}}{Il sistema deve visualizzare un elemento di input che permetta la modifica della chiave associata alla domanda da porre all'LLM}
    
    \subreq{RFF-4.04}{\hyperref[uc:UC-7]{UC-7}}{Il sistema deve visualizzare un elemento di input che permetta la modifica della chiave associata alla risposta da ottenere dall'LLM}
    
    \subreq{RFF-4.05}{\hyperref[uc:UC-51]{UC-51}, \hyperref[uc:UC-7]{UC-7}}{Il sistema deve visualizzare due elementi di input che permettano la modifica di ogni coppia chiave-valore associata alla creazione dell'header delle richieste HTTP verso l'LLM}
    
    \subreq{RFF-4.06}{\hyperref[uc:UC-51]{UC-51}, \hyperref[uc:UC-6]{UC-6}}{Il sistema deve visualizzare un bottone che permetta l'inserimento di una coppia chiave-valore associata alla creazione dell'header delle richieste HTTP verso l'LLM}
    
    \subreq{RFF-4.07}{\hyperref[uc:UC-51]{UC-51}, \hyperref[uc:UC-7]{UC-7}}{Il sistema deve visualizzare due elementi di input che permettano la modifica di ogni coppia chiave-valore associata alla creazione del body delle richieste HTTP verso l'LLM}
    
    \subreq{RFF-4.08}{\hyperref[uc:UC-51]{UC-51}, \hyperref[uc:UC-6]{UC-6}}{Il sistema deve visualizzare un bottone che permetta l'inserimento di una coppia chiave-valore associata alla creazione del body delle richieste HTTP verso l'LLM}

    \subreq{RFF-4.09}{\hyperref[uc:UC-50]{UC-50}, \hyperref[uc:UC-51]{UC-51}}{Il sistema deve visualizzare un bottone che permetta di salvare le modifiche sull'LLM}

    \subreq{RFF-4.10}{\hyperref[uc:UC-50]{UC-50}, \hyperref[uc:UC-51]{UC-51}}{Il sistema deve visualizzare un bottone che permetta di annullare le modifiche sull'LLM}

    \subreq{RFF-4.11}{\hyperref[uc:UC-48]{UC-48}}{Se si cerca di salvare un URL con formato non valido il sistema deve mostrare un messaggio di errore}

    \subreq{RFF-4.12}{\hyperref[uc:UC-49]{UC-49}}{Se si cerca di salvare una coppia chiave-valore non valida, il sistema deve mostrare un messaggio di errore}

    \subreq{RFF-4.13}{\hyperref[uc:UC-13]{UC-13}}{Se avviene un errore durante la modifica dell'LLM il sistema deve visualizzare un messaggio di errore}

\end{freq}

\begin{freq}
    {RFF-5}
    {Il sistema dovrebbe poter visualizzare gli elementi necessari per la creazione di un nuovo LLM}
    \label{rf:RFF-5}%
    
    \subreq{RFF-5.01}{\hyperref[uc:UC-50]{UC-50}}{Il sistema deve visualizzare un elemento di input che permetta di specificare il nome}
    
    \subreq{RFF-5.02}{\hyperref[uc:UC-51]{UC-51}}{Il sistema deve visualizzare un elemento di input che permetta di specificare l'URL}
    
    \subreq{RFF-5.03}{\hyperref[uc:UC-6]{UC-6}}{Il sistema deve visualizzare un elemento di input che permetta di specificare la chiave associata alla domanda da porre all'LLM}
    
    \subreq{RFF-5.04}{\hyperref[uc:UC-6]{UC-6}}{Il sistema deve visualizzare un elemento di input che permetta di specificare la chiave associata alla risposta da ottenere dall'LLM}
    
    \subreq{RFF-5.05}{\hyperref[uc:UC-6]{UC-6}}{Il sistema deve visualizzare un bottone che permetta l'inserimento di una coppia chiave-valore associata alla creazione dell'header delle richieste HTTP verso l'LLM}
    
    \subreq{RFF-5.06}{\hyperref[uc:UC-6]{UC-6}}{Il sistema deve visualizzare un bottone che permetta l'inserimento di una coppia chiave-valore associata alla creazione del body delle richieste HTTP verso l'LLM}

    \subreq{RFF-5.07}{\hyperref[uc:UC-46]{UC-46}}{Il sistema deve visualizzare un bottone che permetta di salvare il nuovo LLM}

    \subreq{RFF-5.08}{\hyperref[uc:UC-46]{UC-46}}{Il sistema deve visualizzare un bottone che permetta di annullare la creazione del nuovo LLM}

    \subreq{RFF-5.09}{\hyperref[uc:UC-48]{UC-48}}{Se si cerca di salvare un URL con formato non valido il sistema deve mostrare un messaggio di errore}

    \subreq{RFF-5.10}{\hyperref[uc:UC-49]{UC-49}}{Se si cerca di salvare una coppia chiave-valore non valida, il sistema deve mostrare un messaggio di errore}

    \subreq{RFF-5.11}{\hyperref[uc:UC-13]{UC-13}}{Se avviene un errore durante la creazione dell'LLM il sistema deve visualizzare un messaggio di errore}

\end{freq}

\begin{freq}
    {RFF-6}
    {Il sistema dovrebbe poter visualizzare i test salvati}
    \label{rf:RFF-6}%
    
    \subreq{RFF-6.01}{\hyperref[uc:UC-36]{UC-36}}{Il sistema deve visualizzare i test salvati sotto forma di una lista \hyperref[rf:RFF-7]{RFF-7}}

    \subreq{RFF-6.02}{\hyperref[uc:UC-36]{UC-36}}{Se non esistono test salvati il sistema deve visualizzare un messaggio che lo indica}

    \subreq{RFF-6.03}{\hyperref[uc:UC-37]{UC-37}}{Il sistema, se esistono test salvati, deve visualizzare una barra di ricerca che specifichi all'utente le informazioni necessarie per l'esecuzione della ricerca stessa}

\end{freq}

\begin{freq}
    [
        \dependency{Se il sistema può gestire il salvataggio degli LLM il sistema deve mostrare il nome dell'LLM utilizzato nel test}
    ]
    {RFF-7}
    {Il sistema dovrebbe poter visualizzare un test salvato}
    \label{rf:RFF-7}%
    
    \subreq{RFF-7.01}{\hyperref[uc:UC-36]{UC-36}}{Il sistema deve visualizzare il nome del test salvato}
    
    \subreq{RFF-7.02}{\hyperref[uc:UC-31]{UC-31}}{Il sistema deve visualizzare il nome del dataset su cui è stato eseguito il test}
    
    \subreq{RFF-7.03}{}{Il sistema deve visualizzare la data in cui il test è stato eseguito}
    
    \subreq{RFF-7.04}{}{Il sistema deve visualizzare un bottone che permetta l'eliminazione del test salvato}

    \subreq{RFF-7.05}{}{Se l'utente richiede l'eliminazione di un test salvato il sistema deve richiedere la conferma}
    
    \subreq{RFF-7.06}{}{Il sistema deve visualizzare un bottone che permetta di rinominare il test salvato \hyperref[rf:RFF-8]{RFF-8}}

    \subreq{RFF-7.07}{}{Il sistema deve visualizzare un bottone che permetta il suo caricamento e visualizzazione \hyperref[rf:RFO-9]{RFO-9}}

    \subreq{RFF-7.08}{}{Se viene richiesto il caricamento di un test e il test attualmente caricato non è stato salvato il sistema deve chiedere la conferma per la sovrascrittura}

\end{freq}

\begin{freq}
    {RFF-8}
    {Il sistema dovrebbe poter visualizzare gli elementi necessari ad assegnare un nome ad un test salvato}
    \label{rf:RFF-8}%
    
    \subreq{RFF-8.01}{\hyperref[uc:UC-41]{UC-41}}{Il sistema deve visualizzare un elemento di input che permetta la specifica del nome}
    
    \subreq{RFF-8.02}{\hyperref[uc:UC-41]{UC-41}}{Il sistema deve visualizzare un bottone per la conferma della denominazione}
    
    \subreq{RFF-8.03}{\hyperref[uc:UC-41]{UC-41}}{Il sistema deve visualizzare un bottone per l'annullamento della denominazione}

    \subreq{RFF-8.04}{\hyperref[uc:UC-15]{UC-15}, \hyperref[uc:UC-16]{UC-16}}{Se si cerca di utilizzare un nome non valido il sistema deve visualizzare un messaggio di errore}

\end{freq}

\begin{freq}
    [
        \dependency{Il sistema deve poter gestire i test salvati}
    ]
    {RFF-9}
    {Il sistema dovrebbe poter permettere di confrontare il test caricato con un test salvato}
    \label{rf:RFF-9}%
    
    \subreq{RFF-9.01}{\hyperref[uc:UC-42]{UC-42}}{Il sistema deve visualizzare, sotto forma di lista, i test salvati che possono essere confrontati con il test caricato, indicando il nome del dataset di test, la data di esecuzione e il nome dell'LLM utilizzato}

    \subreq{RFF-9.02}{\hyperref[uc:UC-42]{UC-42}}{Il sistema deve permettere la selezione di un test salvato da confrontare con il test caricato}

\end{freq}

\begin{freq}
    {RFF-10}
    {Il sistema dovrebbe poter visualizzare il confronto tra due test salvati}
    \label{rf:RFF-10}%
    
    \subreq{RFF-10.01}{\hyperref[uc:UC-42]{UC-42}}{Il sistema deve visualizzare l'indice riassuntivo per il primo test \hyperref[rf:RFO-10]{RFO-10}}
    
    \subreq{RFF-10.02}{\hyperref[uc:UC-42]{UC-42}}{Il sistema deve visualizzare l'indice riassuntivo per il secondo test \hyperref[rf:RFO-10]{RFO-10}}
    
    \subreq{RFF-10.03}{\hyperref[uc:UC-43]{UC-43}\hyperref[uc:UC-44]{UC-44}}{Se i test sono stati eseguiti sulla stessa versione dello stesso dataset e l'utente ha già visualizzato il confronto dei singoli risultati il sistema deve visualizzare l'ultima pagina richiesta \hyperref[rf:RFF-11]{RFF-11}}
    
    \subreq{RFF-10.04}{\hyperref[uc:UC-43]{UC-43},\hyperref[uc:UC-44]{UC-44}}{Se i test sono stati eseguiti sulla stessa versione dello stesso dataset e l'utente visualizza per la prima volta il confronto dei singoli risultati il sistema deve visualizzare la prima pagina \hyperref[rf:RFF-11]{RFF-11}}

    \subreq{RFF-10.05}{\hyperref[uc:UC-43]{UC-43},\hyperref[uc:UC-45]{UC-45}}{Se i test sono stati eseguiti sulla stessa versione dello stesso dataset il sistema deve visualizzare l'elemento di navigazione tra le pagine del confronto}

\end{freq}

\begin{freq}
    {RFF-11}
    {Il sistema dovrebbe poter visualizzare una pagina del confronto tra i singoli risultati di due test}
    \label{rf:RFF-11}%
    
    \subreq{RFF-11.01}{}{Il sistema deve visualizzare una pagina del confronto tra i singoli risultati di due test sotto forma di una lista scorrevole di elementi \hyperref[rf:RFF-12]{RFF-12}}

    \subreq{RFF-11.02}{}{Il sistema deve visualizzare un diagramma di dispersione che rappresenti i singoli risultati confrontati nella pagina \hyperref[rf:RFO-13]{RFO-13}}

    \subreq{RFF-11.03}{\hyperref[uc:UC-2]{UC-2}}{Se il sistema verifica che è stata richiesta una pagina non valida visualizza un messaggio di errore}
        
    \subreq{RFF-11.04}{\hyperref[uc:UC-3]{UC-3}}{Se il sistema riscontra un errore durante l'ottenimento degli elementi appartenenti alla pagina, deve visualizzare un messaggio di errore}

\end{freq}

\begin{freq}
    {RFF-12}
    {Il sistema dovrebbe poter visualizzare un confronto tra due risultati}
    \label{rf:RFF-12}%
    
    \subreq{RFF-12.01}{}{Il sistema deve visualizzare la domanda, la risposta ottenuta e la risposta attesa del primo risultato}

    \subreq{RFF-12.02}{}{Il sistema deve visualizzare la domanda, la risposta ottenuta e la risposta attesa del secondo risultato}

    \subreq{RFF-12.03}{}{Il sistema deve visualizzare il confronto tra i valori di similarità dei due risultati}

    \subreq{RFF-12.04}{}{Il sistema deve visualizzare il confronto tra la correttezza dei due risultati}

\end{freq}

\begin{freq}
    {RFF-13}
    {Il sistema dovrebbe poter visualizzare gli elementi necessari al caricamento di un nuovo dataset da un file JSON}
    \label{rf:RFF-13}%
    
    \subreq{RFF-13.01}{}{Il sistema deve visualizzare un elemento di input per la selezione di un file JSON dal filesystem}

    \subreq{RFF-13.02}{}{Il sistema deve visualizzare gli elementi necessari per la denominazione del nuovo dataset \hyperref[rf:RFO-6]{RFO-6}}

    \subreq{RFF-13.03}{}{Se il file JSON non rispetta il formato supportato, il sistema deve generare un messaggio di errore}

    \subreq{RFF-13.04}{}{Se avviene un errore durante il salvataggio del nuovo dataset il sistema deve visualizzare un messaggio di errore}

\end{freq}

\subsubsection{Logica di business}

\paragraph{Obbligatori}
\begin{freq}
{RFO-14}{Il sistema deve poter gestire il dataset caricato}{}
    \label{rf:RFO-14}%
    
    \subreq{RFO-14.01}{}{Il sistema deve poter determinare se il dataset caricato è vuoto}
    
    \subreq{RFO-14.02}{}{Il sistema deve poter determinare se il dataset caricato è temporaneo}
    
    \subreq{RFO-14.03}{}{Il sistema deve poter determinare se il dataset caricato è incompleto e quindi non può essere usato in un test \hyperref[rf:RFO-15]{RFO-15}}
    
    \subreq{RFO-14.04}{}{Il sistema deve poter determinare se il dataset caricato contiene modifiche non salvate}

    \subreq{RFO-14.05}{}{Il sistema deve poter inserire un nuovo elemento nel dataset caricato \hyperref[rf:RFO-15]{RFO-15}}

    \subreq{RFO-14.06}{}{Il sistema deve ottenere gli elementi incompleti contenuti nel dataset caricato \hyperref[rf:RFO-15]{RFO-15}}
    
    \subreq{RFO-14.07}{}{Il sistema deve poter eliminare un elemento contenuto nel dataset caricato}

    \subreq{RFO-14.08}{}{Il sistema deve poter modificare un elemento contenuto nel dataset caricato \hyperref[rf:RFO-15]{RFO-15}}

    \subreq{RFO-14.09}{}{Il sistema deve poter calcolare il numero della pagina massima del dataset caricato}

    \subreq{RFO-14.10}{}{Il sistema data una pagina deve poter determinare se appartiene al range di pagine valide per il dataset caricato}  

    \subreq{RFO-14.11}{}{Il sistema deve poter risalire all'ultima pagina richiesta del dataset caricato}    

    \subreq{RFO-14.12}{}{Il sistema deve poter identificare e recuperare gli elementi del dataset caricato che appartengono a una specifica pagina}

    \subreq{RFO-14.13}{}{Il sistema deve poter determinare la pagina a cui un dato elemento appartiene}

    \subreq{RFO-14.14}{}{Il sistema deve poter selezionare un sottoinsieme degli elementi del dataset caricato che contengono una data stringa nella propria domanda e/o risposta}
\end{freq}

\begin{freq}
{RFO-15}{Il sistema deve poter gestire un elemento del dataset caricato}
    \label{rf:RFO-15}%
    
    \subreq{RFO-15.01}{\hyperref[uc:UC-8]{UC-8}}{Il sistema deve poter verificare che l'elemento sia valido ovvero non contenga domanda e risposta entrambe vuote o composte da soli spazi}
    
    \subreq{RFO-15.02}{\hyperref[uc:UC-8]{UC-8}}{Il sistema deve poter verificare che un elemento sia completo ovvero contenga domanda e risposta entrambe non vuote e non composte da soli spazi}

    \subreq{RFO-15.03}{\hyperref[uc:UC-1]{UC-1}}{Il sistema deve poter ottenere la domanda di un elemento}

    \subreq{RFO-15.04}{\hyperref[uc:UC-1]{UC-1}}{Il sistema deve poter ottenere la risposta di un elemento}
    
    \subreq{RFO-15.05}{\hyperref[uc:UC-7]{UC-7}}{Il sistema deve poter modificare la risposta di un elemento}

    \subreq{RFO-15.06}{\hyperref[uc:UC-7]{UC-7}}{Il sistema deve poter modificare la domanda di un elemento}

\end{freq}

\begin{freq}
[
    \dependency{Il sistema dovrebbe poter salvare un nuovo dataset a partire da un file il formato JSON}
]
{RFO-16}{Il sistema deve poter gestire l'insieme di dataset salvati}
    \label{rf:RFO-16}%

    \subreq{RFO-16.01}{\hyperref[uc:UC-9]{UC-9}, \hyperref[uc:UC-10]{UC-10}}{Il sistema deve poter salvare un nuovo dataset}

    \subreq{RFO-16.02}{\hyperref[uc:UC-12]{UC-12}}{Il sistema deve poter aggiornare un dataset salvato senza perdere le sue versioni precedenti}

    \subreq{RFO-16.03}{\hyperref[uc:UC-22]{UC-22}}{Il sistema deve poter eliminare un dataset salvato}

    \subreq{RFO-16.04}{\hyperref[uc:UC-22]{UC-22}, \hyperref[uc:UC-24]{UC-24}}{Il sistema deve poter modificare un dataset salvato \hyperref[rf:RFO-17]{RFO-17}}

    \subreq{RFO-16.05}{\hyperref[uc:UC-19]{UC-19}}{Il sistema deve poter ottenere i dataset salvati}

    \subreq{RFO-16.06}{\hyperref[uc:UC-19]{UC-19}}{Il sistema deve poter determinare se esistono dataset salvati}

    \subreq{RFO-16.07}{\hyperref[uc:UC-20]{UC-20}}{Il sistema deve poter selezionare un sottoinsieme dei dataset salvati che contengono una data stringa nel proprio nome}

\end{freq}

\begin{freq}
    {RFO-17}{Il sistema deve poter gestire un dataset salvato}
    \label{rf:RFO-17}%
    
    \subreq{RFO-17.01}{\hyperref[uc:UC-1]{UC-1}}{Il sistema deve poter ottenere il nome del dataset}
    
    \subreq{RFO-17.02}{\hyperref[uc:UC-1]{UC-1}}{Il sistema deve poter determinare la data dell'ultimo aggiornamento del dataset}

    \subreq{RFO-17.03}{\hyperref[uc:UC-24]{UC-24}}{Il sistema deve poter rinominare un dataset}

    \subreq{RFO-17.04}{\hyperref[uc:UC-15]{UC-15}, \hyperref[uc:UC-16]{UC-16}}{Il sistema deve poter determinare se un nome di dataset è valido, ovvero se non è già associato a un altro dataset e non è vuoto o composto da soli spazi}
    
    \subreq{RFO-17.05}{\hyperref[uc:UC-27]{UC-27}}{Il sistema deve poter caricare un dataset salvato eventualmente sovrascrivendo il dataset attualmente caricato}

\end{freq}

\begin{freq}
    [
        \dependency{Il sistema dovrebbe poter determinare l'LLM utilizzato nel test caricato}
        \dependency{Il sistema deve poter determinare se il test è stato salvato}
    ]
    {RFO-18}{Il sistema deve poter gestire il test caricato}
    \label{rf:RFO-18}%
    
    \subreq{RFO-18.01}{\hyperref[uc:UC-31]{UC-31}}{Il sistema deve poter determinare la versione del dataset su cui è stato eseguito il test caricato}

    \subreq{RFO-18.02}{\hyperref[uc:UC-31]{UC-31}}{Il sistema deve poter determinare il nome del dataset su cui è stato eseguito il test caricato}

    \subreq{RFO-18.03}{\hyperref[uc:UC-32]{UC-32}}{Il sistema deve poter calcolare il numero della pagina massima dei risultati del test caricato}

    \subreq{RFO-18.04}{\hyperref[uc:UC-2]{UC-2}}{Il sistema, data una pagina, deve poter determinare se appartiene al range di pagine valide dei risultati del test caricato}  

    \subreq{RFO-18.05}{\hyperref[uc:UC-32]{UC-32}}{Il sistema deve poter risalire all'ultima pagina di risultati richiesta}    

    \subreq{RFO-18.06}{\hyperref[uc:UC-32]{UC-32}}{Il sistema deve poter determinare e ottenere i singoli risultati del test caricato che appartengono a una data pagina}

    \subreq{RFO-18.07}{\hyperref[uc:UC-31]{UC-31}}{Il sistema deve poter ottenere le statistiche del test caricato \hyperref[rf:RFO-19]{RFO-19}}    
\end{freq}

\begin{freq}
    {RFO-19}{Il sistema deve poter eseguire il test sul dataset caricato}
    \label{rf:RFO-19}%
    
    \subreq{RFO-19.01}{\hyperref[uc:UC-28]{UC-28}}{Dato un LLM da testare il sistema deve poterci comunicare usando le informazioni a esso associate per ottenere le risposte alle domande contenute nel dataset caricato}
    
    \subreq{RFO-19.02}{\hyperref[uc:UC-28]{UC-28}}{Il sistema, per ogni risposta ottenuta dall'LLM da testare, deve calcolare il grado similarità con la risposta attesa e rappresentarlo come un valore nell'intervallo $[0,1]$}
    
    \subreq{RFO-19.03}{\hyperref[uc:UC-28]{UC-28}}{Il sistema, per ogni risposta ottenuta, deve richiedere all'oracolo la correttezza della stessa}
    
    \subreq{RFO-19.04}{\hyperref[uc:UC-28]{UC-28}}{Il sistema deve calcolare la media dei gradi di similarità dei risultati}
    
    \subreq{RFO-19.05}{\hyperref[uc:UC-28]{UC-28}}{Il sistema deve calcolare la deviazione standard dei gradi di similarità dei risultati}
    
    \subreq{RFO-19.06}{\hyperref[uc:UC-28]{UC-28}}{Il sistema deve calcolare il numero di risultati che appartiene a ognuno dei seguenti range di similarità: 
    
    $[0, 0.2], [0.2, 0.4], [0.4, 0.6], [0.6, 0.8], [0.8, 1]$}

    \subreq{RFO-19.07}{\hyperref[uc:UC-28]{UC-28}}{Il sistema deve calcolare la percentuale di risposte corrette rispetto al totale}

\end{freq}

\paragraph{Facoltativi}

\begin{freq}
    {RFF-14}{Il sistema dovrebbe poter gestire un insieme di test salvati}
    \label{rf:RFF-14}%
    
    \subreq{RFF-14.01}{\hyperref[uc:UC-35]{UC-35}}{Il sistema deve poter salvare un test}
    
    \subreq{RFF-14.02}{\hyperref[uc:UC-36]{UC-36}}{Il sistema deve poter ottenere i test salvati}
    
    \subreq{RFF-14.03}{\hyperref[uc:UC-40]{UC-40}}{Il sistema deve poter eliminare un test salvato}

    \subreq{RFF-14.04}{\hyperref[uc:UC-40]{UC-40}, \hyperref[uc:UC-41]{UC-41}}{Il sistema deve poter modificare un test salvato \hyperref[rf:RFF-15]{RFF-15}}

    \subreq{RFF-14.05}{\hyperref[uc:UC-38]{UC-38}, \hyperref[uc:UC-39]{UC-39}}{Il sistema deve poter caricare un test salvato eventualmente sovrascrivendo il test attualmente caricato}

    \subreq{RFF-14.06}{\hyperref[uc:UC-36]{UC-36}}{Il sistema deve poter determinare se esistono test salvati}

\end{freq}

\begin{freq}
    [\dependency{Il sistema deve poter ottenere l'LLM utilizzato nel test}]
    {RFF-15}{Il sistema dovrebbe poter gestire un test salvato}
        \label{rf:RFF-15}%
        
        
        \subreq{RFF-15.01}{\hyperref[uc:UC-36]{UC-36}}{Il sistema deve poter ottenere il nome del test}
                
        \subreq{RFF-15.02}{\hyperref[uc:UC-36]{UC-36}}{Il sistema deve poter ottenere il nome del dataset utilizzato}
        
        \subreq{RFF-15.03}{\hyperref[uc:UC-36]{UC-36}}{Il sistema deve poter ottenere la data di esecuzione del test}
        
        \subreq{RFF-15.04}{\hyperref[uc:UC-41]{UC-41}}{Il sistema deve poter rinominare un test salvato }
        
        \subreq{RFF-15.05}{\hyperref[uc:UC-15]{UC-15}, \hyperref[uc:UC-16]{UC-16}}{Il sistema deve poter determinare se il nome di un test salvato è valido ovvero non è associato a un altro test salvato e non è vuoto o composto da soli spazi}

        \subreq{RFF-15.06}{\hyperref[uc:UC-32]{UC-32}}{Il sistema dato un numero di pagina deve poter determinare i risultati del test che vi appartengono}

        \subreq{RFF-15.07}{\hyperref[uc:UC-31]{UC-31}}{Il sistema deve poter determinare il numero di pagina massima dei risultati del test}

        \subreq{RFF-15.08}{\hyperref[uc:UC-2]{UC-2}}{Il sistema dato un numero di pagina deve poter determinare se appartiene al range di pagine valide}

\end{freq}

\begin{freq}
{RFF-16}{Il sistema dovrebbe poter gestire un insieme di LLM salvati}
    \label{rf:RFF-16}%
    
    \subreq{RFF-16.01}{\hyperref[uc:UC-52]{UC-52}}{Il sistema deve poter ottenere gli LLM salvati}
    
    \subreq{RFF-16.02}{\hyperref[uc:UC-54]{UC-54}}{Il sistema deve poter eliminare un LLM salvato}

    \subreq{RFF-16.03}{\hyperref[uc:UC-50]{UC-50}, \hyperref[uc:UC-51]{UC-51}}{Il sistema deve poter modificare un LLM salvato \hyperref[rf:RFF-17]{RFF-17}}

    \subreq{RFF-16.04}{\hyperref[uc:UC-46]{UC-46}}{Il sistema deve poter salvare un nuovo LLM}

    \subreq{RFF-16.05}{\hyperref[uc:UC-52]{UC-52}}{Il sistema deve poter determinare se esistono LLM salvati}

\end{freq}

\begin{freq}
{RFF-17}{Il sistema dovrebbe poter gestire un LLM salvato}
    \label{rf:RFF-17}%
    
    \subreq{RFF-17.01}{\hyperref[uc:UC-50]{UC-50}}{Il sistema deve poter ottenere/modificare il nome dell'LLM}

    \subreq{RFF-17.02}{\hyperref[uc:UC-15]{UC-15}}{Il sistema deve poter determinare se il nome di un LLM è valido ovvero non è vuoto o composto da soli spazi e non è già assegnato a un altro LLM}
    
    \subreq{RFF-17.03}{\hyperref[uc:UC-53]{UC-53}}{Il sistema deve poter ottenere la data di ultima modifica dell'LLM}
    
    \subreq{RFF-17.04}{\hyperref[uc:UC-51]{UC-51}, \hyperref[uc:UC-53]{UC-53}}{Il sistema deve poter ottenere/modificare l'URL associato all'LLM}

    \subreq{RFF-17.05}{\hyperref[uc:UC-48]{UC-48}}{Il sistema deve poter determinare se un URL è in un formato valido}

    \subreq{RFF-17.06}{\hyperref[uc:UC-51]{UC-51}, \hyperref[uc:UC-53]{UC-53}}{Il sistema deve poter ottenere/modificare le chiavi associate alla domanda e alla risposta}

    \subreq{RFF-17.07}{\hyperref[uc:UC-49]{UC-49}}{Il sistema deve poter determinare se le chiavi associate alla domanda e alla risposta sono valide ovvero non sono vuote o composte da soli spazi}
    
    \subreq{RFF-17.08}{\hyperref[uc:UC-47]{UC-47}}{Il sistema deve poter gestire gli insiemi di coppie chiave-valore associate alla creazione degli header e dei body delle richieste HTTP verso l'LLM \hyperref[rf:RFF-18]{RFF-18}}

\end{freq}

\begin{freq}
{RFF-18}{Il sistema dovrebbe poter gestire un insieme di coppie chiave-valore da utilizzare nella comunicazione HTTP con l'LLM}
    \label{rf:RFF-18}%
    
    
    \subreq{RFF-18.01}{\hyperref[uc:UC-53]{UC-53}}{Il sistema deve poter ottenere le coppie chiave-valore}
    
    \subreq{RFF-18.02}{\hyperref[uc:UC-47]{UC-47}}{Il sistema deve poter aggiungere una coppia chiave-valore}
    
    \subreq{RFF-18.03}{\hyperref[uc:UC-51]{UC-51}}{Il sistema deve poter modificare una coppia chiave-valore}
    
    \subreq{RFF-18.04}{\hyperref[uc:UC-49]{UC-49}}{Il sistema deve poter determinare se una coppia chiave-valore è valida ovvero è composta da chiave e valore non vuoti oppure composti da soli spazi}

\end{freq}

\begin{freq}
{RFF-19}{Il sistema dovrebbe poter determinare la correttezza del formato di un file JSON che specifica un dataset}
    \label{rf:RFF-19}%
    
    
    \subreq{RFF-19.01}{\hyperref[uc:UC-17]{UC-17}}{Il sistema deve verificare che il file JSON contenga una lista di oggetti}

    \subreq{RFF-19.02}{\hyperref[uc:UC-17]{UC-17}}{Il sistema deve verificare che ogni oggetto contenga i due attributi "domanda" e "risposta"} 

\end{freq}

\begin{freq}
{RFF-20}{Il sistema dovrebbe poter gestire il confronto tra il test caricato e un test salvato}
    \label{rf:RFF-20}%
    
    \subreq{RFF-20.01}{\hyperref[uc:UC-42]{UC-42}}{Il sistema deve poter determinare quali test salvati possono essere confrontati con il test caricato, ovvero quelli che condividono il dataset o l'LLM utilizzato nel test}
    
    \subreq{RFF-20.02}{\hyperref[uc:UC-42]{UC-42}}{Il sistema deve poter determinare se il test caricato condivide con un dato test salvato la stessa versione di dataset utilizzato}

\end{freq}