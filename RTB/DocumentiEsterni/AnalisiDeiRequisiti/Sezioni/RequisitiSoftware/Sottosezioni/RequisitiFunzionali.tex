\subsection{Requisiti funzionali}
In questa sezione vengono inserite le funzionalità e i comportamenti che il sistema dovrà avere.

\begin{requisitiFunzionali}
    RF-0 & UC-0 & \hyperref[subsubsec:RF-0]{Specifica RF-0} & Il sistema deve permettere la visualizzazione del dataset corrente \\ \hline
    RF-1 & UC-1 & \hyperref[subsubsec:RF-1]{Specifica RF-1} &  Il sistema deve permettere l'inserimento di una nuova coppia nel dataset corrente \\ \hline
    RF-2 & UC-2 & \hyperref[subsubsec:RF-2]{Specifica RF-2} & Il sistema deve permettere di annullare un'operazione sul dataset corrente  \\ \hline
    RF-3 & UC-3 & \hyperref[subsubsec:RF-3]{Specifica RF-3} &  Il sistema deve mostrare un errore se l'utente tenta di registrare una coppia composta da domanda e risposta vuote \\ \hline
    RF-4 & UC-4 & \hyperref[subsubsec:RF-4]{Specifica RF-4} & Il sistema deve permettere la modifica di una coppia contenuta nel dataset corrente \\ \hline
    RF-5 & UC-5 & \hyperref[subsubsec:RF-5]{Specifica RF-5} & Il sistema deve permettere l'eliminazione di una coppia contenuta nel dataset corrente \\ \hline
    RF-6 & UC-6 & \hyperref[subsubsec:RF-6]{Specifica RF-6} & Il sistema deve permettere la ricerca di un sottoinsieme di coppie del dataset corrente tramite parole chiave \\ \hline
    RF-7 & UC-7 & \hyperref[subsubsec:RF-7]{Specifica RF-7} & Il sistema deve permettere la modifica del nome di un dataset archiviato\\ \hline
    RF-8 & UC-8 & \hyperref[subsubsec:RF-8]{Specifica RF-8} & Il sistema deve mostrare un errore se l'utente tenta di registrare un nome vuoto o già esistente per un dataset\\ \hline
    RF-9 & UC-9 & \hyperref[subsubsec:RF-9]{Specifica RF-9} & Il sistema deve permettere la visualizzazione dei dataset archiviati\\ \hline
    RF-10 & UC-10 & \hyperref[subsubsec:RF-10]{Specifica RF-10} & Il sistema deve permettere la copia di un dataset archiviato\\ \hline
    RF-11 & UC-11 & \hyperref[subsubsec:RF-11]{Specifica RF-11} & Il sistema deve mostrare un errore se il salvataggio delle informazioni di un dataset fallisce \\ \hline
    RF-12 & UC-12 & \hyperref[subsubsec:RF-12]{Specifica RF-12} & Il sistema deve permettere di eliminare un dataset archiviato \\ \hline
    RF-13 & UC-13 & \hyperref[subsubsec:RF-13]{Specifica RF-13} & Il sistema deve permettere di creare un nuovo dataset temporaneo \\ \hline
    RF-14 & UC-14 & \hyperref[subsubsec:RF-14]{Specifica RF-14} & Il sistema deve permettere la ricerca di un sottoinsieme dei dataset archiviati tramite parole chiave \\ \hline
    RF-15 & UC-15 & \hyperref[subsubsec:RF-15]{Specifica RF-15} & Il sistema deve permettere di archiviare il dataset corrente \\ \hline
    RF-16 & UC-16 & \hyperref[subsubsec:RF-16]{Specifica RF-16} & Il sistema deve poter aggiornare la versione archiviata del dataset corrente\\ \hline
    RF-17 & UC-17 & \hyperref[subsubsec:RF-17]{Specifica RF-17} & Il sistema deve mostrare un errore per contenuto invalido del dataset corrente \\ \hline
    RF-18 & UC-18 & \hyperref[subsubsec:RF-18]{Specifica RF-18} & Il sistema deve poter archiviare un dataset corrente temporaneo \\ \hline
    RF-19 & UC-19 & \hyperref[subsubsec:RF-19]{Specifica RF-19} & Il sistema deve permettere di caricare come dataset corrente un dataset precedentemente archiviato\\ \hline
    RF-20 & UC-20 & \hyperref[subsubsec:RF-20]{Specifica RF-20} & Il sistema deve permettere la sovrascrittura del dataset corrente \\ \hline
    RF-21 & UC-21 & \hyperref[subsubsec:RF-21]{Specifica RF-21} & Il sistema deve permettere l'esecuzione del test sul dataset corrente \\ \hline
    RF-22 & UC-22 & \hyperref[subsubsec:RF-22]{Specifica RF-22} & Il sistema deve mostrare un messaggio di errore nel caso in cui la comunicazione con LLM fallisca \\ \hline
    RF-23 & UC-23 & \hyperref[subsubsec:RF-23]{Specifica RF-23} & Il sistema deve permettere la visualizzazione dei risultati di un test \\ \hline
    RF-24 & UC-24 & \hyperref[subsubsec:RF-24]{Specifica RF-24} & Il sistema deve permettere di visualizzare la lista dei risultati di un test \\ \hline
    RF-25 & UC-25 & \hyperref[subsubsec:RF-25]{Specifica RF-25} & Il sistema deve permettere di visualizzare un diagramma di dispersione che riassume i risultati di un test \\ \hline
    RF-26 & UC-26 & \hyperref[subsubsec:RF-26]{Specifica RF-26} & Il sistema deve permettere di interagire con il grafico di dispersione \\ \hline



\end{requisitiFunzionali}

\subsubsection{Specifica RF-0}
\label{subsubsec:RF-0}
\begin{enumerate}
    \item[RF-0.1] Il dataset corrente deve essere mostrato come una lista ordinata per ordine di inserimento
    \item[RF-0.2] La visualizzazione del dataset corrente deve avvenire in una singola schermata
    \item[RF-0.3] Se il dataset corrente è vuoto il sistema deve indicarlo
    \item[RF-0.4] Ogni elemento del dataset corrente deve essere mostrato come una coppia di domanda e risposta 
    \item[RF-0.5] Per ogni elemento del dataset corrente il sistema deve fornire bottoni per permettere la modifica e l'eliminazione della coppia
    \item[RF-0.6] Se il dataset corrente non è temporaneo viene mostrato il suo nome altrimenti viene indicato che è temporaneo
    \item[RF-0.8] Il sistema deve mostrare un bottone per l'esecuzione del test sul dataset corrente
\end{enumerate}

\subsubsection{Specifica RF-1}
\label{subsubsec:RF-1}
\begin{enumerate}
    \item[RF-1.1] Il sistema deve permettere all'utente di specificare una domanda e una risposta per la nuova coppia
    \item[RF-1.2] Il sistema durante la procedura di inserimento deve fornire un bottone per annullare la procedura di inserimento, l'annullamento segue \hyperref[subsubsec:RF-2]{Specifica RF-2}
    \item[RF-1.3] Il sistema durante la procedura di inserimento deve fornire un bottone per confermare la procedura di inserimento
    \item[RF-1.4] Se la domanda e la risposta sono entrambe vuote(composte solo da spazi o senza caratteri) viene visualizzato un errore seguendo \hyperref[subsubsec:RF-3]{Specifica RF-3}
\end{enumerate}

\subsubsection{Specifica RF-2}
\label{subsubsec:RF-2}
\begin{enumerate}
    \item[RF-2.1] L'operazione sul dataset corrente viene annullata
    \item[RF-2.2] L'annullamento di un operazione sul dataset corrente lascia invariate le informazioni a esso associate 
\end{enumerate}

\subsubsection{Specifica RF-3}
\label{subsubsec:RF-3}
\begin{enumerate}
    \item[RF-3.1] Il sistema mostra un messaggio di errore che richiede la compilazione di almeno una tra domanda e risposta
    \item[RF-3.2] Il sistema non interrompe l'operazione che ha generato l'errore 
\end{enumerate}

\subsubsection{Specifica RF-4}
\label{subsubsec:RF-4}
\begin{enumerate}
    \item[RF-4.1] Il sistema deve rendere modificabile la coppia di domanda e risposta per cui è stata richiesta l'operazione di modifica
    \item[RF-4.2] Il sistema deve permettere di salvare la coppia modificata tramite l'utilizzo di un bottone
    \item[RF-4.3] Il sistema deve permettere di annullare l'operazione di modifica tramite l'utilizzo di un bottone, l'annullamento segue \hyperref[subsubsec:RF-2]{Specifica RF-2}
    \item[RF-4.4] Il sistema dopo la conferma deve aggiornare il dataset corrente con la coppia modificata
    \item[RF-4.5] Se la domanda e la risposta sono entrambe vuote(composte solo da spazi o senza caratteri) viene visualizzato un errore seguendo \hyperref[subsubsec:RF-3]{Specifica RF-3}
\end{enumerate}

\subsubsection{Specifica RF-5}
\label{subsubsec:RF-5}
\begin{enumerate}
    \item[RF-5.1] Il sistema deve richiedere la conferma dell'operazione di eliminazione tramite un bottone
    \item[RF-5.2] Il sistema deve permettere l'annullamento dell'operazione di eliminazione tramite un bottone, l'annullamento segue \hyperref[subsubsec:RF-2]{Specifica RF-2}
    \item[RF-5.3] Dopo l'eliminazione della coppia il sistema deve aggiornare il dataset corrente
\end{enumerate}

\subsubsection{Specifica RF-6}
\label{subsubsec:RF-6}
\begin{enumerate}
    \item[RF-6.1] Il sistema deve fornire una barra di ricerca per permettere la ricerca tramite parole chiave 
    \item[RF-6.2] Il sistema deve mostrare il sottoinsieme del dataset corrente composto dalle coppie che hanno domanda e/o risposta che contengono le parole chiave
    \item[RF-6.3] Nel caso in cui il sottoinsieme sia vuoto il sistema deve notificare all'utente che la ricerca non ha prodotto risultati
    \item[RF-6.4] Se non vengono indicate parole chiave la ricerca produce l'intero dataset corrente
\end{enumerate}

\subsubsection{Specifica RF-7}
\label{subsubsec:RF-7}
\begin{enumerate}
    \item[RF-7.1] Il sistema deve fornire la possibilità di modificare il nome di un dataset archiviato 
    \item[RF-7.2] Il sistema deve permettere di salvare il nuovo nome del dataset inserito dall'utente
    \item[RF-7.3] Il sistema deve permettere di annullare l'operazione di modifica del nome del dataset
    \item[RF-7.4] Il sistema a seguito della conferma della modifica deve aggiornare il nome del dataset archiviato
    \item[RF-7.5] Se il nome del dataset è vuoto o esiste già viene indicato all'utente seguendo \hyperref[subsubsec:RF-8]{Specifica RF-8}
    \item[RF-7.6] Se il sistema riscontra errori durante il salvataggio del nuovo nome del dataset deve notificarlo all'utente seguendo \hyperref[subsubsec:RF-11]{Specifica RF-11}
\end{enumerate}

\subsubsection{Specifica RF-8}
\label{subsubsec:RF-8}
\begin{enumerate}
    \item[RF-8.1] Se il nome del dataset è vuoto il sistema deve mostrare un messaggio di errore che richiede la compilazione dello stesso 
    \item[RF-8.2] Se il nome del dataset è già in uso il sistema deve notificarlo all'utente
\end{enumerate}

\subsubsection{Specifica RF-9}
\label{subsubsec:RF-9}
\begin{enumerate}
    \item[RF-9.1] Il sistema deve mostrare una lista dei dataset archiviati ordinata per ordine lessicografico sul loro nome
    \item[RF-9.2] Per ogni elemento il sistema deve mostrare un bottone per copiare il dataset
    \item[RF-9.3] Per ogni elemento il sistema deve mostrare un bottone per eliminare il dataset
    \item[RF-9.4] Per ogni elemento il sistema deve mostrare un bottone per modificare il nome del dataset
    \item[RF-9.5] Per ogni elemento il sistema deve mostrare un bottone per caricare il dataset come dataset corrente 
    \item[RF-9.6] Il sistema deve mostrare un bottone per la creazione di un nuovo dataset temporaneo
\end{enumerate}

\subsubsection{Specifica RF-10}
\label{subsubsec:RF-10}
\begin{enumerate}
    \item[RF-10.1] Il sistema deve permettere di copiare un dataset archiviato selezionato
    \item[RF-10.2] Il sistema deve archiviare una nuova copia del dataset selezionato usando un nome di default che segue lo schema \texttt{nome\_dataset\_selezionato(n)} dove \texttt{n} è un numero naturale incrementale che parte da 1
    \item[RF-10.3] Il sistema dopo l'operazione di copiatura deve aggiornare la lista dei dataset archiviati
    \item[RF-10.4] Se il sistema riscontra errori durante il salvataggio della copia deve notificarlo all'utente seguendo \hyperref[subsubsec:RF-11]{Specifica RF-11}
\end{enumerate}

\subsubsection{Specifica RF-11}
\label{subsubsec:RF-11}
\begin{enumerate}
    \item[RF-11.1] Il sistema notifica l'errore di salvataggio all'utente
\end{enumerate}

\subsubsection{Specifica RF-12}
\label{subsubsec:RF-12}
\begin{enumerate}
    \item[RF-12.1] Il sistema deve permettere di eliminare un dataset archiviato selezionato
    \item[RF-12.2] Il sistema deve chiedere la conferma prima di eliminare in modo definitivo il dataset
    \item[RF-12.3] Il sistema deve permettere di annullare l'eliminazione del dataset
    \item[RF-12.4] Il sistema dopo l'operazione di eliminazione deve aggiornare la lista dei dataset archiviati
    \item[RF-12.5] Se il sistema riscontra errori durante l'eliminazione del dataset selezionato deve notificarlo all'utente tramite un messaggio di errore
\end{enumerate}


\subsubsection{Specifica RF-13}
\label{subsubsec:RF-13}
\begin{enumerate}
    \item[RF-13.1] Il nuovo dataset temporaneo inizialmente è vuoto
    \item[RF-13.2] Il nuovo dataset temporaneo viene caricato come dataset corrente
    \item[RF-13.3] Se esiste un dataset corrente che non possiede una versione aggiornata archiviata e non è vuoto il sistema deve iniziare la procedura di sovrascrittura del dataset corrente seguendo \hyperref[subsubsec:RF-20]{Specifica RF-20}
\end{enumerate}

\subsubsection{Specifica RF-14}
\label{subsubsec:RF-14}
\begin{enumerate}
    \item[RF-14.1] Il sistema deve fornire una barra di ricerca per permettere la ricerca tramite parole chiave dei dataset archiviati
    \item[RF-14.2] Il sistema deve mostrare il sottoinsieme dei dataset archiviati il cui nome contiene le parole chiave, la visualizzazione segue \hyperref[subsubsec:RF-9]{RF-9}
    \item[RF-14.3] Nel caso in cui il sottoinsieme sia vuoto il sistema deve notificare all'utente che la ricerca non ha prodotto risultati
    \item[RF-14.4] Se non vengono indicate parole chiave la ricerca produce l'intera lista dei dataset archiviati, la visualizzazione segue \hyperref[subsubsec:RF-9]{RF-9}
\end{enumerate}


\subsubsection{Specifica RF-15}
\label{subsubsec:RF-15}
\begin{enumerate}
    \item[RF-15.1] Il sistema deve controllare che il dataset corrente non sia vuoto
    \item[RF-15.2] Il sistema deve controllare che tutte le coppie del dataset corrente siano valide ovvero composte da domanda e risposta non vuote.
    Se il dataset corrente contiene almeno una coppia invalida il sistema deve notificare l'errore seguendo \hyperref[subsubsec:RF-17]{Specifica RF-17} e interrompere l'operazione di archiviazione
    \item[RF-15.3] Se il dataset corrente possiede già una versione archiviata il sistema deve aggiornarla seguendo \hyperref[subsubsec:RF-18]{Specifica RF-18}
    \item[RF-15.4] Se il dataset corrente è temporaneo ovvero non possiede una versione archiviata il sistema deve crearla seguendo \hyperref[subsubsec:RF-18]{Specifica RF-18} 
\end{enumerate}

\subsubsection{Specifica RF-16}
\label{subsubsec:RF-16}
\begin{enumerate}
    \item[RF-16.1] Il sistema deve chiedere all'utente la conferma per procedere con l'aggiornamento della versione archiviata del dataset corrente 
    \item[RF-16.2] Il sistema deve permettere di annullare l'aggiornamento della versione archiviata del dataset corrente 
    \item[RF-16.3] Quando l'utente conferma l'operazione di aggiornamento il sistema aggiorna della versione archiviata del dataset corrente 
    \item[RF-16.4] Il sistema deve notificare l'utente del corretto aggiornamento
    \item[RF-16.] Se il sistema riscontra errori durante l'archiviazione del dataset deve notificarlo all'utente seguendo \hyperref[subsubsec:RF-11]{Specifica RF-11}
\end{enumerate}

\subsubsection{Specifica RF-17}
\label{subsubsec:RF-17}
\begin{enumerate}
    \item[RF-17.1] Il sistema notifica l'errore indicando la presenza di coppie invalide ovvero composte da domanda e/o risposta vuota
    \item[RF-17.2] Il sistema evidenzia tutte le coppie invalide
\end{enumerate}

\subsubsection{Specifica RF-18}
\label{subsubsec:RF-18}
\begin{enumerate}
    \item[RF-18.1] Il sistema deve permettere l'inserimento di un nome da assegnare
    \item[RF-18.2] Quando l'utente conferma il nome il sistema deve controllare che il nome inserito sia valido (il nome non è vuoto e non esiste già) altrimenti deve segnalarlo seguendo \hyperref[subsubsec:RF-8]{Specifica RF-8}
    \item[RF-18.3] Se il controllo sul nome ha esito positivo il sistema salva il dataset corrente temporaneo con il nome inserito dall'utente
    \item[RF-18.4] Il sistema deve permettere di annullare il salvataggio del dataset temporaneo
    \item[RF-18.5] Se il sistema riscontra errori durante l'archiviazione del dataset deve notificarlo all'utente seguendo \hyperref[subsubsec:RF-11]{Specifica RF-11}
\end{enumerate}

\subsubsection{Specifica RF-19}
\label{subsubsec:RF-19}
\begin{enumerate}
    \item[RF-19.1] Il sistema deve caricare come dataset corrente il dataset scelto
    \item[RF-19.2] Se il dataset caricato non è vuoto e non possiede una versione aggiornata archiviata, il sistema deve iniziare la procedura di sovrascrittura seguendo \hyperref[subsubsec:RF-20]{Specifica RF-20}
\end{enumerate}

\subsubsection{Specifica RF-20}
\label{subsubsec:RF-20}
\begin{enumerate}
    \item[RF-20.1] Il sistema deve richiedere la conferma di sovrascrittura del dataset corrente
    \item[RF-20.2] Il sistema deve permettere di annullare l'operazione di sovrascrittura
    \item[RF-20.3] In caso di conferma il dataset corrente viene sovrascritto e non è più recuperabile
    \item[RF-20.4] In caso di annullamento il dataset corrente resta invariato 
\end{enumerate}

\subsubsection{Specifica RF-21}
\label{subsubsec:RF-21}
\begin{enumerate}
    \item[RF-21.1] Se il dataset corrente è vuoto il sistema non inizia il test e lo notifica all'utente
    \item[RF-21.2] Il sistema deve controllare che tutte le coppie del dataset corrente siano complete altrimenti deve visualizzare un errore seguendo \hyperref[subsubsec:RF-17]{Specifica RF-17} e interrompere l'esecuzione del test
    \item[RF-21.2] Il sistema deve richiedere le risposte date dal LLM sotto test tramite interrogazioni HTTP seguendo lo standard OpenAPI 3.1 inviando le domande del dataset corrente
    \item[RF-21.3] Se la comunicazione con il LLM produce qualche errore il sistema notifica l'utente spiegando la natura del'errore seguendo \hyperref[subsubsec:RF-22]{Specifica RF-22}
    \item[RF-21.4] Il sistema deve confrontare ogni risposta ottenuta con la risposta attesa producendo due valori:
    \begin{enumerate}
        \item Grado di similarità tra risposta attesa e risposta ricevuta
        \item Valore booleano che indica se la risposta ricevuta è corretta o meno
    \end{enumerate}
    \item[RF-21.5] Il sistema mostra i risultati del test in un'apposita schermata seguendo \hyperref[subsubsec:RF-23]{Specifica RF-23}
\end{enumerate}

\subsubsection{Specifica RF-22}
\label{subsubsec:RF-22}
\begin{enumerate}
    \item[RF-22.1] Il sistema deve notificare l'errore prodotto dal LLM
\end{enumerate}

\subsubsection{Specifica RF-23}
\label{subsubsec:RF-23}
\begin{enumerate}
    \item[RF-23.1] Il sistema deve mostrare una lista dei risultati del test seguendo \hyperref[subsubsec:RF-24]{Specifica RF-24}
    \item[RF-23.2] Il sistema deve mostrare un diagramma di dispersione che riassume i risultati del test seguendo \hyperref[subsubsec:RF-25]{Specifica RF-25}
    \item[RF-23.3] Il sistema deve permettere di visualizzare il dataset caricato
\end{enumerate}

\subsubsection{Specifica RF-24}
\label{subsubsec:RF-24}
\begin{enumerate}
    \item[RF-24.1] Il sistema deve mostrare i singoli risultati del test in una lista ordinata per numero di domanda
    \item[RF-24.2] Il sistema deve rappresentare ogni risultato indicando: domanda, risposta attesa, risposta ottenuta, grado di somiglianza tra le due risposte e il valore booleano di correttezza della risposta ottenuta
\end{enumerate}

\subsubsection{Specifica RF-25}
\label{subsubsec:RF-25}
\begin{enumerate}
    \item[RF-25.1] Il sistema deve mostrare un diagramma di dispersione avente l'asse delle ordinate che rappresenta il numero della
     domanda e l'asse delle ascisse che rappresenta il grado di somiglianza tra la risposta attesa e la risposta ottenuta
    \item[RF-25.2] Il sistema deve rappresentare i singoli risultati del test come punti nel diagramma di dispersione 
    \item[RF-25.3] Il sistema deve calcolare la media dei gradi di similarità risultanti dal test
    \item[RF-25.4] Il sistema deve calcolare la deviazione standard dei gradi di similarità risultanti dal test
    \item[RF-25.5] Il sistema deve rappresentare la media nel diagramma come una linea retta
    \item[RF-25.6] Il sistema deve rappresentare la deviazione standard nel diagramma come un linea retta
    \item[RF-25.7] Il sistema deve rappresentare i punti in modo differente in base al valore booleano che indica la correttezza della risposta ottenuta dal LLM
    \item[RF-25.8] Il sistema deve permettere di interagire con il diagramma seguendo \hyperref[subsubsec:RF-26]{Specifica RF-26}
    \item[RF-25.9] Il sistema deve fornire a fianco al diagramma una legenda che spiega la sua composizione
\end{enumerate}

\subsubsection{Specifica RF-26}
\label{subsubsec:RF-26}
\begin{enumerate}
    \item[RF-26.1] Il sistema deve permettere di interagire con i singoli punti del diagramma di dispersione
    \item[RF-26.2] Il sistema a seguito di un interazione con un punto, deve evidenziare l'elemento della lista dei risultati corrispondente al punto coinvolto nell'interazione
\end{enumerate}




