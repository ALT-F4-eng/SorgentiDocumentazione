

\subsection{Introduzione}
In questa sezione vengono elencati i requisiti software trovati dal gruppo per la realizzazione del progetto ArtificalQI.
I requisiti sono stati trovati tramite l'analisi degli use case individuati dal gruppo, dall'analisi del capitolato fornito dall'azienda e dalle riunioni esterne.
Ogni requisito software ha una nomenclatura composta come segue:
\begin{lstlisting}
    R[Tipologia]-[ID_RequisitoSoftware]
\end{lstlisting}
Dove:
\begin{enumerate}
    \item \lstinline|R| stà per requisito
    \item \lstinline|Tipologia| si divide in tre categorie
    \begin{enumerate}
        \item Funzionali (F): sono i requisiti software che soddisfano i comportamenti o le funzionalità del sistema.
        \item Qualitativi (Q): sono i requisiti software che servono per soddisfare gli standard qualitativi e di manutenibilità
        del prodotto software.
        \item Di Vincolo/Dominio (V): sono i requisiti che delineano le restrizione che gli Stakeholders o il capitolato impongono
         per la realizzazione del progetto software.
    \end{enumerate}
    \item \lstinline|ID_requisito| è un numero progressivo a partire da 0 che individua il requisito software.
\end{enumerate}  
Per ogni requisito software viene indicata una specifica che definisce i sotto requisiti di cui esso è composto.
L'identificazione dei sotto requisiti segue la nomeclatura:
\begin{lstlisting}
R[Tipologia]-[ID_RequisitoSoftware].[ID_SottoRequisito]
\end{lstlisting} 
Dove \lstinline|ID_SottoRequisito| è un numero progressivo a partire da 0 che identifica il sotto requisito.