
\subsection{Introduzione}
\label{sec:introduzione_requisiti_software}
In questa sezione vengono elencati i requisiti software individuati dal gruppo per la realizzazione del progetto ArtificialQI.
I requisiti sono stati definiti tramite l'analisi degli use case individuati dal gruppo, l'analisi del capitolato fornito dall'azienda e le riunioni esterne.
Ogni requisito software ha una nomenclatura composta come segue:
\begin{lstlisting}
    R[Tipologia]-[ID_RequisitoSoftware]
\end{lstlisting}
Dove:
\begin{enumerate}
    \item \lstinline|R| sta per requisito.
    \item \lstinline|Tipologia| si suddivide in tre categorie:
    \begin{enumerate}
        \item Funzionali (F): requisiti software che soddisfano i comportamenti o le funzionalità del sistema.
        \item Qualitativi (Q): requisiti software necessari per garantire standard qualitativi e di manutenibilità
        del prodotto software.
        \item Di Vincolo/Dominio (V): requisiti che delineano le restrizioni imposte dagli \glossario{Stakeholder} o dal capitolato 
        per la realizzazione del progetto software.
    \end{enumerate}
    \item \lstinline|ID_Requisito| è un numero progressivo a partire da 0 che identifica il requisito software.
\end{enumerate}  
Per ogni requisito software viene indicata una specifica che definisce i sotto-requisiti di cui esso è composto.
L'identificazione dei sotto-requisiti segue la nomenclatura:
\begin{lstlisting}
R[Tipologia]-[ID_RequisitoSoftware].[ID_SottoRequisito]
\end{lstlisting} 
Dove \lstinline|ID_SottoRequisito| è un numero progressivo a partire da 0 che identifica il sotto-requisito.
