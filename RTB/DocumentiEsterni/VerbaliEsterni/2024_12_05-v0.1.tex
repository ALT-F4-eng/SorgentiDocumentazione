
\documentclass[a4paper, 12pt]{article}

\usepackage{custom}


%--------------------VARIABILI--------------------
\def\lastversion{v0.1}
\def\title{Verbale Esterno Zucchetti}
\def\date{5 dicembre 2024}
%------------------------------------------------

\begin{document}

\primapagina

\begin{registromodifiche}
        v0.1 & 14 dicembre 2024  & Guirong Lan & Enrico Bianchi & prima stesura del documento:\hyperref[sec:Domande]{inserimento delle domande}, \hyperref[sec:Conclusioni]{inserimento delle conclusioni}\\
    \hline 
\end{registromodifiche}

\tableofcontents

\newpage

\section{Registro presenze}
\begin{itemize}
    \item[] \textbf{Data}: \date
    \item[] \textbf{Ora inizio}:  09:30
    \item[] \textbf{Ora fine}: 10:30
    \item[] \textbf{Piattaforma}: Google Meet	
\end{itemize}
\begin{table}[!h]
\centering
{\renewcommand{\arraystretch}{2}
\begin{tabularx}{\textwidth}{| X | X |}
    \hline
        \textbf{\large Componente} & 
        \textbf{\large Presenza} \\ 
    \hline 
    \hline
        Eghosa Matteo Igbinedion Osamwonyi&
        Presente \\
    \hline 
        Guirong Lan&
        Presente \\
    \hline 
        Enrico Bianchi&
        Presente \\
    \hline 
        Francesco Savio&
        Presente \\
    \hline 
        Marko Peric&
        Presente \\
    \hline 
        Pedro Leoni&
        Non Presente \\
    \hline 

\end{tabularx}}
\end{table}

\begin{table}[!h]
    \centering
    {\renewcommand{\arraystretch}{2}
    \begin{tabularx}{\textwidth}{| X | X |}
        \hline
            \textbf{\large Nome} & 
            \textbf{\large Ruolo} \\ 
        \hline 
        \hline
            Gregorio Piccoli&
            Rappresentante dell'azienda \\
        \hline 
    
    \end{tabularx}}
\end{table}


\newpage

\section{Domande}
\label{sec:Domande}
Di seguito vengono riportate le domande fatte dal gruppo:
\begin{enumerate}
    \item Il sistema deve prevedere un metodo di autenticazione?
    \item È necessario includere un sistema di navigazione (dashboard/home) per gestire le varie operazioni, come ad esempio l'inserimento di un file contenente un set di coppie di domande e risposte?
    \item Il caricamento di un file contenente il set di coppie di domande e risposte è un requisito opzionale?
    \item Il sistema deve prevedere l'inserimento, la creazione, la modifica, la ricerca, l'archiviazione, la visualizzazione e la cancellazione delle coppie di domande e risposte. Oltre a queste, ci sono altri requisiti obbligatori?
    \item È sufficiente un \textit{database$_{G}$} locale per una web app?
    \item L'utente deve poter scegliere dall'elenco delle domande/risposte quali testare, o si verifica tutto il set durante la fase di \textit{test$_{G}$}?
    \item Diamo per scontato che le domande possano essere di qualsiasi tipo, o possiamo applicare dei filtri?
    \item Esiste un modo per migliorare il risultato degli \textit{embedding$_{G}$} quando le frasi sono simili ma hanno significati diversi?
    \item Ha metodi di confronto innovativi da segnalarci?
\end{enumerate}
\section{Conclusioni}
\label{sec:Conclusioni}
Il gruppo dalle risposte date dai rappresentanti dell’azienda ha tratto le seguenti conclusioni:
\begin{itemize}
    \item Il fulcro del problema è il confronto tra le coppie di domande e risposte. 
    \item Avere una dashboard per visualizzare i risultati dei test è altamente consigliato, anche se quest’ultima rappresenta un requisito opzionale.
    \item Ci servirà un metodo di archiviazione e caricamento, oltre alle solide procedure come la creazione, la visualizzazione, la ricerca e così via. Insomma, si tratta di un classico programma di gestione dell’archivio per rendere funzionale il sistema: non deve fare cose straordinarie, ma garantire un minimo di funzionalità essenziale, come anche per la parte estetica.
    \item Non è necessario disporre di un database remoto o online; la decisione spetta al gruppo di sviluppo se adottarne uno o meno. Si consiglia di utilizzare gli strumenti con cui ci si sente più a proprio agio.
    \item Avere un filtro può certamente agevolare lo sviluppo, pertanto è un requisito opzionale. Nel caso si decida di svilupparlo, spetta sempre al gruppo decidere se applicarlo durante o prima dell'avvio dei test. Inoltre, la possibilità di caricare un insieme di piccoli file di testo (CSV o Markdown) e trasformarli in coppie di domande e risposte è anch'essa un requisito opzionale.
    \item Tra le tecnologie consigliate da consultare ci sono: Nomic (embedding per la maggiore), Bertino embedding e BGE 5 embedding. Inoltre, si può considerare l'uso della \textit{Non-Negativity Matrix Factorization$_{G}$} e degli LLM con prompt specifici. 
    \item Tuttavia, è consigliato che il gruppo provi a testare le nuove tecnologie per valutarne l'efficacia.
    \item Il gruppo può consultare il sito Lmarena.ai per avere una migliore idea per lo sviluppo del software.
\end{itemize}
\vfill
{\renewcommand{\arraystretch}{2}
\begin{tabular}{l p{5cm}}
    Data: &  \hrulefill \\
    Firma: & \hrulefill \\
\end{tabular}
}
\end{document}