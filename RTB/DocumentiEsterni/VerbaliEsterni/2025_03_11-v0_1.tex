
\documentclass[a4paper, 12pt]{article}

\usepackage{custom}


%--------------------VARIABILI--------------------
\def\lastversion{v0.1}
\def\title{Verbale Esterno Zucchetti}
\def\date{11 Marzo 2025}
%------------------------------------------------

\begin{document}

\primapagina

\begin{registromodifiche}
        v0.1 & 12 Marzo 2025  & Guirong Lan &  & stesura del verbale\\
    \hline 
\end{registromodifiche}

\tableofcontents

\newpage

\section{Registro presenze}
\begin{itemize}
    \item[] \textbf{Data}: \date
    \item[] \textbf{Ora inizio}:  17:00
    \item[] \textbf{Ora fine}: 17:30
    \item[] \textbf{Piattaforma}: Google Meet	
\end{itemize}
\begin{table}[H]
\centering
{\renewcommand{\arraystretch}{2}
\begin{tabularx}{\textwidth}{| X | X |}
    \hline
        \textbf{\large Componente} & 
        \textbf{\large Presenza} \\ 
    \hline 
    \hline
        Eghosa Matteo Igbinedion Osamwonyi&
        Presente \\
    \hline 
        Guirong Lan&
        Presente \\
    \hline 
        Enrico Bianchi&
        Presente \\
    \hline 
        Francesco Savio&
        Presente \\
    \hline 
        Marko Peric&
        Presente \\
    \hline 
        Pedro Leoni&
        Presente \\
    \hline 

\end{tabularx}}
\end{table}

\begin{table}[H]
    \centering
    {\renewcommand{\arraystretch}{2}
    \begin{tabularx}{\textwidth}{| X | X |}
        \hline
            \textbf{\large Nome} & 
            \textbf{\large Ruolo} \\ 
        \hline 
        \hline
            Gregorio Piccoli&
            Rappresentante dell'azienda \\
        \hline 
    
    \end{tabularx}}
\end{table}


\newpage

\section{Domande}
\label{sec:Domande}
Di seguito vengono riportate le domande fatte dal gruppo:
\begin{enumerate}
    \item Quali parti del prodotto finale devono essere conservate e mantenute nel tempo?
    \item Qual è il tempo massimo di risposta desiderato?
    \item Quale percentuale di copertura dei test si vuole raggiungere?
    \item Quali ambienti di esecuzione devono essere supportati (browser, sistemi operativi, ecc.)?
    \item In che misura il prodotto deve rispettare le normative sull’accessibilità?
    \item Qual è il numero massimo di clic accettabile per raggiungere una pagina partendo dalla home?
    \item Quanto deve essere portabile il prodotto? Deve funzionare su telefoni, tablet e altri dispositivi simili?
\end{enumerate}
\section{Conclusioni}
\label{sec:Conclusioni}
Il gruppo dalle risposte date dai rappresentanti dell’azienda ha tratto le seguenti conclusioni:
\begin{itemize}
    \item La principale area di miglioramento riguarda la valutazione della pertinenza tra la risposta attesa e quella ricevuta. Inoltre, è fondamentale concentrarsi sui compiti piuttosto che sulle singole funzionalità. 
    \item La scelta degli ambienti di esecuzione e il raggiungimento della percentuale di copertura dei test sono responsabilità del gruppo, eventualmente seguendo le indicazioni del Prof. Cardin per la definizione della soglia minima della copertura dei test. 
    \item Il tempo massimo di risposta desiderato deve stare circa 500 domande all’ora. 
    \item Per quanto riguarda l’accessibilità, spetta al gruppo decidere fino a che livello implementarla. Ad esempio, il contrasto dei colori è un aspetto facilmente gestibile. 
    \item Infine, il progetto deve essere un sistema unico e non frammentato. Per questo motivo, il programma dovrà essere contenuto all’interno di un \glossario{container}.  
\end{itemize}
\vfill
{\renewcommand{\arraystretch}{2}
\begin{tabular}{l p{5cm}}
    Data: &  \hrulefill \\
    Firma: & \hrulefill \\
\end{tabular}
}
\end{document}